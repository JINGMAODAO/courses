\documentclass[letterpaper,12pt,oneside,onecolumn]{article}
\usepackage[margin=1in, bottom=1in, top=1in]{geometry} %1 inch margins
\usepackage{amsmath, amssymb, amstext}
\usepackage{fancyhdr}
\usepackage{mathtools}
\usepackage{algorithm}
\usepackage{algpseudocode}
\usepackage{theorem}
\usepackage{tikz}
\usepackage{tkz-berge}

%Macros
\newcommand{\A}{\mathbb{A}} \newcommand{\C}{\mathbb{C}}
\newcommand{\D}{\mathbb{D}} \newcommand{\F}{\mathbb{F}}
\newcommand{\N}{\mathbb{N}} \newcommand{\R}{\mathbb{R}}
\newcommand{\T}{\mathbb{T}} \newcommand{\Z}{\mathbb{Z}}
\newcommand{\Q}{\mathbb{Q}}
 
 
\newcommand{\cA}{\mathcal{A}} \newcommand{\cB}{\mathcal{B}}
\newcommand{\cC}{\mathcal{C}} \newcommand{\cD}{\mathcal{D}}
\newcommand{\cE}{\mathcal{E}} \newcommand{\cF}{\mathcal{F}}
\newcommand{\cG}{\mathcal{G}} \newcommand{\cH}{\mathcal{H}}
\newcommand{\cI}{\mathcal{I}} \newcommand{\cJ}{\mathcal{J}}
\newcommand{\cK}{\mathcal{K}} \newcommand{\cL}{\mathcal{L}}
\newcommand{\cM}{\mathcal{M}} \newcommand{\cN}{\mathcal{N}}
\newcommand{\cO}{\mathcal{O}} \newcommand{\cP}{\mathcal{P}}
\newcommand{\cQ}{\mathcal{Q}} \newcommand{\cR}{\mathcal{R}}
\newcommand{\cS}{\mathcal{S}} \newcommand{\cT}{\mathcal{T}}
\newcommand{\cU}{\mathcal{U}} \newcommand{\cV}{\mathcal{V}}
\newcommand{\cW}{\mathcal{W}} \newcommand{\cX}{\mathcal{X}}
\newcommand{\cY}{\mathcal{Y}} \newcommand{\cZ}{\mathcal{Z}}

\newcommand\numberthis{\addtocounter{equation}{1}\tag{\theequation}}


\newenvironment{proof}{{\bf Proof:  }}{\hfill\rule{2mm}{2mm}}
\newenvironment{proofof}[1]{{\bf Proof of #1:  }}{\hfill\rule{2mm}{2mm}}
\newenvironment{proofofnobox}[1]{{\bf#1:  }}{}\newenvironment{example}{{\bf Example:  }}{\hfill\rule{2mm}{2mm}}

%\renewcommand{\thesection}{\lecnum.\arabic{section}}
%\renewcommand{\theequation}{\thesection.\arabic{equation}}
%\renewcommand{\thefigure}{\thesection.\arabic{figure}}

\newtheorem{fact}{Fact}[section]
\newtheorem{lemma}[fact]{Lemma}
\newtheorem{theorem}[fact]{Theorem}
\newtheorem{definition}[fact]{Definition}
\newtheorem{corollary}[fact]{Corollary}
\newtheorem{proposition}[fact]{Proposition}
\newtheorem{claim}[fact]{Claim}
\newtheorem{exercise}[fact]{Exercise}
\newtheorem{note}[fact]{Note}
\newtheorem{conjecture}[fact]{Conjecture}

\newcommand{\size}[1]{\ensuremath{\left|#1\right|}}
\newcommand{\ceil}[1]{\ensuremath{\left\lceil#1\right\rceil}}
\newcommand{\floor}[1]{\ensuremath{\left\lfloor#1\right\rfloor}}

%END MACROS
%Page style
\pagestyle{fancy}

\listfiles

\raggedbottom

\lhead{2017-01-25}
\rhead{W. Justin Toth: Optimization Seminar Helper Document} %CHANGE n to ASSIGNMENT NUMBER ijk TO COURSE CODE
\renewcommand{\headrulewidth}{1pt} %heading underlined
%\renewcommand{\baselinestretch}{1.2} % 1.2 line spacing for legibility (optional)

\begin{document}
\section*{Notation}
$\mathbf{G=(A\cup B, E)}$ will denote a bipartite graph with vertex partitions $A$ and $B$. The vertex set of $G$ is denoted $\mathbf{V(G)}$, and the edge set $\mathbf{E(G)}$.
\\
$\mathbf{\delta(v)} = \{e \in E(G) : v \in e\}$ for all $v \in V(G)$.
\\
$\mathbf{N(v)} = \{w \in V(G): vw \in E(G)\}$ for all $V\in V(G)$.
\\
$\mathbf{>_v}$ denotes the strict total preference order of $v$ over $N(v)$. By convention $v$ prefers being matched to unmatched.
\\
$\mathbf{\delta^{>w}(v)} = \{vu \in \delta(v) : u >_v w\}$ for all $v,w \in V(G)$.
\\
$\mathbf{N_{max}(v)} = w$ if and only if $w \geq_v u$ for all $u \in N(v)$. Similarly $\mathbf{N_{min}(v)} = w$ if and only if $w \leq_v u$ for all $u \in N(v)$.
\\
Commonly $\mathbf{M} \subseteq E$ such that $|\delta(v) \cap M| \leq 1$ for all $v \in V$ denotes a matching, and $\mathbf{M(v)} = w $ if and only if $vw \in M$. 
\\
For matchings $M_1$ and $M_2$, $\mathbf{M_1 \Delta M_2} = (M_1 \cup M_2) \backslash (M_1 \cap M_2)$ denotes their symmetric difference
\\
$\mathbf{P(G)}$ denotes the set of $x \in \R^{E(G)}$ satisfying
\begin{align*}
 x(\delta(v)) &\leq 1 &\forall v \in V(G) \numberthis \\
x(\delta^{>b}(a)) + x(\delta^{>a}(b)) + x_{ab} &\geq 1 &\forall ab \in E(G) \numberthis \\
x &\geq 0\numberthis.
\end{align*}
Where $\mathbf{x(M) }= \sum_{e \in M} x_e$ for $M \subseteq E(G)$ and $x \in \R^{E(G)}$.
\\
We let $\mathbf{\chi(M)} \in \{0,1\}^{E(G)}$ be the characteristic vector of $M$, defined by $\chi(M)_e = 1$ if $e\in M$ and $\chi(M)_e = 0$ otherwise.
\\
For $x \in P(G)$ we let $\mathbf{V_x}$ denote a set of vertices corresponding to constraints of the form $(1)$ satisfied at equality by $x$.
\\Similarly $\mathbf{E_x}$ denotes a set of constraints of the form $(2)$ satisfied at equality by $x$.
\\Together $|E_x| + |V_x| = |E|$ and the constraints corresponding to $E_x$ and $V_x$ are linearly independent.
\section*{Outline (of proof $P(G)$ is integral)}
\begin{enumerate}
\item Show there exists $e \in E$ for which $x_e \in \{0,1\}$ where $x$ is an extreme point of $P(G)$.
\begin{enumerate}
\item Contradict that number of variables $|E| = |E_x| + |V_x|$ for appropriately chosen $E_x,V_x$ when $0 < x < 1$.
\end{enumerate}
\item Proceed by induction on $|E(G)|$: for graphs $G'$ with $|E(G')| < |E(G)|$, $P(G')$ has integral extreme points.
\item Case $x_{ab} = 0$
\begin{enumerate}
\item If $x' = x\mid_{E-ab}$ is an extreme point of $P(G-ab)$ then done. 
\item Otherwise: (Hard Case, if there's time) using Stable Matchings, and Geometry.
 \end{enumerate}
\item Case $x_{ab} = 1$ reducing to case $x_{ab} = 0$ or simply induction if no $0$ variables.
\end{enumerate}
\end{document}
