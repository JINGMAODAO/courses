\documentclass[letterpaper,12pt,oneside,onecolumn]{article}
\usepackage[margin=1in, bottom=1in, top=1in]{geometry} %1 inch margins
\usepackage{amsmath, amssymb, amstext}
\usepackage{fancyhdr}
\usepackage{mathtools}
\usepackage{algorithm}
\usepackage{algpseudocode}
\usepackage{theorem}
\usepackage{tikz}
\usepackage{tkz-berge}

%Macros
\newcommand{\A}{\mathbb{A}} \newcommand{\C}{\mathbb{C}}
\newcommand{\D}{\mathbb{D}} \newcommand{\F}{\mathbb{F}}
\newcommand{\N}{\mathbb{N}} \newcommand{\R}{\mathbb{R}}
\newcommand{\T}{\mathbb{T}} \newcommand{\Z}{\mathbb{Z}}
\newcommand{\Q}{\mathbb{Q}}
 
 
\newcommand{\cA}{\mathcal{A}} \newcommand{\cB}{\mathcal{B}}
\newcommand{\cC}{\mathcal{C}} \newcommand{\cD}{\mathcal{D}}
\newcommand{\cE}{\mathcal{E}} \newcommand{\cF}{\mathcal{F}}
\newcommand{\cG}{\mathcal{G}} \newcommand{\cH}{\mathcal{H}}
\newcommand{\cI}{\mathcal{I}} \newcommand{\cJ}{\mathcal{J}}
\newcommand{\cK}{\mathcal{K}} \newcommand{\cL}{\mathcal{L}}
\newcommand{\cM}{\mathcal{M}} \newcommand{\cN}{\mathcal{N}}
\newcommand{\cO}{\mathcal{O}} \newcommand{\cP}{\mathcal{P}}
\newcommand{\cQ}{\mathcal{Q}} \newcommand{\cR}{\mathcal{R}}
\newcommand{\cS}{\mathcal{S}} \newcommand{\cT}{\mathcal{T}}
\newcommand{\cU}{\mathcal{U}} \newcommand{\cV}{\mathcal{V}}
\newcommand{\cW}{\mathcal{W}} \newcommand{\cX}{\mathcal{X}}
\newcommand{\cY}{\mathcal{Y}} \newcommand{\cZ}{\mathcal{Z}}

\newcommand\numberthis{\addtocounter{equation}{1}\tag{\theequation}}


\newenvironment{proof}{{\bf Proof:  }}{\hfill\rule{2mm}{2mm}}
\newenvironment{proofof}[1]{{\bf Proof of #1:  }}{\hfill\rule{2mm}{2mm}}
\newenvironment{proofofnobox}[1]{{\bf#1:  }}{}\newenvironment{example}{{\bf Example:  }}{\hfill\rule{2mm}{2mm}}

%\renewcommand{\thesection}{\lecnum.\arabic{section}}
%\renewcommand{\theequation}{\thesection.\arabic{equation}}
%\renewcommand{\thefigure}{\thesection.\arabic{figure}}

\newtheorem{fact}{Fact}[section]
\newtheorem{lemma}[fact]{Lemma}
\newtheorem{theorem}[fact]{Theorem}
\newtheorem{definition}[fact]{Definition}
\newtheorem{corollary}[fact]{Corollary}
\newtheorem{proposition}[fact]{Proposition}
\newtheorem{claim}[fact]{Claim}
\newtheorem{exercise}[fact]{Exercise}
\newtheorem{note}[fact]{Note}
\newtheorem{conjecture}[fact]{Conjecture}

\newcommand{\size}[1]{\ensuremath{\left|#1\right|}}
\newcommand{\ceil}[1]{\ensuremath{\left\lceil#1\right\rceil}}
\newcommand{\floor}[1]{\ensuremath{\left\lfloor#1\right\rfloor}}

%END MACROS
%Page style
\pagestyle{fancy}

\listfiles

\raggedbottom

\lhead{\today}
\rhead{William Justin Toth CS798-Convexity and Optimization Assignment 3} %CHANGE n to ASSIGNMENT NUMBER ijk TO COURSE CODE
\renewcommand{\headrulewidth}{1pt} %heading underlined
%\renewcommand{\baselinestretch}{1.2} % 1.2 line spacing for legibility (optional)

\begin{document}
%Question 1
\section{}
Let $K \subseteq R^n$ be a convex body with $rB^n_2 \subseteq K \subseteq RB^n_2$ for some $0 < r < R$. Suppose there exists a separation oracle for $K$. Consider the following algorithm, $\cA$:
\begin{enumerate}
\item Start with $\cE = RB^n_2$. Throughout we use $v_1, \dots, v_n$ to denote the semi-axes of $\cE$, $\lambda_1, \dots, \lambda_n$ the respective side-lengths, and $c$ the centre of $\cE$.
\item Repeat:
	\begin{enumerate}
	\item For each $p \in \{ c+ \frac{v_i}{n+1} : i \in [n]\} \cup \{c-\frac{v_i}{n+1} : i \in [n]\}$:
		\begin{enumerate}
		\item Call separation oracle on $p$.
		\item If $p \not \in K$ let $v^Tx = b$ be the hyperplane returned by the oracle (with $v^Tp > b$). ``Shrink" $\cE$ with respect to $v^Tx=b$ (to be detailed) and $\cE$ to be the new ellipsoid from shrinking. Continue to the next iteration of Step $2$.
		\end{enumerate}
	\item Return $\cE$ (each $p \in K$).
	\end{enumerate}
\end{enumerate}
\paragraph{Shrinking Process}
As in the ellipsoid method, given some ellipsoid $\cE_t$ (with $K \subseteq \cE_t$) and a hyperplane $v^Tx = b$ we want to find a new ellipsoid $\cE_{t+1}$ such that $\cE \cap \{x : v^Tx \leq b\} \subseteq \cE_{t+1}$. Critically we want $vol(\cE_{t+1}) \leq O(\exp(\frac{-1}{n}))  vol(\cE_t)$. As we have seen in Lecture $15$, via an affine change of variables this problem can be reduced to one where $\cE_t$ is the unit ball centred at the origin. We use the same change of variables here and detail the process in the special case below. Although we will get a different estimate on the volume this reduction is still valid since the critical point is that the volume proportions are invariant under an affine map.
\paragraph{}
As in the lecture notes, we will use $\cB = \cE_t$ to denote the origin-centered unit ball. We may assume that $||v|| =1$ (we use $2$-norm throughout). We will use $\cE$ to denote $\cE_{t+1}$. We will describe $\cE$ as 
$$\cE := \{x \in \R^n : (x-c)^TH^{-1}(x-c) \leq 1\}.$$
The centre $c$ will be at $-\alpha v$ for some $\alpha \in [0,1]$. We will use $v$ as one semi-axes, while the others shall be orthogonal to $v$ of the same side length. Hence we have
$$H^{-1} = avv^T + b(I - vv^T)$$
where $a$ is the inverse side length square of $v$, and $b$ is the inverse side length square of the other semi-axes. To ensure $\cE$ satisfies the desired containment, and yet is as small as possible we want the boundary of $\cE$ to touch $-v$ and the points on the boundary of $B$ that intersect the hyperplane $v^Tx = b$. The first condition yields
$$(-v+\alpha v)^T(avv^T + b(I-vv^T))(-v+\alpha v) = 1$$
which implies
$$a = \frac{1}{(\alpha-1)^2}.$$
The second condition implies that for all $y$ orthogonal to $v$, with $||y + \frac{n}{n+1}v || =1$ we have
$$(y + (\frac{n}{n+1} + \alpha)v)^T(avv^T + b(I-vv^T))(y+(\frac{n}{n+1} + \alpha)v) = 1.$$
This condition implies
$$ b = \frac{1-\beta^2a}{||y||^2}$$
where $\beta = \frac{n}{n+1} + \alpha.$ Note that we may use the pythagorean theorem to find $||y||$:
$$||y||^2 =\frac{2n+1}{(n+1)^2}.$$ 
We hope to bound the ratio of $vol(\cE)$ to $vol(B)$. We have the equation
$$\frac{vol(\cE)}{vol(\cB)} = \frac{1}{\sqrt{ab^{n-1}}} = \frac{||v||^{n-1}}{\sqrt{a(1-\beta^2a)^{n-1}}}.$$
So we want to maximize
$$f(\alpha) = a(1-\beta^2a)^{n-1}.$$
Simplifying $1-\beta^2a$ we obtain
$$1-\beta^2a = \frac{2n+1}{(n+1)^2}(\frac{2(n+1)}{1-\alpha} - \frac{2n+1}{(1-\alpha)^2}.$$
Setting $h = \frac{1}{1-\alpha}$ this gives
$$1-\beta^2a = \frac{2(n+1)}{((n+1)^2}(2(n+1)h - (2n+1)h^2).$$
Thus $f$ may be rewritten in terms of $h$ as
$$f(h) = (\frac{2n+1}{(n+1)^2})^{n-1}[h^2(2(n+1)h - (2n+1)h^2)^{n-1}].$$
To maximize $f$ we take the derivate and set $f'(h) = 0$. Solving for $h$ yields
$$h = \frac{n+1}{2n+1}.$$
Such $h$ allows us to find $\alpha$, $a$, and $b$. Thus we have found our ellipse. Now we need to verify the volume. Plugging $h$ into $f$ yields
$$f(h) = (\frac{1}{2n+1})^{n-1}(\frac{4n+3}{2n+1})^{n-1}.$$
Therefore
\begin{align*}
\frac{f(h)}{||y||^{2(n-1)}} &= (\frac{1}{2n+1})^{n-1}(\frac{4n+3}{2n+1})^{n-1}(\frac{(n+1)^2}{2n+1})^{n-1} \\
&=(\frac{(n+1)^2(4n+3)}{(2n+1)^3})^{n-1} \\
&= (\frac{4m^3 - m^2}{(2m-1)^3})^{m-2} &\text{where $m=n+1$} \\
&\geq (\frac{1}{2} - \frac{1}{8m})^{m-2} \\
&\geq (\frac{1}{2}- \frac{1}{8(n-1)})^{n-1} \\
&\geq exp(\frac{1}{8(n-1)}).
\end{align*}
Therefore $\frac{vol(\cE)}{vol(\cB)} = O(exp(\frac{-1}{n})).$
\paragraph{Runtime of Algorithm}
Now we bound the oracle calls used by $\cA$. Each iteration of Step $2$ used $O(n)$ oracle calls in Step $2a$. The question now is how many times Step $2$ iterates. Let $\cE_t$ be the current ellipsoid at iteration $t$ of Step $2$. By the argument above (in big-$O$),
$$vol(\cE_{t+1}) \leq exp(\frac{-1}{n}) vol(\cE_t).$$
Let $t$ be the last iteration of $\cA$. With $\cE_0 = RB^n_2$ we have by induction
$$vol(\cE_t) \leq exp(\frac{-1}{n})^t vol(RB^n_2).
$$Since $rB^n_2 \subseteq K$, $vol(rB^n_2) \leq vol(\cE_t)$. Thus we have
$$vol(rB^n_2) \leq exp(\frac{-1}{n})^t vol(RB^n_2)$$
Solving for $t$ we have
$$\frac{t}{n} = \log(\frac{R}{r})^n$$
and so 
$$t = O(n^2\log(\frac{R}{r})).$$
Therefore the total number of oracle calls in $\cA$ is $O(n^3\log(\frac{R}{r}))$.
\paragraph{Correctness}
By our shrinking procedure, $K \subseteq \cE$. When $\cA$ terminates we have that
$c\pm \frac{v_i}{n+1} \in K$ for all $i \in [n]$. Let $C$ be the convex hull of $\{c\pm \frac{v_i}{n+1} : i \in [n]\}$. So $C \subseteq K$. From the definition of $C$, we have that $C$ is a symmetric convex body. Since the vertices of $C$ are semiaxes endpoints of $\cE' := c + \frac{1}{n+1}(\cE - c)$, we have that $\cE'$ is the minimum volume ellipsoid containing $C$. Thus by John's theorem, $c + \frac{1}{\sqrt{n}}(\cE'-c) \subseteq C$. Hence 
$$c + \frac{1}{\sqrt{n}(n+1)}(\cE  - c) \subseteq C \subseteq K \subseteq \cE$$
as desired. $\blacksquare$
%Question 2
\section{}

%Question 3
\section{}
\end{document}
