\documentclass[letterpaper,12pt,oneside,onecolumn]{article}
\usepackage[margin=1in, bottom=1in, top=1in]{geometry} %1 inch margins
\usepackage{amsmath, amssymb, amstext}
\usepackage{fancyhdr}
\usepackage{mathtools}
\usepackage{algorithm}
\usepackage{algpseudocode}
\usepackage{theorem}
\usepackage{tikz}
\usepackage{tkz-berge}

%Macros
\newcommand{\A}{\mathbb{A}} \newcommand{\C}{\mathbb{C}}
\newcommand{\D}{\mathbb{D}} \newcommand{\F}{\mathbb{F}}
\newcommand{\N}{\mathbb{N}} \newcommand{\R}{\mathbb{R}}
\newcommand{\T}{\mathbb{T}} \newcommand{\Z}{\mathbb{Z}}
\newcommand{\Q}{\mathbb{Q}}
 
 
\newcommand{\cA}{\mathcal{A}} \newcommand{\cB}{\mathcal{B}}
\newcommand{\cC}{\mathcal{C}} \newcommand{\cD}{\mathcal{D}}
\newcommand{\cE}{\mathcal{E}} \newcommand{\cF}{\mathcal{F}}
\newcommand{\cG}{\mathcal{G}} \newcommand{\cH}{\mathcal{H}}
\newcommand{\cI}{\mathcal{I}} \newcommand{\cJ}{\mathcal{J}}
\newcommand{\cK}{\mathcal{K}} \newcommand{\cL}{\mathcal{L}}
\newcommand{\cM}{\mathcal{M}} \newcommand{\cN}{\mathcal{N}}
\newcommand{\cO}{\mathcal{O}} \newcommand{\cP}{\mathcal{P}}
\newcommand{\cQ}{\mathcal{Q}} \newcommand{\cR}{\mathcal{R}}
\newcommand{\cS}{\mathcal{S}} \newcommand{\cT}{\mathcal{T}}
\newcommand{\cU}{\mathcal{U}} \newcommand{\cV}{\mathcal{V}}
\newcommand{\cW}{\mathcal{W}} \newcommand{\cX}{\mathcal{X}}
\newcommand{\cY}{\mathcal{Y}} \newcommand{\cZ}{\mathcal{Z}}

\newcommand\numberthis{\addtocounter{equation}{1}\tag{\theequation}}


\newenvironment{proof}{{\bf Proof:  }}{\hfill\rule{2mm}{2mm}}
\newenvironment{proofof}[1]{{\bf Proof of #1:  }}{\hfill\rule{2mm}{2mm}}
\newenvironment{proofofnobox}[1]{{\bf#1:  }}{}\newenvironment{example}{{\bf Example:  }}{\hfill\rule{2mm}{2mm}}

%\renewcommand{\thesection}{\lecnum.\arabic{section}}
%\renewcommand{\theequation}{\thesection.\arabic{equation}}
%\renewcommand{\thefigure}{\thesection.\arabic{figure}}

\newtheorem{fact}{Fact}[section]
\newtheorem{lemma}[fact]{Lemma}
\newtheorem{theorem}[fact]{Theorem}
\newtheorem{definition}[fact]{Definition}
\newtheorem{corollary}[fact]{Corollary}
\newtheorem{proposition}[fact]{Proposition}
\newtheorem{claim}[fact]{Claim}
\newtheorem{exercise}[fact]{Exercise}
\newtheorem{note}[fact]{Note}
\newtheorem{conjecture}[fact]{Conjecture}

\newcommand{\size}[1]{\ensuremath{\left|#1\right|}}
\newcommand{\ceil}[1]{\ensuremath{\left\lceil#1\right\rceil}}
\newcommand{\floor}[1]{\ensuremath{\left\lfloor#1\right\rfloor}}

%END MACROS
%Page style
\pagestyle{fancy}

\listfiles

\raggedbottom

\lhead{2017-01-31}
\rhead{William Justin Toth CO750-Approximation Algorithms Lecture 8} %CHANGE n to ASSIGNMENT NUMBER ijk TO COURSE CODE
\renewcommand{\headrulewidth}{1pt} %heading underlined
%\renewcommand{\baselinestretch}{1.2} % 1.2 line spacing for legibility (optional)

\begin{document}
\paragraph{}
This lecture is a continuation of the previous lecture. We are proving the following theorem:
\begin{theorem}
For any xps $x$ to $LP$ there is a collection of $m$ tight sets such that their incidence vectors are linearly independent, and the collection is a laminar family.
\end{theorem}
\begin{proof}
Let $\cL$ be a laminar family of tight sets whose incidence vectors are linearly independent. We will prove through the following lemmas that if $span(\cL) \in \R^m$ then there exists a tight set we can add to $\cL$, and if this set violates laminarity then we can ``uncross" and recover a larger laminar set than $\cL$ as desired.
\end{proof}
\begin{lemma}
Let $A$ and $B$ be two crossing tight sets. The one of the following holds:
\begin{enumerate}
\item $A\backslash B$ and $B \backslash A$ are both tight and $\chi(A) + \chi(B) = \chi(A\backslash B) + \chi(B\backslash A)$.
\item $A\cup B$ and $A\cap B$ are both tight and $\chi(A) + \chi(B) = \chi(A\cup B) + \chi(B\cap A)$.
\end{enumerate}
\end{lemma}
\begin{proof}
The proof follows by case distinction on $f$. Since $f$ is weakly supermodular (wsm) either:
\begin{enumerate}
\item $f(A) + f(B) \leq f(A\backslash B) + f(B\backslash A)$ or
\item $f(A) + f(B) \leq f(A\cup B) + f(A \cap B)$.
\end{enumerate}
We will prove the first case, and the second case follows similarly.
\paragraph{}
Since $A$ and $B$ are tight
$$x(\delta(A)) + x(\delta(B)) = f(A) + f(B).$$
By feasibility
$$x(\delta(A\backslash B)) + x(\delta(B\backslash A)) \geq f(A\backslash B) + f(B\backslash A).$$
However $\delta(A) \cup \delta(B) \supseteq \delta(A\backslash B) + \delta(B \backslash A)$, so
$$x(delta(A)) + x(\delta(B)) \geq x(\delta(A\backslash B)) + x(\delta(B\backslash A).$$
Combining inequalities we see
$$f(A\backslash B) + f(B\backslash A) \leq x(\delta(A\backslash B)) + x(\delta(B\backslash A) \leq f(A\backslash B) + f(B\backslash A) $$
Hence $A\backslash B$ and $B \backslash A$ are tight.
\paragraph{}
Further by our chain of inequalities holding as equality we have
$$x(delta(A)) + x(\delta(B)) = x(\delta(A\backslash B)) + x(\delta(B\backslash A)$$
and since all $e$ have $x_e > 0$, we know that there are no edges from $A\cap B$ to $V\backslash (A\cup B)$, otherwise the above equality would be violated. Hence
$$\chi(A) + \chi(B) = \chi(A\backslash B) + \chi(B\backslash A).$$
\end{proof}
\paragraph{}
For a set $S \subseteq V$ we define the $\textit{crossing number}$ of $S$ with respect to $\cL$ to be the number of sets in $\cL$ that cross with $S$.
\begin{lemma}
Let $S$ be a tight set crossing with $T \in \cL$. Then each of the sets $S\backslash T$, $T\backslash S$. $S\cup T$, and $S\cap T$ have a crossing number smaller than that of $S$.
\end{lemma}
\begin{proof}
Let $T'$ be any set in $\cL$ crossing with any of the four sets. One checks that $T'$ crosses $S$ too. This implies that each of those four sets do not cross with $T$.
\end{proof}
\begin{lemma}
Let $S$ be a tight set such that $\chi(S)\not\in span(\cL)$, and $S$ crosses with some set in $\cL$. Then there is a tight set $S'$ that has a smaller crossing number than $S$ and $\chi(S')\not\in span(\cL)$.
\end{lemma}
\begin{proof}
We again proceed by cases on $f$ and only prove the first one, with the second following similarly.
\begin{enumerate}
\item $f(A) + f(B) \leq f(A\backslash B) + f(B\backslash A)$ or
\item $f(A) + f(B) \leq f(A\cup B) + f(A \cap B)$.
\end{enumerate}
\paragraph{}
By prior lemmas $S\backslash T$ and $T\backslash S$ have smaller crossing numbers than $S$, and further $S\backslash T$ and $T\backslash S$ are tight with
$$\chi(S) + \chi(T) = \chi(S\backslash T) + \chi(T\backslash S).$$
Hence one of  $\chi(T\backslash S)$ or $\chi(S\backslash T)$ are not in $span(\cL)$, as otherwise $\chi(S) \in span(\cL)$.
\end{proof}
\paragraph{}
Together these lemmas yield the theorem. Now with this theorem in hand we are ready to prove final unproven result from the previous lecture:
\begin{theorem}
For any wsm function $f$, any xps to $LP$ satisfies $x_e \geq \frac{1}{2}$ for at least one $e \in E$.
\end{theorem}
\begin{proof}
We have $m =|E|$ with $x_e >0$ for all $e\in E$, and $m = |\cL|$. Suppose for a contradiction that $x_e \leq \frac{1}{2}$ for all $e \in E$. We proceed by a token argument.
\paragraph{}
Assign $1$ token to every $e \in E$. For each $uv \in E$ distribute $x_{uv}$ units to each of $u$ and $v$, and distribute $1-2x_{uv}$ units to $uv$. Now reassign these values to sets in $\cL$ as follows. Let $S \in \cL$ and let $S_1,\dots, S_n \in \cL$ be $S$'s children. Ie sets strictly contained in $S$:
\begin{itemize}
\item if $v \in S\backslash \cup_i S_i$ give $S$ the value held by $v$.
\item if $uv \in E(S) \backslash \cup_i E(S_i)$ give $S$ the value held by $uv$.
\end{itemize}
Our goal is to show that after this process $S$ gets value at least one. We need some definitions:
\begin{itemize}
\item Let $C_0$ be set of edges with both endpoints in $S\backslash \cup_i S_i$.
\item Let $C_1$ be the set of edges with one endpoint in $S$ and one in $V\backslash S$.
\item Let $C_2$ be the set of edges with one endpoint in $S\backslash \cup_i S_i$ and one endpoint in some $S_i$.
\item Let $C_3$ be the set of edges with one endpoint in some $S_i$ and the other in some $S_j$ for $i\neq j$.
\end{itemize}
Observe that if $C_0 \neq \emptyset$ then $S$ trivially obtains value at least one. So suppose that $C_0 = \emptyset$. Now each $S$ gets value
$$z_S \geq \sum_{e\in C_1} x_e + \sum_{e \in C_2}(x_e + (1-2x_e)) + \sum_{e \in C_3} (1-2x_e) = x(C_1) + |C_2| -x(C_2) + |C_3| - 2x(C_3).$$
Note that since each $x_e \leq \frac{1}{2}$ this value $z_S > 0$ unless $C_1 = C_2 = C_3 = \emptyset$ which cannot happen as this would imply that $\chi(S) = \sum_i \chi(S_i)$ which contradicts linear independence.
\paragraph{}
To complete the proof we need only demonstrate that $z_S$ is integral. To see this observe by tightness of the sets in $\cL$ that
$$x(C_1) -x(C_2) -2(C_3) = x(\delta(S)) - \sum_i x(\delta(S_i)) = f(S) - \sum_i f(S_i)$$
and since $f$ is integer valued the result holds. Therefore $z_S \geq 1$ for all $S$. Now observe that no set in $\cL$ will receive value from $uv \in \delta(S)$ where $S\in \cL$ is maximal with respect to $\cL$. This implies $|E| > m$ a contradiction.
\end{proof}
\end{document}
