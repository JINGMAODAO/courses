\documentclass[letterpaper,12pt,oneside,onecolumn]{article}
\usepackage[margin=1in, bottom=1in, top=1in]{geometry} %1 inch margins
\usepackage{amsmath, amssymb, amstext}
\usepackage{fancyhdr}
\usepackage{mathtools}
\usepackage{algorithm}
\usepackage{algpseudocode}
\usepackage{theorem}
\usepackage{tikz}
\usepackage{tkz-berge}

%Macros
\newcommand{\A}{\mathbb{A}} \newcommand{\C}{\mathbb{C}}
\newcommand{\D}{\mathbb{D}} \newcommand{\F}{\mathbb{F}}
\newcommand{\N}{\mathbb{N}} \newcommand{\R}{\mathbb{R}}
\newcommand{\T}{\mathbb{T}} \newcommand{\Z}{\mathbb{Z}}
\newcommand{\Q}{\mathbb{Q}}
 
 
\newcommand{\cA}{\mathcal{A}} \newcommand{\cB}{\mathcal{B}}
\newcommand{\cC}{\mathcal{C}} \newcommand{\cD}{\mathcal{D}}
\newcommand{\cE}{\mathcal{E}} \newcommand{\cF}{\mathcal{F}}
\newcommand{\cG}{\mathcal{G}} \newcommand{\cH}{\mathcal{H}}
\newcommand{\cI}{\mathcal{I}} \newcommand{\cJ}{\mathcal{J}}
\newcommand{\cK}{\mathcal{K}} \newcommand{\cL}{\mathcal{L}}
\newcommand{\cM}{\mathcal{M}} \newcommand{\cN}{\mathcal{N}}
\newcommand{\cO}{\mathcal{O}} \newcommand{\cP}{\mathcal{P}}
\newcommand{\cQ}{\mathcal{Q}} \newcommand{\cR}{\mathcal{R}}
\newcommand{\cS}{\mathcal{S}} \newcommand{\cT}{\mathcal{T}}
\newcommand{\cU}{\mathcal{U}} \newcommand{\cV}{\mathcal{V}}
\newcommand{\cW}{\mathcal{W}} \newcommand{\cX}{\mathcal{X}}
\newcommand{\cY}{\mathcal{Y}} \newcommand{\cZ}{\mathcal{Z}}

\newcommand\numberthis{\addtocounter{equation}{1}\tag{\theequation}}


\newenvironment{proof}{{\bf Proof:  }}{\hfill\rule{2mm}{2mm}}
\newenvironment{proofof}[1]{{\bf Proof of #1:  }}{\hfill\rule{2mm}{2mm}}
\newenvironment{proofofnobox}[1]{{\bf#1:  }}{}\newenvironment{example}{{\bf Example:  }}{\hfill\rule{2mm}{2mm}}

%\renewcommand{\thesection}{\lecnum.\arabic{section}}
%\renewcommand{\theequation}{\thesection.\arabic{equation}}
%\renewcommand{\thefigure}{\thesection.\arabic{figure}}

\newtheorem{fact}{Fact}[section]
\newtheorem{lemma}[fact]{Lemma}
\newtheorem{theorem}[fact]{Theorem}
\newtheorem{definition}[fact]{Definition}
\newtheorem{corollary}[fact]{Corollary}
\newtheorem{proposition}[fact]{Proposition}
\newtheorem{claim}[fact]{Claim}
\newtheorem{exercise}[fact]{Exercise}
\newtheorem{note}[fact]{Note}
\newtheorem{conjecture}[fact]{Conjecture}

\newcommand{\size}[1]{\ensuremath{\left|#1\right|}}
\newcommand{\ceil}[1]{\ensuremath{\left\lceil#1\right\rceil}}
\newcommand{\floor}[1]{\ensuremath{\left\lfloor#1\right\rfloor}}

%END MACROS
%Page style
\pagestyle{fancy}

\listfiles

\raggedbottom

\lhead{2017-02-14}
\rhead{William Justin Toth CO750-Approximation Algorithms Lecture 12} %CHANGE n to ASSIGNMENT NUMBER ijk TO COURSE CODE
\renewcommand{\headrulewidth}{1pt} %heading underlined
%\renewcommand{\baselinestretch}{1.2} % 1.2 line spacing for legibility (optional)

\begin{document}
\section{TSP}
\paragraph{}
In the Traveling Salesman Problem (TSP) we are given an undirected graph $G = (V,E)$ and a cost function $c: E \rightarrow \R_+$. We are to find a tour (Hamiltonian Cycle) visiting all vertices of $V$, of minimum cost.
\paragraph{Proposition}
TSP in general is ``in-approximable" (unless $P=NP$).
\paragraph{Proof}
Reduction from Hamiltonian Cycle ($NP$-hard). Suppose there exists an $\alpha(n)$-approximation algorithm for TSP. Given a graph $G=(V,E)$ we set $c_e = 1$ for all $e\in E$ and set $c_e = n\alpha(n) + 1$ for all $e \in \bar{E}$ (the complement of $E$). Run our TSP algorithm on $(V,E\cup \bar{E})$. The solution has cost at most $n\alpha(n)$ if and only if $G$ has a Hamiltonian Cycle.$\blacksquare$
\section{Metric TSP}
Now we suppose we are given a metric graph instead. Going forward we will implicitly assume this without writing it.
\begin{theorem} (Christofides '76) There is a $\frac{3}{2}$-approximation algorithm for TSP.
\end{theorem}

\begin{proof}
Consider the following algorithm for TSP:
\begin{enumerate}
\item Find a minimum cost spanning tree $F$ of $G$.
\item Let $T\subseteq V$ be the set of odd degree vertices in in $F$. It is easy to see by handshaking that $|T|$ is even.
\item Add minimum-cost $T$-join $H$.
\item Shortcut $F\cup H$ (without increasing cost, by triangle inequality) into a tour.
\end{enumerate}
Recall that a $T$-join is a set of edges such that only the vertices in $T$ have odd degree, and $T$-joins can be found in polynomial time. We have $c(F) \leq OPT$ since OPT spans $G$. We also have $H \leq OPT/2$. If we start at some vertex in $T$ and follow the cycle given by $OPT$ to the next vertex in $T$ we obtain a path. Traveling further to the next vertex in $T$, obtaining another a path, and proceeding this way yield two $T$-joins depending on which way around the cycle you decided to go when starting this process. Hence $OPT$ contains two feasible $T$-joins and the claimed bound on $H$ holds. Therefore $c(F\cup H) \leq c(F) + c(H) \leq \frac{3}{2} OPT$.
\end{proof}
\paragraph{Big Open Question}
Can one improve over $\frac{3}{2}$? There is an $LP$-relaxation for TSP that seems to be promising: The Held-Karp relaxation $(HK)$:
\begin{align*}
\min &\sum_e c_e x_e \\
\text{s.t.} x(\delta(v)) &= 2 &\forall v \in V \\
x(\delta(S)) &\geq 2 &\forall S\subseteq V \\
x&\geq 0.
\end{align*}
\begin{lemma}
The integrality gap of $(HK)$ is $\leq \frac{3}{2}$ and $\geq \frac{4}{3}$.
\end{lemma}
\begin{proof}
For the lower bound consider graph formed by taking two $K_3$ and connecting each vertex in one $K_3$ to exactly one vertex in the other $K_3$ via a path of length $n$ with all edge weights one. Assigning $x_e = \frac{1}{2}$ for the $K_3$ edges and $x_e = 1$ otherwise is clearly feasible for $(HK)$. Any integral tour must use each $n$ length path at least once, and in particular must cross one of them twice. Thus an integral tour has cost $\frac{4}{3}n-2$.
\paragraph{}
To prove the upper bound, we'll use the solution output by Christofides algorithm: $F \cup H$. Recall (Edmonds '70) The convex hull of incidence vectors of spanning trees of $G = (V,E)$ is given by $(MST):$
\begin{align*}
x(E) &n-1 \\
\sum_{e \in E(U)} x_e &\leq |U| - 1 &\forall \emptyset\neq U \subseteq V\\
x&\geq 0.
\end{align*}
Also observe that if $x$ is feasible for $(HK)$ then $\bar{x} = \frac{n-1}{n}x$ is feasible for $(MST)$. This is an immediate and easy calculation.
\paragraph{}
Thus if we let $x^*$ be optimal for $(HK)$ then $$c(F) \leq c(x^*).$$
\paragraph{}
Now we use (Edmonds and Johnson '73) which says that given a graph $G=(V,E)$ with costs $c:E\rightarrow \R_+$ and $T\subseteq V$ that the cost of a minimum cost $T$-join on $G$ is equal to the optimal solution to $(T-JOIN)$:
\begin{align*}
\min &\sum_e c_ex_e \\
\text{s.t.} x(\delta(U)) &\geq 1 &\forall U \subseteq V : |U\cap T| \text{ is odd}\\
x&\geq 0.
\end{align*}
Observe that if $x$ is feasible for $(HK)$ then $\frac{x}{2}$ is feasible for $(T-JOIN)$ (trivially). Thus we see that
$$c(H) \leq \frac{c(x^*)}{2}.$$
Therefore $c(F\cup H) \leq c(F) + c(H) \leq \frac{3}{2}c(x^*).$
\end{proof}
\paragraph{Another Big Open Question}
Conjecture: The integrality gap of $(HK)$ is $\leq \frac{4}{3}$.
\paragraph{A related proposition on $T$-joins}
(Edmonds and Johnson '73) The convex hull of incidence vectors of $T$-joins is given by:
$$|A| - x(A) + x(\delta(U\backslash A)) \geq 1 \quad \forall U\subseteq V, A\subseteq \delta(U) : |U\cap T| + |A| \text{ is odd}.$$
\section{Special Case: Graphic TSP}
\paragraph{}
Given an undirected graph $G=(V,E)$, a graphic TSP instance is the TSP instance defined on the metric closure of $G$ (edge weights of $G$ are all $1$). We can reformulate this problem as: 
\paragraph{}
Given an unweight $G=(V,E)$, find an Eulerian spanning (multi) subgraph of $G$ with minimum number of edges.
\paragraph{Improvements over Christofides for graphic TSP}
\begin{itemize}
\item(Oveis, Gharan,  Saber, Singh '11) A tiny improvement. Strategy was to solve $(HK)$ obtaining $x^*$, then express $\frac{n-1}{n}x^*$ as a convex combination of spanning trees, then sample a tree according to the maximum entropy distribution, then add a minimum cost $T$-join as usual.
\item They conjecture that this above algorithm should give a $<\frac{3}{2}$-approximation for the general case of metric TSP.
\item (Monke and Svensson '11) Removable Pairs
\begin{itemize}
\item (Mucha '12) A better analysis of same algorithm leading for $\frac{13}{9}$-approximation 
\end{itemize}
\item (Sebo, Vygen '12) Ear Decomposition plus Removable Pairs leads to a $\frac{7}{5}$-approximation.
\end{itemize}
\end{document}
