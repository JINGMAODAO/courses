\documentclass[letterpaper,12pt,oneside,onecolumn]{article}
\usepackage[margin=1in, bottom=1in, top=1in]{geometry} %1 inch margins
\usepackage{amsmath, amssymb, amstext}
\usepackage{fancyhdr}
\usepackage{mathtools}
\usepackage{algorithm}
\usepackage{algpseudocode}
\usepackage{theorem}
\usepackage{tikz}
\usepackage{tkz-berge}

%Macros
\newcommand{\A}{\mathbb{A}} \newcommand{\C}{\mathbb{C}}
\newcommand{\D}{\mathbb{D}} \newcommand{\F}{\mathbb{F}}
\newcommand{\N}{\mathbb{N}} \newcommand{\R}{\mathbb{R}}
\newcommand{\T}{\mathbb{T}} \newcommand{\Z}{\mathbb{Z}}
\newcommand{\Q}{\mathbb{Q}}
 
 
\newcommand{\cA}{\mathcal{A}} \newcommand{\cB}{\mathcal{B}}
\newcommand{\cC}{\mathcal{C}} \newcommand{\cD}{\mathcal{D}}
\newcommand{\cE}{\mathcal{E}} \newcommand{\cF}{\mathcal{F}}
\newcommand{\cG}{\mathcal{G}} \newcommand{\cH}{\mathcal{H}}
\newcommand{\cI}{\mathcal{I}} \newcommand{\cJ}{\mathcal{J}}
\newcommand{\cK}{\mathcal{K}} \newcommand{\cL}{\mathcal{L}}
\newcommand{\cM}{\mathcal{M}} \newcommand{\cN}{\mathcal{N}}
\newcommand{\cO}{\mathcal{O}} \newcommand{\cP}{\mathcal{P}}
\newcommand{\cQ}{\mathcal{Q}} \newcommand{\cR}{\mathcal{R}}
\newcommand{\cS}{\mathcal{S}} \newcommand{\cT}{\mathcal{T}}
\newcommand{\cU}{\mathcal{U}} \newcommand{\cV}{\mathcal{V}}
\newcommand{\cW}{\mathcal{W}} \newcommand{\cX}{\mathcal{X}}
\newcommand{\cY}{\mathcal{Y}} \newcommand{\cZ}{\mathcal{Z}}

\newcommand\numberthis{\addtocounter{equation}{1}\tag{\theequation}}


\newenvironment{proof}{{\bf Proof:  }}{\hfill\rule{2mm}{2mm}}
\newenvironment{proofof}[1]{{\bf Proof of #1:  }}{\hfill\rule{2mm}{2mm}}
\newenvironment{proofofnobox}[1]{{\bf#1:  }}{}\newenvironment{example}{{\bf Example:  }}{\hfill\rule{2mm}{2mm}}

%\renewcommand{\thesection}{\lecnum.\arabic{section}}
%\renewcommand{\theequation}{\thesection.\arabic{equation}}
%\renewcommand{\thefigure}{\thesection.\arabic{figure}}

\newtheorem{fact}{Fact}[section]
\newtheorem{lemma}[fact]{Lemma}
\newtheorem{theorem}[fact]{Theorem}
\newtheorem{definition}[fact]{Definition}
\newtheorem{corollary}[fact]{Corollary}
\newtheorem{proposition}[fact]{Proposition}
\newtheorem{claim}[fact]{Claim}
\newtheorem{exercise}[fact]{Exercise}
\newtheorem{note}[fact]{Note}
\newtheorem{conjecture}[fact]{Conjecture}

\newcommand{\size}[1]{\ensuremath{\left|#1\right|}}
\newcommand{\ceil}[1]{\ensuremath{\left\lceil#1\right\rceil}}
\newcommand{\floor}[1]{\ensuremath{\left\lfloor#1\right\rfloor}}

%END MACROS
%Page style
\pagestyle{fancy}

\listfiles

\raggedbottom

\lhead{2017-02-28}
\rhead{William Justin Toth CO750-Approximation Algorithms Lecture 14} %CHANGE n to ASSIGNMENT NUMBER ijk TO COURSE CODE
\renewcommand{\headrulewidth}{1pt} %heading underlined
%\renewcommand{\baselinestretch}{1.2} % 1.2 line spacing for legibility (optional)

\begin{document}
\section{Unsplittable Flow}
\paragraph{}
In the Unsplittable Flow problem (UF) we are given a directed graph $G = (V,E)$, edge capacities $c : E \rightarrow \R_+$, a source $s \in V$, and $k$ terminals $t_i \in V$, each with demand $d_i \in \R_+$. We are to find a path $P_i$ for all $i \in [k]$ along which we can route $d_i$ from $s$ to $t_i$, while respecting capacities $c$.
\paragraph{Observation}
A necessary condition for the existence of a solution is that there exists a fractional (splittable) flow. A fractional flow exists if and only if the cut condition holds:
$$\sum_{e\in \delta(S)} c(e) \geq \sum_{i:t_i \in S} d_i \quad \forall S \subseteq V\backslash \{s\}.$$
\paragraph{Proposition}
Deciding whether an unsplittable flow exists is $NP$-complete.
\paragraph{Proof Idea}
Reduction from PARTITION: Given a (multi)-set of integers $a_1,\dots, a_n \in \N$ find a partition into $S_1$ and $S_2$ such that $\sum_{i \in S_1} a_i = \sum_{i\in S_2} a_i$. To perform the reduction we construct a graph with source $s$, a vertex $t$, and terminals $t_1, \dots, t_n$. The terminals correspond to each $a_i$ and have demand $d_i = a_i$. There is an arc from $t$ to $t_i$ for each $i$ with capacity $a_i$. There are two arcs from $s$ to $t$ each with capacity $\sum_{i} a_i /2$. Then it is not hard to see that the decision of which of the two arcs to take in path $P_i$ (top or bottom) indicates which partition to put $a_i$ into should a solution exist.
\begin{theorem}
(Dinitz, Garg, Goemans, '99) Given a fractional flow $f$ satisfying demands $d_i$ without violating $c$, we can construct an unsplittable flow satisfying demands such that the total flow on each edge is at most $c(e) + d_{max}$ where $d_{max} = \max_i \{d_i\}$.
\end{theorem}
\paragraph{}
Note that we may assume without loss of generality that $G$ is acyclic (we remove arcs with $f=0$) and $c(e) = f(e)$ for all $e$ (this is worst case, we can replace flow with one of this form as desired). The algorithm we present will:
\begin{itemize}
\item modify the flow
\item move terminals (until they reach $s$)
\item eliminate edges $e$ with $f(e) = 0$.
\end{itemize}
The algorithm will maintain the important invariant that the flow at each iteration satisfies the current demands.
\paragraph{Definitions}
A terminal $t_i$ at a vertex $v$ is called $\textit{regular}$ if $d_i > f(e)$ for all $e$ entering $v$, and is called $\textit{irregular}$ otherwise. An edge $(u,v)$ is called $\textit{singular}$ if $v$ and all vertices reachable from $v$ have out-degree at most $1$.
\paragraph{Algorithm}
Preliminary Phase:
\begin{enumerate}
\item Let $t_i$ be an irregular terminal at $v$, and $(u,v)$ be s.t. $f(u,v) \geq d_i$. Move $t_i$ to $u$. Set $f(u,v) = f(u,v) - d_i$.
\item If $f(u,v) = 0$, eliminate $(u,v)$. If $t_i = s$, eliminate $t_i$.
\item End phase when no more irregular terminals.
\end{enumerate}
Iteration(s):
\begin{enumerate}
\item Find an ``alternating cycle"
\item ``Augment" the flow along with cycle.
\item ``Move Terminals".
\item STOP when all terminals have reached $s$.
\end{enumerate}
Each step will be explained in turn.
\paragraph{1. Alternating Cycle}
Take an arbitrary vertex $v$. Follow the outgoing edges forming a forward path until we reach a node with no outgoing edges. Since $G$ is acyclic, this node is a terminal $t_i$. Use singular edges to construct a backward path from $t_i$ until we find a vertex $u$ with an outgoing edge. Start a forward path from $u$. Repeat this process until we find a vertex we have previously visited.
\paragraph{}
We can view this cycle $C$ as a union of alternating forward and backward paths. Hence the name.
\paragraph{2. Augment}
We augment the flow on $C$ by increasing flow on backward paths and decreasing the flow on forward paths by $\epsilon$.
$$\epsilon := \min\{\epsilon_1, \epsilon_2\}$$
$$\epsilon_1 := \min\{f(e): e \text{ is on a forward path in } C\}$$
$$\epsilon_2 := \min\{d_i - f(e): e=(v',v'') \text{ for } e \text{ on a backward path to } t_i \text{ with } d_i > f(e) \}.$$
\paragraph{3. Move Terminals}
Move $t_i$ along $e=(v,t_i)$ (as in Preliminary phase) whenever either
\begin{itemize}
\item $e$ is singular and $f(e) = d_i$, or,
\item $e$ is not singular and $f(e) \geq d_i$.
\end{itemize}
\subsection{Analysis}
\paragraph{Observation}
Whenever an edge becomes singular it stays singular because we do not add edges.
\paragraph{Lemma 1}
If, at the beginning of an iteration, $v$ contains an irregular terminal $t_j$, then:
\begin{enumerate}
\item The out-degree of $v$ is $0$.
\item $v$ contains no other irregular terminals.
\item $v$ contains another (regular) terminal.
\end{enumerate}
\paragraph{Proof}
Since $t_j$ is irregular, there exists $e=(u,v)$ such that $d_j < f(e)$ (if $d_j = f(e)$ we would have moved $t_j$ in previous iteration). We claim that $t_j$ became irregular exactly when moved to $v$. To see this observe that we increase flow only on backwards edges, but we would not exceed $d_j$ by our choice of $\epsilon_2$, so if $t_j$ was not irregular when first moved to $v$ it will not become irregular in later iterations.
\paragraph{}
Consider when $t_j$ was moved from (say) $w$ to $v$. Then $(u,v)$ is singular, which implies $(v,w)$ disappears proving $1.$. Then note that by $1.$ $v$ has to contain another terminal. But this terminal cannot be irregular, since when we move an irregular terminal, the out-degree becomes $0$ and we cannot move another such terminal to $v$. This proves $2.$ and $3.$ $\blacksquare$
\paragraph{Corollary}
At the beginning of any iteration, the in-degree of any vertex containing one or more terminals is at least $2$.
\paragraph{Lemma 2}
Unless all terminals reach sources, the algorithm will find an alternating cycle.
\paragraph{Proof}
When constructing a forward path, we stop at a vertex of out-degree $0$, hence it has a terminal. So by corollary we can start constructing a backward path. When constructing a backward path we will either find a previously visited node (node that flows must join source), or one with out-degree $2$. $\blacksquare$
\paragraph{Proof of Theorem}
Before an edge becomes singular it never increases flow value. After, we move at most $1$ terminal along the edge. Thus the increase in demand is at most that of the demand of one terminal which is at most $d_{max}$.$\blacksquare$
\end{document}
