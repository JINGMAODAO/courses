\documentclass[letterpaper,12pt,oneside,onecolumn]{article}
\usepackage[margin=1in, bottom=1in, top=1in]{geometry} %1 inch margins
\usepackage{amsmath, amssymb, amstext}
\usepackage{fancyhdr}
\usepackage{mathtools}
\usepackage{algorithm}
\usepackage{algpseudocode}
\usepackage{theorem}
\usepackage{tikz}
\usepackage{tkz-berge}

%Macros
\newcommand{\A}{\mathbb{A}} \newcommand{\C}{\mathbb{C}}
\newcommand{\D}{\mathbb{D}} \newcommand{\F}{\mathbb{F}}
\newcommand{\N}{\mathbb{N}} \newcommand{\R}{\mathbb{R}}
\newcommand{\T}{\mathbb{T}} \newcommand{\Z}{\mathbb{Z}}
\newcommand{\Q}{\mathbb{Q}}
 
 
\newcommand{\cA}{\mathcal{A}} \newcommand{\cB}{\mathcal{B}}
\newcommand{\cC}{\mathcal{C}} \newcommand{\cD}{\mathcal{D}}
\newcommand{\cE}{\mathcal{E}} \newcommand{\cF}{\mathcal{F}}
\newcommand{\cG}{\mathcal{G}} \newcommand{\cH}{\mathcal{H}}
\newcommand{\cI}{\mathcal{I}} \newcommand{\cJ}{\mathcal{J}}
\newcommand{\cK}{\mathcal{K}} \newcommand{\cL}{\mathcal{L}}
\newcommand{\cM}{\mathcal{M}} \newcommand{\cN}{\mathcal{N}}
\newcommand{\cO}{\mathcal{O}} \newcommand{\cP}{\mathcal{P}}
\newcommand{\cQ}{\mathcal{Q}} \newcommand{\cR}{\mathcal{R}}
\newcommand{\cS}{\mathcal{S}} \newcommand{\cT}{\mathcal{T}}
\newcommand{\cU}{\mathcal{U}} \newcommand{\cV}{\mathcal{V}}
\newcommand{\cW}{\mathcal{W}} \newcommand{\cX}{\mathcal{X}}
\newcommand{\cY}{\mathcal{Y}} \newcommand{\cZ}{\mathcal{Z}}

\newcommand\numberthis{\addtocounter{equation}{1}\tag{\theequation}}


\newenvironment{proof}{{\bf Proof:  }}{\hfill\rule{2mm}{2mm}}
\newenvironment{proofof}[1]{{\bf Proof of #1:  }}{\hfill\rule{2mm}{2mm}}
\newenvironment{proofofnobox}[1]{{\bf#1:  }}{}\newenvironment{example}{{\bf Example:  }}{\hfill\rule{2mm}{2mm}}

%\renewcommand{\thesection}{\lecnum.\arabic{section}}
%\renewcommand{\theequation}{\thesection.\arabic{equation}}
%\renewcommand{\thefigure}{\thesection.\arabic{figure}}

\newtheorem{fact}{Fact}[section]
\newtheorem{lemma}[fact]{Lemma}
\newtheorem{theorem}[fact]{Theorem}
\newtheorem{definition}[fact]{Definition}
\newtheorem{corollary}[fact]{Corollary}
\newtheorem{proposition}[fact]{Proposition}
\newtheorem{claim}[fact]{Claim}
\newtheorem{exercise}[fact]{Exercise}
\newtheorem{note}[fact]{Note}
\newtheorem{conjecture}[fact]{Conjecture}

\newcommand{\size}[1]{\ensuremath{\left|#1\right|}}
\newcommand{\ceil}[1]{\ensuremath{\left\lceil#1\right\rceil}}
\newcommand{\floor}[1]{\ensuremath{\left\lfloor#1\right\rfloor}}

%END MACROS
%Page style
\pagestyle{fancy}

\listfiles

\raggedbottom

\lhead{2017-02-09}
\rhead{William Justin Toth CO750-Approximation Algorithms Lecture 11} %CHANGE n to ASSIGNMENT NUMBER ijk TO COURSE CODE
\renewcommand{\headrulewidth}{1pt} %heading underlined
%\renewcommand{\baselinestretch}{1.2} % 1.2 line spacing for legibility (optional)

\begin{document}
(This lecture is a continuation of Lecture 10)
\paragraph{Recall}
Given an optimal Steiner Tree $S^*$ we construct witness tree $W$ by the following process:
\begin{enumerate}
\item Turn $S^*$ into a binary tree by adding $0$ cost edges.
\item Choose one edge incident to each Steiner Node with probability $\frac{1}{2}$ and call the set of edges chosen $\tilde{B}$.
\item Set $W = \{ uv \in {R \choose 2} : |P_{uv} \cap \tilde{B}| = 1\}$.
\item For each $e \in S^*$ set $w(e) = \{ uv \in W : e \in P_{uv}\}$
\end{enumerate}
The Deletion Process is a follows: When a component $C^*$ is sampled, we delete some edges from $W$ such that $Pr[e \in W \text{ is deleted in each iteration}] \geq \frac{1}{M}$. When $w(e)$ are deleted, delete $e \in S^*$.
\paragraph{}
We hope to answer the question: How many iterations until $e \in S^*$ is deleted?
\paragraph{Lemma $3$} For any $e \in S^*$ at level $k_e \leq |R| - 1$ from the root of the binary tree, one has
$$Pr[|w(e)| = q] = \begin{cases}
\frac{1}{2^q}, &1\leq q < k_e \\
\frac{2}{2^q}, &q=k_e \\
0, &\text{otherwise}.
\end{cases}$$
\paragraph{Proof of Lemma $3$}
Let $e=v_0v_1$. If $(v_{q-1}, v_q)$ is the first edge of $\tilde{B}$ on the path $v_0$ to $r$ then $|w(e)| = q$. This happens with probability $\frac{1}{2^q}$. If $(v_{k_e - 1}, v_{k_e})$ is the first edge of $\tilde{B}$ on this path, OR if no edges of $\tilde{B}$ are on this path then $|w(e)| = k_e$. Each happens with probability $\frac{1}{2^{k_e}}$, so in total we have probability $\frac{1}{2^{k_e}} + \frac{1}{2^{k_e}} = \frac{2}{2^{k_e}}$.$\blacksquare$
\paragraph{Lemma $4$}
Let $\tilde{W} \subseteq W$. The the expected number of iterations until all edges of $\tilde{W}$ are removed is at most $H_{|\tilde{W}|} \cdot M$.
\paragraph{Proof Idea Sketch for Lemma $4$}
This situation is analogous to the coupon collector's problem: You have $n$ coupons, and you sample one in each iteration (with repetition). How many iterations $T$ do we wait (in expectation) until we collect all our coupons? You can define a random variable $t_i := $ time needed to collect $i$-th coupon after you have collected $i-1$ coupons. Then
$$Pr[\text{collect new coupon } i] = \frac{n -(i-1)}{n}.$$
One proves that $E[T] = \sum_{i=1}^n E[t_i] = \frac{n}{n} + \frac{n}{n-1} + \dots + \frac{n}{1} = n H_n$.
\paragraph{Proof of Lemma $4$}
We proceed by induction $q$ where $m_q$ is the number of iterations until $\tilde{W}$ is deleted ($q = |\tilde{W}|$). In the Base Case: $q=1$. Each edge is deleted with probability $\geq \frac{1}{M}$ which implies that $m_1 \leq M$.
\paragraph{}
Inductive Case: If we condition on the event that $i \in \{0,\dots,q\}$ edges are deleted in the first iteration then we need in expectation at most $m_{q-i}$ more iterations to delete the rest. Let $\lambda_i$ be the probability that at least $i$ edges are deleted in the first iteration. We obtain:
\begin{align*}
m_q &\leq 1 + \sum_{i=0}^q Pr[i \text{ edge are deleted in 1st iteration}] m_{q-i} \\
&\leq 1 + \sum_{i=0}^q (\lambda_i - \lambda_{i+1}) \cdot m_q &(\lambda_0 = 1, \lambda_{q+1} = 0, m_0 = 0) \\
&\leq 1 + M \sum_{i=1}^q \lambda_i(H_{q-i} - H_{q-i + 1}) +\lambda_1H_qM +(1-\lambda_1)m_q \\
&\leq 1 - \frac{1}{q}M\sum_{i=1}^q\lambda_i + \lambda_1H_qM(1-\lambda_1)m_q \\
&\leq \lambda_1 H_q M + 1-\lambda_1 m_q.
\end{align*}
Thus we have $m_q \leq H_q M.$ $\blacksquare$
\paragraph{Theorem}
For every fixed $k$ there exists a $\ln(4)$-approximation algorithm for $k$-restricted Steiner Tree.
\paragraph{Proof}
For each $e \in S^*$, let $D(e) = \max\{t | e \in S^t\}$. Then we have
\begin{align*}
E[D(e)] &= \sum_{q=1}^{k_e} Pr[|w(e)|=q] \cdot E[D(e) | |w(e)|=q] \\
&\leq \sum_{q=1}^{k_e} Pr[|w(e)|=q] \cdot H_q \cdot M \\
&\leq \sum_{q=1}^{k_e-1}\frac{1}{2^q}H_qM  \frac{2}{2^{k_e}}H_{k_e}M \\
&\leq \sum_{q\geq 1} \frac{1}{2^q}H_q M \\
&= \ln(4) M.
\end{align*}
Finally we see:
\begin{align*}
E[\sum_{t\geq 1} c(C^t)] &\leq \frac{1}{M} \sum_{t\geq 1} E[opt_k^t] \\
&\leq \frac{1}{M} \sum_{t \geq 1} E[c(S^t)] \\
&\leq \frac{1}{M} \sum_{e\in S^*} E[D(e)]c_e \\
&=\ln(4) \cdot c(S^*) \\
&= \ln(4) opt_k.\blacksquare
\end{align*}
\paragraph{Corollary}
There exists a $\ln(4)+epsilon \leq 1.39$-approximation algorithm for Steiner Tree.
\paragraph{Examples}
Quasi-bipartite instances: instances where Steiner nodes are not adjacent to each other. Consider $S^*$ in this case. Each component of $S^*$ is a star graph centred at a Steiner Node. For each component of $S^*$ we take uniformly at random all but one edge in $\tilde{B}$. Let $\bar{k}$ be the number of terminals in one such component. Let $e$ be an edge of this component. Then
$$|w(e)| = \begin{cases}
1, &\text{with probability }1-\frac{1}{\bar{k}} \\
\bar{k}-1, &\text{with probability } \frac{1}{\bar{k}}.
\end{cases}$$
So $E[D(e)] \leq \frac{1}{\bar{k}}H_{\bar{k}-1}M + (1-\frac{1}{\bar{k}} H_1 M \leq \frac{73}{60}M$.
\paragraph{Final Remarks}
The algorithm can be derandomized. There is a theorem of Goemans, Olver, Rothvoss, and Zenklusen from $2012$ which states the integrality gap is at most $1.39$ for the Directed Cut Relaxation Linear Program. Open question: Can one prove $<2$ bound on the integrality gap of the Bidirected Cut Relaxation?
\paragraph{Theorem}
(Feldman, Koenemann, Olver, S' '14) BCR and DCR are equivalent for all instances such that Steiner Nodes induce a graph of maximum degree $2$.
\end{document}
