\documentclass[letterpaper,12pt,oneside,onecolumn]{article}
\usepackage[margin=1in, bottom=1in, top=1in]{geometry} %1 inch margins
\usepackage{amsmath, amssymb, amstext}
\usepackage{fancyhdr}
\usepackage{mathtools}
\usepackage{algorithm}
\usepackage{algpseudocode}
\usepackage{theorem}
\usepackage{tikz}
\usepackage{tkz-berge}

%Macros
\newcommand{\A}{\mathbb{A}} \newcommand{\C}{\mathbb{C}}
\newcommand{\D}{\mathbb{D}} \newcommand{\F}{\mathbb{F}}
\newcommand{\N}{\mathbb{N}} \newcommand{\R}{\mathbb{R}}
\newcommand{\T}{\mathbb{T}} \newcommand{\Z}{\mathbb{Z}}
\newcommand{\Q}{\mathbb{Q}}
 
 
\newcommand{\cA}{\mathcal{A}} \newcommand{\cB}{\mathcal{B}}
\newcommand{\cC}{\mathcal{C}} \newcommand{\cD}{\mathcal{D}}
\newcommand{\cE}{\mathcal{E}} \newcommand{\cF}{\mathcal{F}}
\newcommand{\cG}{\mathcal{G}} \newcommand{\cH}{\mathcal{H}}
\newcommand{\cI}{\mathcal{I}} \newcommand{\cJ}{\mathcal{J}}
\newcommand{\cK}{\mathcal{K}} \newcommand{\cL}{\mathcal{L}}
\newcommand{\cM}{\mathcal{M}} \newcommand{\cN}{\mathcal{N}}
\newcommand{\cO}{\mathcal{O}} \newcommand{\cP}{\mathcal{P}}
\newcommand{\cQ}{\mathcal{Q}} \newcommand{\cR}{\mathcal{R}}
\newcommand{\cS}{\mathcal{S}} \newcommand{\cT}{\mathcal{T}}
\newcommand{\cU}{\mathcal{U}} \newcommand{\cV}{\mathcal{V}}
\newcommand{\cW}{\mathcal{W}} \newcommand{\cX}{\mathcal{X}}
\newcommand{\cY}{\mathcal{Y}} \newcommand{\cZ}{\mathcal{Z}}

\newcommand\numberthis{\addtocounter{equation}{1}\tag{\theequation}}


\newenvironment{proof}{{\bf Proof:  }}{\hfill\rule{2mm}{2mm}}
\newenvironment{proofof}[1]{{\bf Proof of #1:  }}{\hfill\rule{2mm}{2mm}}
\newenvironment{proofofnobox}[1]{{\bf#1:  }}{}\newenvironment{example}{{\bf Example:  }}{\hfill\rule{2mm}{2mm}}

%\renewcommand{\thesection}{\lecnum.\arabic{section}}
%\renewcommand{\theequation}{\thesection.\arabic{equation}}
%\renewcommand{\thefigure}{\thesection.\arabic{figure}}

\newtheorem{fact}{Fact}[section]
\newtheorem{lemma}[fact]{Lemma}
\newtheorem{theorem}[fact]{Theorem}
\newtheorem{definition}[fact]{Definition}
\newtheorem{corollary}[fact]{Corollary}
\newtheorem{proposition}[fact]{Proposition}
\newtheorem{claim}[fact]{Claim}
\newtheorem{exercise}[fact]{Exercise}
\newtheorem{note}[fact]{Note}
\newtheorem{conjecture}[fact]{Conjecture}

\newcommand{\size}[1]{\ensuremath{\left|#1\right|}}
\newcommand{\ceil}[1]{\ensuremath{\left\lceil#1\right\rceil}}
\newcommand{\floor}[1]{\ensuremath{\left\lfloor#1\right\rfloor}}

%END MACROS
%Page style
\pagestyle{fancy}

\listfiles

\raggedbottom

\lhead{2017-01-10}
\rhead{William Justin Toth CO750-Approximation Algorithms Lecture 3} %CHANGE n to ASSIGNMENT NUMBER ijk TO COURSE CODE
\renewcommand{\headrulewidth}{1pt} %heading underlined
%\renewcommand{\baselinestretch}{1.2} % 1.2 line spacing for legibility (optional)

\begin{document}
\section{Steiner Trees}
\paragraph{}
In the Steiner Tree (ST) problem we are given a graph $G=(V,E)$ and edge lengths (costs) $c_e \geq 0$, and a set $R \subseteq V$ called the set of terminals. We are to find a minimum cost tree $T$ connecting all the terminals.
\paragraph{}
Note that if $R = V$ we have a spanning tree instance (thus we can solve in polynomial time). Also if $|R| = 1$ we have a shortest path instance (and again we can solve in polynomial time). But for general $|R| = k$ the problem in $NP$-hard.
\paragraph{}
A function $d : V \times V \rightarrow \R_+$ is a {\it metric} if 
\begin{enumerate}
\item $d(u,u) = 0$
\item $d(u,v) = d(v,u)$ for all $u,v \in V$
\item $d(u,v) \leq d(u,z) + d(z,v)$ for all $u,v,z \in V$.
\end{enumerate}
An edge weighted graph is called a {\it metric graph} if its weight function induces a metric on $V \times V$. A metric graph is necessarily complete in order to make sense.
\paragraph{}
Let $G=(V,E)$ be a graph. Let $c : E \rightarrow \R_+$ be a weight function. The {\it metric arising from} $G$ is the function $d$ defined on $V \times V$ where
$$d(u,v) := \text{ the length of a shortest path from } u \text{ to } v \text{ on } G \text{ according to } c.$$ Given $(G,c)$ we can construct the {\it metric completion} of $G$. That is, a complete graph $G^M := (V, E^M)$ with edge cost $d(u,v)  = $ length of shortest $uv$-path in $G$.
\begin{lemma}
There exists a Steiner Tree $T$ connecting $R\subseteq V$ of optimal cost $z$ if and only if there exists a Steiner Tree $\bar{T}$ connecting $R$ on $G^M$ of optimal cost $z$. Furthermore there is a polynomial time map between $T$ and $\bar{T}$.
\end{lemma}
\begin{proof}
First we observe that the cost of the path between any two vertices $u,v$ in $T$ is at most the cost of the shortest $uv$-path in $G$, as otherwise we could replace the $uv$-path in $T$ with the shortest $uv$-path in $G$, and break cycles arbitrarily to obtain a new tree $T'$ of strictly lower cost than $T$. Hence the distance between any two vertices in $T$ is a shortest path distance. Thus by definition of $G^M$, both $T$ and $\bar{T}$ have same optimal cost.
\paragraph{}
To map from $\bar{T}$ to $T$ replace each edge that is not an edge of $G$ with a shortest path in $G$ (reverse this operation to go from $T$ to $\bar{T}$). If cycles form, break them arbitrarily.
\end{proof}
\paragraph{}
Due to the previous lemma we may assume our input (ST) instances is a metric graph. Consider the following algorithm for (ST) given a metric graph $G$, weights $d$, and terminals $R \subseteq V(G)$.
\begin{enumerate}
\item Find a Minimum Spanning Tree $\bar{T}$ on $G[R]$
\item Return $\bar{T}$.
\end{enumerate}
\begin{theorem}
The above algorithm is a $2$-approximation for (ST).
\end{theorem}
\begin{proof}
Let $T^*$ be an optimal solution to a problem instance of (ST). Create an Eulerian multigraph from $T^*$ by replacing each edge with two parallel edges. The cost of this multigraph is is at most $2d(T^*)$. Since this multigraph is Eulerian there exists an Eulerian tour $P$. Let $r \in R$. Traverse $P$ starting at $r$ and shortcut it to obtain a Hamiltonian cycle $C$ on $G[R]$. To be more precise order the vertices of $R$ as $r_1, \dots, r_{|R|}$ in the order they are traversed starting from $r$ along $P$. Form cycle $C$ by taking $C = \{r_1r_2, \dots, r_{|R|-1}r_{|R|}\}$. By the triangle inequality $d(C) \leq d(P) \leq 2d(T^*)$. Further by removing any edge of $C$ we can obtain a spanning tree on $G[R]$, hence $d(\bar{T}) \leq d(C)$.
Thus we have
$$d(\bar{T}) \leq d(C) \leq 2d(T^*).$$
Since computing Minimum Spanning Trees can be done in polynomial time, this completes our result.
\end{proof}
\paragraph{}
To see the analysis is tight consider the graph $K_{n+1}$. Label a vertex $v$ and set $d(v,w) = 1$ for all $w \in V\backslash\{v\}$. Set $d(w,u) = 2$ for all $u,w \in V\backslash\{v\}$. It is not hard to verify this is a metric graph. Set $R = V\backslash\{v\}$. Then a minimum spanning tree on $G[R]$ has cost $2(n-1)$ while an optimal solution to (ST) has cost $n$. Hence the approximation ratio $\alpha$ is
$$\alpha = \frac{2(n-1)}{2n} = 2 - \frac{1}{n}.$$
QUESTION: Why is this a ``tight" example?
\paragraph{}
For several optimization problems, optimizing on $G$ or $G^M$ is equivalent.
\paragraph{}
A {\it tree metric} $(V,d^T)$ is a metric that arises from a weighted tree $T$. Many optimization problems are easy on trees. For example minimum Steiner Tree is very easy, taking only the nodes needed to connect the terminals. So then, given a generic metric $(V,d)$ can we find a tree metric $(V,d^T)$ such that
$$d(u,v) \leq d^T(u,v) \leq \alpha d(u,v)$$
for any $u,v \in V$ for some small {\it distortion factor} $\alpha$?
\paragraph{}
Some bad news: For a metric arising from an $n$-cycle (simplest imaginable example), any tree-metric must have distortion $\Omega(n)$. This is easy to see for spanning tree distance (any removed edge has their distance increase from $1$ to $n-1$ in unit-weighted case), but the observation holds in general. So then, what if we take a spanning tree randomly? In other words, do the following:
\begin{enumerate}
\item Take uniformly at random one edge from the cycle
\item Output tree $T$ obtained by deleting chosen edge.
\end{enumerate}
We will now attempt to bound $E[d^T(u,v)]$. Let $u,v \in V$ and assume $d(u,v) = k$. Then (consider $k$ out of $n$ edges break shortest path, $n-k$ don't)
$$E[d^T(u,v)] = \frac{k}{n}(n-k) + (1-\frac{k}{n})k \leq 2k$$
and our expected distortion factor is thus $\alpha = 2$.
\begin{theorem}
Given any metric $(V,d)$ one can randomly find a tree $T$ such that
\begin{enumerate}
\item $d(u,v) \leq d^T(u,v)$ for all $u,v\in V$
\item $E[d^T(u,v)] \leq O(\log n) d(u,v)$ for all $u,v\in V$.
\end{enumerate}
\end{theorem}
\begin{proof}
We may assume $\max_{u,v \in V} d(u,v) = 2^\delta$ for some $\delta \in \N$ and $d(u,v) > 1$ for all $u,v\in V$. Let $\cA$ be the algorithm given by
\begin{enumerate}
\item Choose randomly a permutation $\pi$ of $V$.
\item Choose $\beta \in [0,1]$ uniformly at random.
\item Set $D_\delta = \{V\}$.
\item For $i = \delta -1$ to $0$ do:
\begin{enumerate}
\item Assign every node to the first node according to $\pi$ that has distance at most $2^\beta \cdot 2^{i-1}$.
\item All nodes assigned to the same node that are currently in the same cluster will form a new cluster in $D_i$.
\end{enumerate}
\end{enumerate}
(TO BE CONTINUED)
\end{proof}
\end{document}
