\documentclass[letterpaper,12pt,oneside,onecolumn]{article}
\usepackage[margin=1in, bottom=1in, top=1in]{geometry} %1 inch margins
\usepackage{amsmath, amssymb, amstext}
\usepackage{fancyhdr}
\usepackage{mathtools}
\usepackage{algorithm}
\usepackage{algpseudocode}
\usepackage{theorem}
\usepackage{tikz}
\usepackage{tkz-berge}

%Macros
\newcommand{\A}{\mathbb{A}} \newcommand{\C}{\mathbb{C}}
\newcommand{\D}{\mathbb{D}} \newcommand{\F}{\mathbb{F}}
\newcommand{\N}{\mathbb{N}} \newcommand{\R}{\mathbb{R}}
\newcommand{\T}{\mathbb{T}} \newcommand{\Z}{\mathbb{Z}}
\newcommand{\Q}{\mathbb{Q}}
 
 
\newcommand{\cA}{\mathcal{A}} \newcommand{\cB}{\mathcal{B}}
\newcommand{\cC}{\mathcal{C}} \newcommand{\cD}{\mathcal{D}}
\newcommand{\cE}{\mathcal{E}} \newcommand{\cF}{\mathcal{F}}
\newcommand{\cG}{\mathcal{G}} \newcommand{\cH}{\mathcal{H}}
\newcommand{\cI}{\mathcal{I}} \newcommand{\cJ}{\mathcal{J}}
\newcommand{\cK}{\mathcal{K}} \newcommand{\cL}{\mathcal{L}}
\newcommand{\cM}{\mathcal{M}} \newcommand{\cN}{\mathcal{N}}
\newcommand{\cO}{\mathcal{O}} \newcommand{\cP}{\mathcal{P}}
\newcommand{\cQ}{\mathcal{Q}} \newcommand{\cR}{\mathcal{R}}
\newcommand{\cS}{\mathcal{S}} \newcommand{\cT}{\mathcal{T}}
\newcommand{\cU}{\mathcal{U}} \newcommand{\cV}{\mathcal{V}}
\newcommand{\cW}{\mathcal{W}} \newcommand{\cX}{\mathcal{X}}
\newcommand{\cY}{\mathcal{Y}} \newcommand{\cZ}{\mathcal{Z}}

\newcommand\numberthis{\addtocounter{equation}{1}\tag{\theequation}}


\newenvironment{proof}{{\bf Proof:  }}{\hfill\rule{2mm}{2mm}}
\newenvironment{proofof}[1]{{\bf Proof of #1:  }}{\hfill\rule{2mm}{2mm}}
\newenvironment{proofofnobox}[1]{{\bf#1:  }}{}\newenvironment{example}{{\bf Example:  }}{\hfill\rule{2mm}{2mm}}

%\renewcommand{\thesection}{\lecnum.\arabic{section}}
%\renewcommand{\theequation}{\thesection.\arabic{equation}}
%\renewcommand{\thefigure}{\thesection.\arabic{figure}}

\newtheorem{fact}{Fact}[section]
\newtheorem{lemma}[fact]{Lemma}
\newtheorem{theorem}[fact]{Theorem}
\newtheorem{definition}[fact]{Definition}
\newtheorem{corollary}[fact]{Corollary}
\newtheorem{proposition}[fact]{Proposition}
\newtheorem{claim}[fact]{Claim}
\newtheorem{exercise}[fact]{Exercise}
\newtheorem{note}[fact]{Note}
\newtheorem{conjecture}[fact]{Conjecture}

\newcommand{\size}[1]{\ensuremath{\left|#1\right|}}
\newcommand{\ceil}[1]{\ensuremath{\left\lceil#1\right\rceil}}
\newcommand{\floor}[1]{\ensuremath{\left\lfloor#1\right\rfloor}}

%END MACROS
%Page style
\pagestyle{fancy}

\listfiles

\raggedbottom

\lhead{2017-04-04}
\rhead{W. Justin Toth CO750-Approximation Algorithms Report} %CHANGE n to ASSIGNMENT NUMBER ijk TO COURSE CODE
\renewcommand{\headrulewidth}{1pt} %heading underlined
%\renewcommand{\baselinestretch}{1.2} % 1.2 line spacing for legibility (optional)

\begin{document}
\paragraph{Paper Title:} Popularity, Mixed Matchings, and Self-duality
\paragraph{Authors:} Chien-Chun Huang and Telikepalli Kavitha
\paragraph{Reference:} \cite{huang2017popularity}

\section{Preliminaries}
\paragraph{}
We will begin by introducing the definitions of the main objects of study, and some useful notation. Then we will state the theorems to be shown in the later sections of this report.
\paragraph{}
As in most problems in the stable matching family we are given a graph $G=(V,E)$ where each vertex has ranking over its neighbours. We will assume throughout that these preference orders are strict, and a vertex prefers any neighbour to being unmatched. We will denote by $N(v)$ the set of neighbours of vertex $v$, and by $\delta(v)$ the set of edges incident with $v$. We will use the notation $>_v$ to describe $v$'s preference order over $N(v)$. Precisely for $u, w \in N(v)$, we write $u >_v w$ if $u$ is preferred by $v$ to $w$. We will use $\geq_v$, $\leq_v$, $=_v$, and $<_v$ analogously.
\paragraph{}
One way to think of $\textit{popular}$ matchings is as matchings which never lose an election to any other matching. Since we will speak often about partners in matchings, we introduce the notation: if $M$ is a matching then $M(u) = v$ provided $uv \in M$. For any $u \in V$ and $v,v' \in N(v)$ we define $vote_u(v,v')$ as
$$vote_u(v,v') := \begin{cases}
1, \text{ if } v >_u v' \\
-1, \text{ if } v <_u v' \\
0, \text{ otherwise (ie. } v = v').
\end{cases}$$
To decide the winner of the election we some the votes. The aggregate voting between two matchings $M$ and $M'$ is
$$\Delta(M, M') = \sum_{u \in V} vote_u(M(u), M'(u)).$$
Observe that $\Delta(M,M') \geq 0$ if at least as many vertices prefer their partner in $M$ to their partner in $M'$. A $\textit{popular}$ matching $M$ is  a matching where
$$\Delta(M, M') \geq 0$$
for any matching $M'$.
\paragraph{}
We are interested in the $\textit{max weight popular matching}$ problem in this report. That is, given a weight function $w: E \rightarrow \R_+$ find a popular matching $M$ on $G$ which maximizes $w(M) := \sum_{e \in M} w(e).$ A standard approach to such problems is to attempt to formulate them as linear programs. This will motivate our definitions of fractional popular matchings to follow. In this approach the major question is whether or not your linear programming formulation gives back integral solutions which can be used to solve the original combinatorial problem. In this paper we will prove that there is an integral formulation for a special case, and a $\frac{1}{2}$-integral formulation in general.
\paragraph{}
The fractional matching polytope $FM_{G}$ is defined as
$$FM_{G} := \{x \in \R_{\geq 0}^{E} : x(\delta(v)) \leq 1, \forall v \in V \}.$$
This object has been well-studied in combinatorial optimization. We want to study the analogous $\textit{popular fractional matching polytope}$ which we will call $P_G$. To formally define $P_G$ we will need to extend our notions related to voting as used in defining popular matchings. Let $x \in FM_G$. We define
$$vote_u(x, v') := \sum_{uv \in \delta(u)} x_{uv} vote_u(v,v') = x(\delta^{>v'}(u)) - x(\delta^{<v'}(u)),$$
where $\delta^{>v'}(u) := \{vu \in \delta(u) : v >_u v'\}$ denotes the set of edges $u$ prefers to $uv'$, and $\delta^{<v'}(u)$ is defined analogously.  We define $vote_u(v',x) = -vote_u(x,v')$. Now we have a means of voting between two fractional matchings $x,y \in FM_G$:
$$vote_u(x,y) := \sum_{uv' \in \delta(u)} y_{uv'}vote_u(x,v'), $$
and the total votes:
$$\Delta(x,y) := \sum_{u \in V} vote_u(x,y).$$
Now we can formally describe $P_G$ as 
$$P_G = \{ x\in FM_G: \Delta(x,y) \geq 0, \forall y \in FM_G\}.$$
\paragraph{}
We finally have sufficient definitions to describe the results of this paper. In section \ref{sec:formulation} we describe a useful extended formulation which describes $P_G$ that will come into play when studying popular fractional matchings throughout this work. In section \ref{sec:special} we build up the following theorem
\begin{theorem}\label{th:special}
Let $G = (A \cup B, E)$ be a bipartite graph with strict preferences. If $G$ admits a perfect stable matching the $P_G$ is integral.
\end{theorem}
Building upon this theorem we show that $\frac{1}{2}$-integrality holds in general in section \ref{sec:general}.
\begin{theorem}
Let $G=(V,E)$ be an arbitrary graph with strict preferences. Then the extreme points of $P_G$ lie in $\{0,\frac{1}{2}, 1\}^E$.
\end{theorem}
In section \ref{sec:complexity} we discuss some results showing that max weight popular matching is $NP$-hard and under the Unique Games Conjecture is difficult to approximate. We conclude with a section discussing open problems.
\section{Extended Formulation}\label{sec:formulation}

\paragraph{}
Here we present the extended formulation $P'_G$ which we will use to study the popular fractional matching polytope. The properties of $P'_G$ were first studied in \cite{kavitha2011popular}. For a more rigorous treatment of its connection to $P_G$ refer to that paper. Here we try to present ideas about how $P'_G$ was derived.

\paragraph{}
We begin by augmenting our graph $G$ with a set of last resort neighbours $\ell(u)$ for every $u \in V$. The vertices $\ell(u)$ are only connected to their respective vertex $u$. The preference lists of each $u$ are modified so that $\ell(u)$ is their last choice. That is for all $v \in N(u)$, $v \geq_u \ell(u)$. The vertex set and edge set now look like
$$V(G) = V \cup \bigcup_{u \in V} \ell(u) \quad\text{and}\quad E(G) = E \cup \bigcup_{u \in V} u\ell(u).$$  With the addition of the last resort vertices we may now assume that every $v \in V$ (the original vertices) is fully matched in any popular fractional matching.

\paragraph{}
Fix a fractional matching $x \in FM_G$.  We seek a linear program which characterizes when $x$ is popular. Let $w: E(G) \rightarrow \R_+$ be the weight function given by assigning  $$w(ab) := vote_a(b,x) + vote_b(a,x) = x(\delta^{<b}(a)) - x(\delta^{>b}(a)) + x(\delta^{<a}(b)) - x(\delta^{>a}(b))$$
for any $ab \in E$, and assigning
$$w(u\ell(u)) := vote_u(\ell(u),x) = -x(\delta^{>\ell(u)}(u)).$$
Observe that if $y \in FM_G$ fully matches each $u \in V$ then $\sum_{ab \in E(G)} w(ab)y_{ab} = \Delta(y,x).$ So the fractional matching $x$ is popular if and only if the optimal value of the following LP is $0$ (it cannot be less than $0$ as $x$ is a feasible solution, and $\Delta(x,x) = 0$) :
\begin{align*}
\max\  &\sum_{e \in E(G)} w(e) y_{e} \\
\text{s.t.}\ y(\delta(u)) &= 1 &\forall u \in V \\
y_e &\geq 0 &\forall e \in E(G).
\end{align*}
Consider the dual to previous linear program, to be denoted $(LP1)$:
\begin{align*}
\min\ &\sum_{u \in V} \alpha_u \\
\text{s.t.}\ \alpha_a + \alpha_b &\geq w(ab) &\forall ab \in E \\
\alpha_u &\geq w(u\ell(u)) &\forall u \in V.
\end{align*}
If $x$ is popular then the optimal value of $(LP1)$ is $0$. Also note that if $x$ is not popular then the optimal value is greater than $0$. So one means of finding popular fractional matchings would be to let $x$ vary as a fractional matching and minimize $\sum_{u} \alpha_u$. We call that linear program $(LP2)$:
\begin{align*}
\min\ &\sum_{u \in V} \alpha_u \\
\text{s.t.}\ \alpha_a + \alpha_b &\geq w(ab) &\forall ab \in E \\
\alpha_u &\geq w(u\ell(u)) &\forall u \in V \\
x(\delta(u)) &= 1 &\forall u \in V \\
x_e &\geq 0 &\forall e \in E(G).
\end{align*}
As discussed above, the optimal value of $(LP2)$ is $0$, and so a corresponding optimal solution for $(LP2)$ is a vector $(x,\alpha) \in \R^{E(G)}\times \R^{V}$ where $x$ is a popular fractional matching and $\sum_{u \in V} \alpha_u = 0$. Such $\alpha$ is called a $\textit{witness}$ to the popularity of $x$. The polytope of optimal solutions of $(LP2)$ is what we will use as our extended formulation for the popular fractional matching polytope. We can describe it as $P'_G$, the set of solutions to the set of linear inequalities:
\begin{align*}
\sum_{u \in V} \alpha_u &= 0\\
\alpha_a + \alpha_b &\geq vote_a(b,x) + vote_b(a,x) &\forall ab \in E \\
\alpha_u &\geq vote_u(\ell(u),x) &\forall u \in V \\
x(\delta(u)) &= 1 &\forall u \in V \\
x_e &\geq 0 &\forall e \in E(G).
\end{align*}
Thus $P'_G$ involves $O(|E| + |V|)$ constraints, and hence provides a polynomial size description of popular fractional matchings. The following lemma demonstrates a very interesting property of $(LP2)$
\begin{lemma}\label{lemma:self-dual}
The linear program $(LP2)$ is its own dual program program. That is $(LP2)$ is self-dual.
\end{lemma}
\begin{proof}
Consider the dual to $(LP2)$ in variables $y \in \R^{E(G)}$ and $\beta \in \R^V$:
\begin{align*}
\max\ &\sum_{u \in V} \beta_u \\
\text{s.t.} \beta_a + \beta_b + y(\delta^{<b}(a))  - y(\delta^{>b}(a)) + y(\delta^{<a}(b)) - y(\delta^{>a}(b)) &\leq 0 &\forall ab \in E \\
\beta_u - y(\delta^{>\ell(u)}(u)) &\leq 0 &\forall u \in V \\
y(\delta(u)) &= 1 &\forall u \in V \\
y_e &\geq 0 &\forall e \in E(G).
\end{align*}
If we perform the substitutions $\alpha_u = -\beta_u$ for each $u \in V$ and $x_e = y_e$ for each $e \in E(G)$ then the above dual LP becomes exactly the same as $(LP2)$.
\end{proof}
\paragraph{}
This lemma leads to the following very useful lemma, which follows immediately from the previous result via complementary slackness.
\begin{lemma}\label{lemma:self-dual-cs}
Let $(x, \alpha^x) \in P'_G$.  For every $ab \in E$, if $x_{ab} > 0$ then the ``covering constraint" in $P'_G$ for $ab$ is tight. Formally we have
$$\alpha_a^x + \alpha_b^x = x(\delta^{<b}(a)) - x(\delta^{>b}(a)) + x(\delta^{<a}(b)) - x(\delta^{>a}(b)).$$
\end{lemma}
\section{Integrality of $P_G$ in a special case}\label{sec:special}
\paragraph{}
In this section we turn our attention to theorem \ref{th:special}. So suppose that our graph $G$ is bipartite with bipartition $A \cup B$. Also suppose that $G$ admits a perfect stable matching. This condition actually implies that every popular matching is perfect. This follows since stable matchings always match the same set of vertices, and stable matchings are minimum size popular matchings \cite{huang2011popular}. It is not hard to see that this extends to popular fractional matchings, meaning every popular fractional matching fully matches each $u \in V$ to its true neighbours (not last resort vertices). 
\paragraph{}
Let $(x, \alpha^x) \in P'_G$. We want to write $x$ as a convex combination of popular matchings. First we want bounds on the values the witness vector $\alpha^x$ takes
\begin{lemma}
For every $u \in V$, $\alpha_u^x \in [-1,1]$.
\end{lemma}
\begin{proof}First
$$\alpha^x_u \geq vote_u(\ell(u), x) = -1 \quad\text{(since }x_{u,\ell(u)} = 0\text{).}$$
Now let $v$ be the least preferred neighbour of $u$ such that $x_{uv} >0$. Then
$$vote_u(v,x) = -(1-x_{uv}).$$
Also
$$vote_v(u,x) = x(\delta^{<u}(v))-x(\delta^{>u}(v))= 1-x_{uv}.$$
So by lemma \ref{lemma:self-dual-cs},
$$\alpha_u + \alpha_v = vote_u(v,x) + vote_v(u,x) = -(1-x_{uv}) + (1-x_{uv}) = 0.$$
Since $\alpha_v \geq -1$ this implies $\alpha_u \leq 1$.
\end{proof}
\paragraph{}
Thinking ahead, we want to write $x$ as a convex combination of popular matchings. In doing this we will use a witness vector in $\{-1,1\}^V$ to certify the popularity of each chosen matching. From our previous lemma we can think of each $\alpha^x_u$ as a convex combination of $-1$ and $1$. For each $a \in A$ let $r_a$ denote the fraction of popular matchings in the convex combination forming $x$ whose $\alpha_a^x$ value is assigned $1$. Similarly define for each $b \in B$ the value $r_b$ to be the fraction assigning $\alpha^x_b$ to $-1$. Precisely we have $2r_a -1 = \alpha^x_a$ and $1-2r_b = \alpha^x_b$.
\paragraph{}
Let $X_a$ be the array for the assignment of $a\in A$ in $x$. Each cell of $X_a$ corresponds to some $b \in N(a)$ and the length of the cell is $x_{ab} > 0$ (we omit $0$-length cells). The cells in $X_a$ are arranged in increasing order with respect to $a$'s preferences. The first $r_a$ fraction of $X_a$ is called the $\textit{positive}$ subarray and the remaing the $\textit{negative}$ subarray. We split and reorder the arrays as shown in the following figure to obtain $X'_a$
\begin{figure}[H]
\centering
\includegraphics[scale=0.25]{Xa}
\end{figure}
We do the same thing for each $b \in B$, creating $X_b$. But the only difference is that we arrange $X_b$'s cell in decreasing order of $b$'s preferences. Also notice that since $r_b$ corresponds to the negative $\alpha_b$ cells, the array begins with the negative subarray. We again split and reorder to obtain $X'_b$ as shown
\begin{figure}[H]
\centering
\includegraphics[scale=0.25]{Xb}
\end{figure}
One interesting thing about our choice of the $r_a$'s and $r_b$'s is that they allow us to rewrite our covering constraints for $ab$ in a simpler form.
\begin{lemma}\label{lemma:covering}
For any $ab \in E$ we have
$$x_{ab} + x(\delta^{<b}(a)) + x(\delta^{<a}(b)) \leq r_a + (1-r_b),$$
and if $x_{ab} > 0$ then the above is tight.
\end{lemma}
\begin{proof}
Take the constraint
$$ \alpha^x_a + \alpha^x_b \geq x(\delta^{<b}(a)) - x(\delta^{>b}(a)) + x(\delta^{<a}(b)) - x(\delta^{>a}(b)) $$
and rewrite using $\alpha^x_a = 2r_a-1$ and $\alpha^x_b = 1-2r_b$. Further substitute $-x(\delta^{>b}(a)) = -(1 -x_{ab} - x(\delta^{<b}(a)))$ and $-x(\delta^{>a}(b)) = -(1- x_{ab} - x(\delta^{<a}(b)))$. After simplifying the inequality in the lemma statement is obtained. By lemma \ref{lemma:self-dual-cs} this inequality is tight when $x_{ab} > 0$.
\end{proof}

\paragraph{}
Now we show how to obtain popular matchings whose convex hull contains $x$. Form a table $T$ by stacking each $X'_u$ for $u \in V$ in rows. The width of $T$ is $1$, so for any $t \in [0,1)$ we can define a set of edges $M_t$ as follows:
\begin{enumerate}
\item Form a vertical line at distance $t$ from the left end of $T$
\item For each $u \in V$ the vertical line intersects (or touches the left boundary of) a cell in $X'_u$. We will denote this cell by $c_u(t)$.
\item Form the set $M_t$ as $M_t := \{uv : u \in V \text{ and } uv \text{ corresponds to cell } c_u(t)\}$.
\end{enumerate}
\paragraph{}
We need to verify two things: the first is that $M_t$ is a matching, and the second is that $M_t$ is indeed popular. This will be done in the following two subsections. Supposing we are successful in the promised arguments to follow, we will show how this allows us to find a set of popular matchings for which $x$ is a convex combination. Execute the following algorithm:
\begin{enumerate}
\item Initialize $i=0$.
\item Sweep a vertical line from left to right across $T$. Denote its distance from the left by $t$. 
\item Whenever a new cell is encountered fix matching $M^i = M_t$ and fix $t_i = t$. 
\end{enumerate}
Say we find $k$ such matchings. We can then write $x$ as:
$$x = t_1\chi(M^0) + (t_2 - t_1)\chi(M^1) + \dots + (1 - t_{k-1}) \chi(M^{k-1})$$
where $\chi(\cdot)$ if the function mapping sets to incidence vectors.

\subsection{$M_t$ is a matching}

\subsection{$M_t$ is popular}
\section{$\frac{1}{2}$-Integrality of $P_G$ in general}\label{sec:general}

\section{Complexity Results}\label{sec:complexity}

\section{Conclusions}

\bibliography{references}
\bibliographystyle{plain}
\end{document}
