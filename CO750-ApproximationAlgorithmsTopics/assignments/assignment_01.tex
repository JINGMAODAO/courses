\documentclass[letterpaper,12pt,oneside,onecolumn]{article}
\usepackage[margin=1in, bottom=1in, top=1in]{geometry} %1 inch margins
\usepackage{amsmath, amssymb, amstext}
\usepackage{fancyhdr}
\usepackage{mathtools}
\usepackage{algorithm}
\usepackage{algpseudocode}
\usepackage{theorem}
\usepackage{tikz}
\usepackage{tkz-berge}

%Macros
\newcommand{\A}{\mathbb{A}} \newcommand{\C}{\mathbb{C}}
\newcommand{\D}{\mathbb{D}} \newcommand{\F}{\mathbb{F}}
\newcommand{\N}{\mathbb{N}} \newcommand{\R}{\mathbb{R}}
\newcommand{\T}{\mathbb{T}} \newcommand{\Z}{\mathbb{Z}}
\newcommand{\Q}{\mathbb{Q}}
 
 
\newcommand{\cA}{\mathcal{A}} \newcommand{\cB}{\mathcal{B}}
\newcommand{\cC}{\mathcal{C}} \newcommand{\cD}{\mathcal{D}}
\newcommand{\cE}{\mathcal{E}} \newcommand{\cF}{\mathcal{F}}
\newcommand{\cG}{\mathcal{G}} \newcommand{\cH}{\mathcal{H}}
\newcommand{\cI}{\mathcal{I}} \newcommand{\cJ}{\mathcal{J}}
\newcommand{\cK}{\mathcal{K}} \newcommand{\cL}{\mathcal{L}}
\newcommand{\cM}{\mathcal{M}} \newcommand{\cN}{\mathcal{N}}
\newcommand{\cO}{\mathcal{O}} \newcommand{\cP}{\mathcal{P}}
\newcommand{\cQ}{\mathcal{Q}} \newcommand{\cR}{\mathcal{R}}
\newcommand{\cS}{\mathcal{S}} \newcommand{\cT}{\mathcal{T}}
\newcommand{\cU}{\mathcal{U}} \newcommand{\cV}{\mathcal{V}}
\newcommand{\cW}{\mathcal{W}} \newcommand{\cX}{\mathcal{X}}
\newcommand{\cY}{\mathcal{Y}} \newcommand{\cZ}{\mathcal{Z}}

\newcommand\numberthis{\addtocounter{equation}{1}\tag{\theequation}}


\newenvironment{proof}{{\bf Proof:  }}{\hfill\rule{2mm}{2mm}}
\newenvironment{proofof}[1]{{\bf Proof of #1:  }}{\hfill\rule{2mm}{2mm}}
\newenvironment{proofofnobox}[1]{{\bf#1:  }}{}\newenvironment{example}{{\bf Example:  }}{\hfill\rule{2mm}{2mm}}

%\renewcommand{\thesection}{\lecnum.\arabic{section}}
%\renewcommand{\theequation}{\thesection.\arabic{equation}}
%\renewcommand{\thefigure}{\thesection.\arabic{figure}}

\newtheorem{fact}{Fact}[section]
\newtheorem{lemma}[fact]{Lemma}
\newtheorem{theorem}[fact]{Theorem}
\newtheorem{definition}[fact]{Definition}
\newtheorem{corollary}[fact]{Corollary}
\newtheorem{proposition}[fact]{Proposition}
\newtheorem{claim}[fact]{Claim}
\newtheorem{exercise}[fact]{Exercise}
\newtheorem{note}[fact]{Note}
\newtheorem{conjecture}[fact]{Conjecture}

\newcommand{\size}[1]{\ensuremath{\left|#1\right|}}
\newcommand{\ceil}[1]{\ensuremath{\left\lceil#1\right\rceil}}
\newcommand{\floor}[1]{\ensuremath{\left\lfloor#1\right\rfloor}}

%END MACROS
%Page style
\pagestyle{fancy}

\listfiles

\raggedbottom

\lhead{2017-02-09}
\rhead{William Justin Toth CO750-Approximation Algorithms Assignment 1} %CHANGE n to ASSIGNMENT NUMBER ijk TO COURSE CODE
\renewcommand{\headrulewidth}{1pt} %heading underlined
%\renewcommand{\baselinestretch}{1.2} % 1.2 line spacing for legibility (optional)

\begin{document}
%Q1
\section{}
\paragraph{}
Let $E$ be the edge set of a graph $G$. Let $\cS$ be the collection of all sets $S \subseteq E$ that are matchings of $G$. Let $w: E \rightarrow \R_+$ be a weight function on the edges. For a subset $\bar{E} \subseteq E$ we denote by $G[\bar{E}]$ the subgraph of $G$ induced by the edges in $\bar{E}$.
	%Pa
\subsection{a}
\paragraph{}
Let $\bar{E} \subseteq E$. Suppose that $M \subseteq \bar{E}$ is an inclusion-wise maximal matching of $G[\bar{E}]$. Let $\bar{M}$ be a matching of $G[\bar{E}]$. Let $f : M \rightarrow \{ \{e_1,e_2\} | e_1, e_2 \in \bar{M}\}$ be defined by
$$ f(uv) = \{e_1, e_2\}$$
where $e_1 \in \bar{M}$ covers $u$ and $e_2 \in \bar{M}$ covers $v$, for each $uv \in M$.
\paragraph{}
We claim that for all $uv \in \bar{M}$, there exists $P \in Im(f)$ (where $Im(f)$ denotes the image of $f$) such that $uv \in P$. To see this suppose for a contradiction that there exists some $uv \in \bar{M}$ where each $P \in Im(f)$ does not contain $uv$. Then by the definition of $f$, neither $u$ nor $v$ are covered by $M$. But this violates that $M$ is inclusion-wise maximal since $uv \not\in M$ and $M \cup \{uv\}$ is a matching. Hence the claim holds.
\paragraph{}
From the previous claim we observe $|\bar{M}| \leq | \cup_{P \in Im(f)} P|$. But this yields a string of inequalities which complete the proof:
\begin{align*}
|\bar{M}| &\leq |\cup_{P \in Im(f)} P| \\
&\leq \sum_{P \in Im(f)} |P| \\
&\leq \sum_{P \in Im(f)} 2 \\
&\leq 2|Im(f)| \\
&\leq 2|M|. 
\end{align*}
With the last inequality following since $M$ is the domain of $f$. Hence we have $|M| \geq \frac{1}{2}|\bar{M}|$ as desired. $\blacksquare$
	%Pb
\subsection{b}
\paragraph{}
Let $e$

%Q2
\section{}

%Q3
\section{}
\paragraph{}
Let $G = (V,E)$ be a graph. For any $S, T \subseteq V$ we denote by $E(S,T)$ the set of edges between $S$ and $T$. Formally $E(S,T) = \{\{s,t\} \in E: s \in S, t \in T\}$.
%Pa
\subsection{a}
\paragraph{}
The fact that the Greedy Algorithm runs in polynomial time is obvious. $|E(v,S)|$ can be computed in $O(|E|)$ time and the while loop runs for $O(|V|)$ iterations. Further it clearly returns a cut. It remains to verify the approximation factor of $\frac{1}{2}$ holds.
\paragraph{}
We show that the approximation factor is maintained throughout operation of the algorithm on the graph $G[V\backslash \bar{V}]$ induced by vertices considered so far. After the first iteration only one vertex has been considered and the approximation factor holds trivially on the graph induced by that one vertex. Now for induction let $V' = V \backslash \bar{V}$ be the set of vertices considered so far by the Greedy Algorithm, and at the start of the next iteration the vertex $v \in V$ is being considered. Let $S \subseteq V$ be the cut set considered so far, and let $S^*$ be the optimal cut set on $G[V']$. Let $\bar{S} = V'\backslash S$ and $\bar{S^*} = V' \backslash S^*$. We may assume without loss of generality that the Greedy Algorithm puts $v \in S$ and the optimal solution puts $v \in S^*$ after this iteration (if this does not hold simply relabel $S$ with $\bar{S}$ below, or $S^*$ with $\bar{S^*}$ with respect to which assumption does not hold). 
\paragraph{}
So after this iteration the optimal solution has value
$$|E(S^* \cup \{v\}, \bar{S^*})| =  \sum_{u \in S^* \cup \{v\}} |E(u, \bar{S^*})| = |E(v,\bar{S^*})| + \sum_{u \in S^*} |E(u, \bar{S^*})| \leq |E(v,\bar{S^*})| + 2\sum_{u \in S} |E(u, \bar{S})|$$
with the inequality following by induction. Hence the approximation factor will hold as desired provided that
$$|E(v,\bar{S^*})| \leq 2|E(v,\bar{S})|.$$
Suppose for a contradiction that
$$|E(v,\bar{S^*})| > 2|E(v,\bar{S})|.$$
We observe that
$$|E(v, \bar{S^*})| = |E(v,\bar{S^*}\backslash S)| + |E(v, \bar{S^*}\cap S)| \leq  |E(v, \bar{S})| + |E(v, S)|.$$
The inequality follows since $\bar{S^*}\backslash S \subseteq V' \backslash S = \bar{S}$, and $\bar{S^*} \cap S \subseteq S$. Combining this inequality with the contradiction assumption we see
$$2|E(v,\bar{S})| <|E(v, \bar{S})| + |E(v, S)|$$
and subtracting $|E(v, \bar{S})|$ from both sides yields
$$ |E(v,\bar{S})| < |E(v,S)|.$$
This contradicts our Greedy choice $v \in S$ since such choice implies
$$ |E(v,\bar{S})| \geq |E(v,S)|.$$
Hence we have
$$|E(v,\bar{S^*})| \leq 2|E(v,\bar{S})|.$$
Thus after this iteration the optimal solution has value:
$$|E(S^* \cup \{v\}, \bar{S^*})|  \leq |E(v,\bar{S^*})| + 2\sum_{u \in S} |E(u, \bar{S})| \leq 2|E(v,\bar{S})| + 2\sum_{u \in S} |E(u, \bar{S})| = 2 |E(S \cup \{v\}, \bar{S})|.$$
Therefore $|E(S \cup \{v\}, \bar{S})| \geq \frac{1}{2} |E(S^* \cup \{v\}, \bar{S^*})|$ as desired. 
\paragraph{}
Now we observe that, from the invariant we just demonstrated, upon termination of the algorithm the approximation factor holds for the greedy solution versus the optimal solution on $G[V \backslash \emptyset] = G$, and hence the Greedy Algorithm is $\frac{1}{2}$-approximation algorithm.$\blacksquare$
%Pb
\subsection{b}
\paragraph{}
Our goal is to find a maximum $k$-Cut of $G$. That is, we seek to find a partition of $V$ into $V_1, \dots, V_k$ maximizing
$$\sum_{i=1}^k \sum_{j = 1}^{i-1} |E(V_i, V_j)| := c(\{V_1, \dots, V_k\}).$$
Intuitively we are maximizing the edges between the selected cuts. Let $\cB$ denote the $2$-approximation algorithm for max cut given in problem $3a$. Consider the following algorithm, which we will denote by $\cA$:
\begin{enumerate}
\item Set $V_1 = S$ and $V_2 = V\backslash S$ where $S$ is the cut returned by $\cB$ when run on $G$.
\item Set $V_i = \emptyset$ for $i = 3, \dots, k$.
\item For $i = 2, \dots, k-1$
\begin{enumerate}
\item For $j = 1, \dots, i$ set $S_j$ to the cut returned by $\cB$ when run on $G[V_j]$.
\item Set $j^* = \text{argmax}_{j=1,\dots, i} \{|E(S_j, V_j\backslash S_j)|\}$.
\item Set $V_{i+1} = S_{j^*}$ and set $V_{j^*} = V_{j^*} \backslash S_{j^*}$.
\end{enumerate}
\item Return $V_1, \dots, V_k$.
\end{enumerate}
The idea behind the operation of $\cA$ is to start with a $2$-cut of $G$, and in every successive iteration take the current partition and $2$-cut one of the parts in an approximately optimal way until we reach a $k$-cut.
\begin{lemma}\label{lemma:3b1}
Let $\cO = \{V^*_1, \dots, V^*_k\}$. Consider an arbitrary iteration $i = 1, \dots, k-1$ (iteration $1$ is original $2$-cut). Let $\cS = \{V_1, \dots, V_{i+1}\}$ be the set of partitions chosen so far by $\cA$. Suppose that at the end of iteration $i$, the algorithm has split $V_j$ into $V_j'$ and $V_{i+1}$. That is the solution after iteration $i$ is
$$\cS' = \cS \backslash V_j \cup \{V_j', V_{i+1}\}.$$
Then 
$$2k(c(\cS') - c(\cS)) \geq c(\cO) - c(\cS).$$
\end{lemma}
\begin{proof}
The difference between $c(\cS')$ and $c(\cS)$ is precisely $|E(V_j', V_{i+1})|$ (the number of edges gained by splitting $V_j$). Further the difference $c(\cO) - c(\cS)$ is upper-bounded by the edges counted in $c(\cO)$ that are not counted in $c(\cS)$. Call such edges $E'$.  The edges $E'$ are necessarily internal to a part of $\cS$ (that is, both endpoints lie in some $V_k \in \cS$). Otherwise they are counted in $c(\cS)$ since their endpoints lie in distinct parts of $\cS$. Thus if we consider the $k$-cut induced by $\cO$ on $G[E']$ it consists of at most $k$ splits ($2$-cuts) of parts $V_k \in \cS$.
\paragraph{}
Let $V'_k$ denote the best possible $2$-cut of $V_k \in \cS$ for all $k = 1, \dots, i$. Now let $$k^* = \text{argmax}_{k=1,\dots,i}  \{|E(V'_k, V_k \backslash V'_k)|\}.$$ We have
$$ |E'| \leq k |E(V'_{k^*}, V_k \backslash V'_{k^*})|$$
by our previous discussion, since $E'$ edges are formed by a $k$-cut on edges internal to parts of $\cS$, and  $|E(V'_{k^*}, V_k \backslash V'_{k^*}|$ is largest $2$-cut to be formed by edges internal to a part of $\cS$. Now the cut found by our algorithm $\cA$ in this iteration in step $3(c)$ is a $2$-approximation of the best possible split cut. That is,
$$|E(V'_{k^*}, V_k \backslash V'_{k^*})| \leq 2|E(V_j', V_{i+1})| .$$
Hence, combining these inequalities, we see that
$$|E'| \leq 2k |E(V_j', V_{i+1})|.$$
Therefore $2k(c(\cS') - c(\cS)) \geq c(\cO) - c(\cS)$ as desired.
\end{proof}
\paragraph{}
Using the previous lemma, we can demonstrate that $\cA$ is a $\alpha := 1-(1-\frac{1}{2k})^k$-approximation algorithm for the max $k$-cut problem.
\paragraph{Main Proof}
Since $\cB$ can be run in polynomial time, and $k$ is at most $|V|$, it is clear that $\cA$ runs in polynomial time. At termination the algorithm returns a partition of $V$ into $k$ parts so it returns a feasible solution.
\paragraph{} 
Now to see the approximation factor holds let $\cS^i$ denote the solution maintained by $\cA$ at the end of iteration $i$. Then the returned solution is $\cS^{k-1}$. Let $\cO$ be an optimal solution. Observe from Lemma \ref{lemma:3b1} that $c(\cS^{i}) \geq \frac{1}{2k} c(\cO) + (1- \frac{1}{2k})c(S^{i-1})$. We will apply this inequality inductively to achieve the bound:
\begin{align*}
c(\cS^{k-1} &\geq \frac{1}{2k} c(\cO) + (1- \frac{1}{2k})c(S^{k-2}) \\
		&\geq \frac{1}{2k} c(\cO) + (1- \frac{1}{2k})(\frac{1}{2k} c(\cO) + (1- \frac{1}{2k})c(S^{k-3})) \\
		&\geq \frac{1}{2k} c(\cO) + (1- \frac{1}{2k})(1 + (1-\frac{1}{2k}) + (1-\frac{1}{2k})^2 + \dots + (1-\frac{1}{2k})^{k-1}) \\
		&= \frac{c(\cO)}{2k} \cdot \frac{1-(1-\frac{1}{2k})^k}{1-(1-\frac{1}{2k})} \\
		&=\frac{c(\cO)}{2k} \cdot \frac{1-(1-\frac{1}{2k})^k}{\frac{1}{2k}}\\
		&= 1-(1-\frac{1}{2k})^kc(\cO).
\end{align*}
Thus the approximation factor holds as desired. $\blacksquare$
\end{document}
