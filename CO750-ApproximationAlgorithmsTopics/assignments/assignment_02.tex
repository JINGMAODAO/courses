\documentclass[letterpaper,12pt,oneside,onecolumn]{article}
\usepackage[margin=1in, bottom=1in, top=1in]{geometry} %1 inch margins
\usepackage{amsmath, amssymb, amstext}
\usepackage{fancyhdr}
\usepackage{mathtools}
\usepackage{algorithm}
\usepackage{algpseudocode}
\usepackage{theorem}
\usepackage{tikz}
\usepackage{tkz-berge}

%Macros
\newcommand{\A}{\mathbb{A}} \newcommand{\C}{\mathbb{C}}
\newcommand{\D}{\mathbb{D}} \newcommand{\F}{\mathbb{F}}
\newcommand{\N}{\mathbb{N}} \newcommand{\R}{\mathbb{R}}
\newcommand{\T}{\mathbb{T}} \newcommand{\Z}{\mathbb{Z}}
\newcommand{\Q}{\mathbb{Q}}
 
 
\newcommand{\cA}{\mathcal{A}} \newcommand{\cB}{\mathcal{B}}
\newcommand{\cC}{\mathcal{C}} \newcommand{\cD}{\mathcal{D}}
\newcommand{\cE}{\mathcal{E}} \newcommand{\cF}{\mathcal{F}}
\newcommand{\cG}{\mathcal{G}} \newcommand{\cH}{\mathcal{H}}
\newcommand{\cI}{\mathcal{I}} \newcommand{\cJ}{\mathcal{J}}
\newcommand{\cK}{\mathcal{K}} \newcommand{\cL}{\mathcal{L}}
\newcommand{\cM}{\mathcal{M}} \newcommand{\cN}{\mathcal{N}}
\newcommand{\cO}{\mathcal{O}} \newcommand{\cP}{\mathcal{P}}
\newcommand{\cQ}{\mathcal{Q}} \newcommand{\cR}{\mathcal{R}}
\newcommand{\cS}{\mathcal{S}} \newcommand{\cT}{\mathcal{T}}
\newcommand{\cU}{\mathcal{U}} \newcommand{\cV}{\mathcal{V}}
\newcommand{\cW}{\mathcal{W}} \newcommand{\cX}{\mathcal{X}}
\newcommand{\cY}{\mathcal{Y}} \newcommand{\cZ}{\mathcal{Z}}

\newcommand\numberthis{\addtocounter{equation}{1}\tag{\theequation}}


\newenvironment{proof}{{\bf Proof:  }}{\hfill\rule{2mm}{2mm}}
\newenvironment{proofof}[1]{{\bf Proof of #1:  }}{\hfill\rule{2mm}{2mm}}
\newenvironment{proofofnobox}[1]{{\bf#1:  }}{}\newenvironment{example}{{\bf Example:  }}{\hfill\rule{2mm}{2mm}}

%\renewcommand{\thesection}{\lecnum.\arabic{section}}
%\renewcommand{\theequation}{\thesection.\arabic{equation}}
%\renewcommand{\thefigure}{\thesection.\arabic{figure}}

\newtheorem{fact}{Fact}[section]
\newtheorem{lemma}[fact]{Lemma}
\newtheorem{theorem}[fact]{Theorem}
\newtheorem{definition}[fact]{Definition}
\newtheorem{corollary}[fact]{Corollary}
\newtheorem{proposition}[fact]{Proposition}
\newtheorem{claim}[fact]{Claim}
\newtheorem{exercise}[fact]{Exercise}
\newtheorem{note}[fact]{Note}
\newtheorem{conjecture}[fact]{Conjecture}

\newcommand{\size}[1]{\ensuremath{\left|#1\right|}}
\newcommand{\ceil}[1]{\ensuremath{\left\lceil#1\right\rceil}}
\newcommand{\floor}[1]{\ensuremath{\left\lfloor#1\right\rfloor}}

%END MACROS
%Page style
\pagestyle{fancy}

\listfiles

\raggedbottom

\lhead{2017-03-14}
\rhead{W. Justin Toth CO750-Approximation Algorithms Assignment 2} %CHANGE n to ASSIGNMENT NUMBER ijk TO COURSE CODE
\renewcommand{\headrulewidth}{1pt} %heading underlined
%\renewcommand{\baselinestretch}{1.2} % 1.2 line spacing for legibility (optional)

\begin{document}

%Question 1
\section{}
\paragraph{}
Let $G = (V,E)$ be an undirected graph with edge cost $c: E\rightarrow \R_+$. Consider the problem $(IP)$:
\begin{align*}
\min\ \sum_{e\in E} c(e) x_e&\\
\text{s.t.}\  \sum_{e\in\delta(S)} x_e &\geq f(S) &\forall S \subseteq V \\
x&\geq 0\\
x&\in \Z,
\end{align*}
where $f : S \rightarrow \{0,1\}$ satisfies the maximality property:
$$\text{ for all disjoint } A,B\subseteq V, \text{ we have } f(A \cup B ) \leq \max\{f(A), f(B)\},$$
and $f$ satisfies that there does not exist $S \subseteq V$ and $A, B \subseteq S$ such that
\begin{equation}\label{eq:1}
A\cap B = \emptyset,\ f(S)=f(A)=f(B)=0 \text{ and } f(S\backslash A) = f(S\backslash B) = 1.
\end{equation}
We may assume that $f(V') = 0$, for the every connected component vertex set $V'$, otherwise $(IP)$ is infeasible. Let $(LP)$ denote the linear programming relaxation of $(IP)$ and let $(D)$ denote the dual of $(LP)$:
\begin{align*}
\max\ \sum_{S\subseteq V} f(S) y_S&\\
\text{s.t.}\ \sum_{S : e \in \delta(S)} y_S &\leq c(e) &\forall e \in E \\
y&\geq 0. 
\end{align*}
\paragraph{Algorithm}
Consider the primal-dual algorithm, $\cA$, for this problem:
\begin{enumerate}
\item Start with $x=0$, $y=0$, and $F \rightarrow \emptyset$.
\item While $F$ is not feasible do:
	\begin{enumerate}
	\item Increase $y_S$ uniformly for ``minimal violated cuts" with respect to $F$ until some dual constraint corresponding to an edge $e$ becomes tight.
	\item Set $x_e = 1$, $F = F\cup \{e\}$.
	\end{enumerate}
\item Reverse Deletion: Consider all edges $e$ in $F$ in reverse of the order they were added to $F$ and if $F\backslash \{e\}$ is feasible then set $F = F\backslash\{e\}$ and $x_e = 0$.
\end{enumerate}
\paragraph{}
We intend to show that $\cA$ is a polynomial time $2$-approximation algorithm for $(IP)$. We will need the following lemma from class:
\begin{lemma}\label{lemma:maximality}
Let $f$ be a $01$-function satisfying maximality. Let $F\subseteq E$. Then
\begin{enumerate}
\item $F$ is feasible for $(IP)$ with $f$ if and only if every connected component $C$ of $(V,F)$ satisfies $f(C) = 0$.
\item The minimal violated cuts are connected components $C$ that have $f(C) = 1$.
\end{enumerate}
\end{lemma}
This lemma allows us to show that $\cA$ terminates in polynomial time.
\begin{lemma}\label{lemma:poly1}
Suppose we have a polynomial time oracle for $f$. Then $\cA$ returns a feasible solution for $(IP)$ in polynomial time
\end{lemma}
\begin{proof}
By definition of $\cA$ if $\cA$ terminates then the result is feasible. To see that $\cA$ terminates observe that in every iteration of step $2$ some edge $e \in |E|$ is added to $F$. Since $E$ is feasible, the while loop terminates after $O(|E|)$ iterations. Hence $\cA$ terminates, and returns a feasible solution.
\paragraph{} 
It remains to show that $\cA$ runs in polynomial time. Since we have a polynomial time oracle for $f$ it is easy to see that step $3$ runs in polynomial time. Further we need only maintain $y_S$ for non-zero $y_S$, of which we will show there are a polynomial number. We need to verify step $2(a)$ can be done in polynomial time. By lemma \ref{lemma:maximality} we need only maintain connected components of $F$ since the connected components $C$ with $f(C) = 1$ are precisely the minimal violated cuts. There are a polynomial number of connected components at each iteration (hence a polynomial number of non-zero $y_S$), and since we have a polynomial time oracle for $f$ this implies step $2(a)$ can be done in polynomial time.
\end{proof}
\paragraph{}
Now we need to check the approximation factor. We present the following reduction from class as a lemma:
\begin{lemma}\label{lemma:degree}
Let $w_i$ denote the number of minimally violated sets at iteration $i$.  Let $\cS_i$ denote the set of minimal violated cuts at iteration $i$. If $\sum_{S\in \cS_i} |\delta_F(S)| \leq 2w_i$ for all $i$ then $\cA$ is a $2$-approximation algorithm for $(IP)$ (for any $01$-function $f$).
\end{lemma}
\begin{proof}
It is immediate from our construction of $x$ and $y$ that if $e \in F$ then
$$\sum_{S : e\in\delta(S)} y_S = c(e).$$
Thus we have the objection function value on $F$ is
$$\sum_{e\in F} c(e) = \sum_{e\in F} \sum_{S : e\in\delta(S)} y_S = \sum_{S\subseteq V} |\delta_F(S)|y_S.$$
Let $\Delta_i$ denote the amount by which all minimal violated cuts get increased in iteration $i$. Then
$$\sum_{S\subseteq V} |\delta_F(S)|y_S = \sum_{i} \sum_{S\in \cS_i} |\delta_F(S)|\Delta_i \leq \sum_{i} 2w_i \Delta_i = 2\sum_{S\subseteq V} f(S) y_S.$$
The inequality follows from our hypothesis, and the last equality follows since $f$ is a $01$-function for which non-zero $y_S$ (minimally violated sets) have $f(S) = 1$. 
\end{proof}
\paragraph{}
Now we intend to show that the hypothesis of the previous lemma holds. Consider some iteration $i$ of $\cA$, and let $A$ be the of edges chosen so far at the end of $i$. Let $B = F \cap A$. Construct a graph $H$ by taking the graph $(V,F)$ and contracting the connected components of $(V,B)$ to vertices. Then $E(H) = F\backslash B$. Let $S_v$ denote the set of vertices in $G$ corresponding to vertex $v \in V(H)$. It is not hard to see that $H$ is a forest. Indeed if $H$ has some cycle $C$ then we can remove any edge in $E(C)$ from $F$ without increasing the number of connected components. 

\begin{lemma}\label{lemma:leaves}
$\sum_{v\in W} d_v \leq 2|W|$.
\end{lemma}
\begin{proof}
First we claim that for every connected component $C$ of $H$, at most one leaf of $C$ satisfies $f(S_v) = 0$. Suppose for a contradiction there exists some connected component $C$ of $H$ that has two leaves $v$ and $w$ such that $f(S_v) = f(S_w) = 0$. Let $S_C = \cup_{u\in C} S_u$. Let $e_v, e_w \in F$ be the edges incident to $v$ and $w$ respectively in $H$. By minimality via the reverse deletion step, $F\backslash \{e_v\}$ and $F\backslash \{e_w\}$ are infeasible. But then $f(S_C\backslash S_v) =1 $ and $f(S_C\backslash S_w) = 1$. Since $C$ is a connected component of $H$ we have $f(S_C) = 0$. Hence
$$f(S_v) = f(S_w) = f(S_C) = 0 \quad\text{and}\quad f(S_C\backslash S_v) = f(S_C \backslash S_w) = 1$$
but $S_v, S_w \subseteq S_C$ and $S_v \cap S_w = \emptyset$. This contradictions property \ref{eq:1}.
\end{proof}
\end{document}
