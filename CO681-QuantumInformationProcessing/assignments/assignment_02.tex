\documentclass[letterpaper,12pt,oneside,onecolumn]{article}
\usepackage[margin=1in, bottom=1in, top=1in]{geometry} %1 inch margins
\usepackage{amsmath, amssymb, amstext}
\usepackage{fancyhdr}
\usepackage{mathtools}
\usepackage{algorithm}
\usepackage{algpseudocode}
\usepackage{theorem}
\usepackage{tikz}
\usepackage{tkz-berge}
\usepackage[braket, qm]{qcircuit}

%Macros
\newcommand{\A}{\mathbb{A}} \newcommand{\C}{\mathbb{C}}
\newcommand{\D}{\mathbb{D}} \newcommand{\F}{\mathbb{F}}
\newcommand{\N}{\mathbb{N}} \newcommand{\R}{\mathbb{R}}
\newcommand{\T}{\mathbb{T}} \newcommand{\Z}{\mathbb{Z}}
\newcommand{\Q}{\mathbb{Q}}
 
 
\newcommand{\cA}{\mathcal{A}} \newcommand{\cB}{\mathcal{B}}
\newcommand{\cC}{\mathcal{C}} \newcommand{\cD}{\mathcal{D}}
\newcommand{\cE}{\mathcal{E}} \newcommand{\cF}{\mathcal{F}}
\newcommand{\cG}{\mathcal{G}} \newcommand{\cH}{\mathcal{H}}
\newcommand{\cI}{\mathcal{I}} \newcommand{\cJ}{\mathcal{J}}
\newcommand{\cK}{\mathcal{K}} \newcommand{\cL}{\mathcal{L}}
\newcommand{\cM}{\mathcal{M}} \newcommand{\cN}{\mathcal{N}}
\newcommand{\cO}{\mathcal{O}} \newcommand{\cP}{\mathcal{P}}
\newcommand{\cQ}{\mathcal{Q}} \newcommand{\cR}{\mathcal{R}}
\newcommand{\cS}{\mathcal{S}} \newcommand{\cT}{\mathcal{T}}
\newcommand{\cU}{\mathcal{U}} \newcommand{\cV}{\mathcal{V}}
\newcommand{\cW}{\mathcal{W}} \newcommand{\cX}{\mathcal{X}}
\newcommand{\cY}{\mathcal{Y}} \newcommand{\cZ}{\mathcal{Z}}

\newcommand\numberthis{\addtocounter{equation}{1}\tag{\theequation}}


\newenvironment{proof}{{\bf Proof:  }}{\hfill\rule{2mm}{2mm}}
\newenvironment{proofof}[1]{{\bf Proof of #1:  }}{\hfill\rule{2mm}{2mm}}
\newenvironment{proofofnobox}[1]{{\bf#1:  }}{}\newenvironment{example}{{\bf Example:  }}{\hfill\rule{2mm}{2mm}}

%\renewcommand{\thesection}{\lecnum.\arabic{section}}
%\renewcommand{\theequation}{\thesection.\arabic{equation}}
%\renewcommand{\thefigure}{\thesection.\arabic{figure}}

\newtheorem{fact}{Fact}[section]
\newtheorem{lemma}[fact]{Lemma}
\newtheorem{theorem}[fact]{Theorem}
\newtheorem{definition}[fact]{Definition}
\newtheorem{corollary}[fact]{Corollary}
\newtheorem{proposition}[fact]{Proposition}
\newtheorem{claim}[fact]{Claim}
\newtheorem{exercise}[fact]{Exercise}
\newtheorem{note}[fact]{Note}
\newtheorem{conjecture}[fact]{Conjecture}

\newcommand{\size}[1]{\ensuremath{\left|#1\right|}}
\newcommand{\ceil}[1]{\ensuremath{\left\lceil#1\right\rceil}}
\newcommand{\floor}[1]{\ensuremath{\left\lfloor#1\right\rfloor}}

\DeclarePairedDelimiter\abs{\lvert}{\rvert}%
\DeclarePairedDelimiter\norm{\lVert}{\rVert}%
%END MACROS
%Page style
\pagestyle{fancy}

\listfiles

\raggedbottom

\lhead{\today}
\rhead{W. Justin Toth CO681-Quantum Information Processing A2} %CHANGE n to ASSIGNMENT NUMBER ijk TO COURSE CODE
\renewcommand{\headrulewidth}{1pt} %heading underlined
%\renewcommand{\baselinestretch}{1.2} % 1.2 line spacing for legibility (optional)

\begin{document}
\section{}
\subsection{a}
\paragraph{}
$$11+12 \mod 13 = 10$$
\subsection{b}
\paragraph{}
$$\Z^\times_{13} = \{2^0=1, 2^1 = 2, 2^2 = 4, 2^3 = 8, 2^4 = 3, 2^5 = 6, 2^6=12, 2^7=11, 2^8 = 9, 2^9=5, 2^{10}=10, 2^{11} = 7\}$$
\subsection{c}
\paragraph{}
$2^8\equiv 9 \mod 13$ so $\log_2(9)$ is $8$ in $\Z^\times_{13}$.
\subsection{d}
\paragraph{}
\begin{itemize}
\item $3^3 \equiv 1 \mod 13$ so $3$ is not a generator
\item $4^6 \equiv 1 \mod 13$ so $4$ is not a generator
\item $5^4 \equiv 1 \mod 13$ so $5$ is not a generator
\item The order of $6$ is $12$ in $\Z_{13}^\times$ so is $6$ is a generator of $Z_{13}^\times$
\item The order of $7$ is $12$ in $\Z_{13}^\times$ so is $7$ is a generator of $Z_{13}^\times$
\item $8^4 \equiv 1 \mod 13$ so $8$ is not a generator
\item $9^3 \equiv 1 \mod 13$ so $9$ is not a generator
\item $10^6 \equiv 1 \mod 13$ so $10$ is not a generator
\item The order of $11$ is $12$ in $\Z_{13}^\times$ so is $11$ is a generator of $Z_{13}^\times$
\item $12 \equiv 1 \mod 13$ so $12$ is not a generator
\end{itemize}
Hence the generators of $\Z_{13}^\times$ besides $2$ are: $6,7,$ and $11$.

\section{}
\subsection{a}
\paragraph{}
We check the action of $H$ on $X$ and $Z$.
$$HXH^{\dagger} = \frac{1}{2} \begin{bmatrix} 1 & 1 \\ 1 &  -1 \end{bmatrix} \begin{bmatrix} 0 & 1 \\ 1 & 0 \end{bmatrix} \begin{bmatrix} 1 & 1 \\ 1 &  -1 \end{bmatrix} = \frac{1}{2} \begin{bmatrix} 1 & 1 \\ 1 &  -1 \end{bmatrix} \begin{bmatrix}1 & -1 \\ 1 & 1 \end{bmatrix} =\frac{1}{2} \begin{bmatrix} 2 & 0 \\ 0 & 2 \end{bmatrix} = I. $$
$$HZH^{\dagger} = \frac{1}{2} \begin{bmatrix} 1 & 1 \\ 1 &  -1 \end{bmatrix} \begin{bmatrix} 1 & 0 \\ 0 & -1 \end{bmatrix} \begin{bmatrix} 1 & 1 \\ 1 &  -1 \end{bmatrix} = \frac{1}{2} \begin{bmatrix} 1 & 1 \\ 1 &  -1 \end{bmatrix} \begin{bmatrix} 1 & 1 \\ -1 & 1\end{bmatrix} = \frac{1}{2} \begin{bmatrix} 0 & 2 \\ 2 & 0\end{bmatrix} = X.$$
Thus $H$ is in the Clifford group on $1$ qubit.
\subsection{b}
\paragraph{}
Similar to the previous subsection we consider the action of $T$ on $X$ and $Z$.
$$TXT^{\dagger} = \begin{bmatrix} 1 & 0 \\ 0& \zeta_8 \end{bmatrix}\begin{bmatrix} 0 & 1 \\ 1 & 0 \end{bmatrix}\begin{bmatrix} 1 & 0 \\ 0& \zeta_8^* \end{bmatrix} = \begin{bmatrix} 1 & 0 \\ 0& \zeta_8 \end{bmatrix}\begin{bmatrix} 0 & \zeta^*_8 \\ 1 & 0 \end{bmatrix} = \begin{bmatrix} 0 & \zeta_8^* \\ \zeta_8 & 0 \end{bmatrix}.$$
Such a matrix cannot be written as  $\alpha V$ for phase $\alpha$ and $V \in \{I,X,Y,Z\}$. Thus $T$ is not in Clifford group on $1$ qubit.
\subsection{c}
\paragraph{}
$$SXS^\dagger = \begin{bmatrix} 1 & 0 \\ 0 & i \end{bmatrix} \begin{bmatrix} 0 & 1 \\ 1 & 0 \end{bmatrix} \begin{bmatrix} 1 & 0\\ 0 & -i \end{bmatrix} = \begin{bmatrix} 1 & 0 \\ 0 & i \end{bmatrix} \begin{bmatrix} 0 & -i\\ 1 & 0 \end{bmatrix} =\begin{bmatrix} 0 & -i \\ i & 0 \end{bmatrix} = Y.$$
$$SZS^\dagger = \begin{bmatrix} 1 & 0 \\ 0 & i \end{bmatrix}\begin{bmatrix} 1 & 0 \\ 0 & -1\end{bmatrix} \begin{bmatrix} 1 & 0 \\0 & -i \end{bmatrix} = \begin{bmatrix} 1 & 0 \\ 0 & i \end{bmatrix}\begin{bmatrix} 1 & 0 \\ 0 & i \end{bmatrix} = Z.$$
Thus $S$ is in the Clifford group on $1$ qubit.
\subsection{d}
\paragraph{}
Let $C$ be the unitary matrix representing the controlled-not gate. We consider the action of $C$ on $I \otimes X$, $X\otimes I$, $I \otimes Z$. and $Z \otimes I$.
$$C (I\otimes X) C^\dagger = \begin{bmatrix} 1 & 0 & 0 & 0 \\ 0 & 1 & 0 & 0 \\ 0 & 0 & 0 &1 \\ 0& 0& 1 & 0 \end{bmatrix}\begin{bmatrix}0 & 1 & 0 & 0 \\ 1 & 0 & 0 &0 \\ 0 & 0 &0 & 1 \\ 0 & 0 & 1 & 0 \end{bmatrix} \begin{bmatrix} 1 & 0 & 0 & 0 \\ 0 & 1 & 0 & 0 \\ 0 & 0 & 0 &1 \\ 0& 0& 1 & 0 \end{bmatrix} = \begin{bmatrix} 0 & 1& 0& 0\\ 1 & 0  & 0& 0\\ 0 & 0 & 1 & 0\\ 0 & 0 & 0 & 1\end{bmatrix} \begin{bmatrix} 1 & 0 & 0 & 0 \\ 0 & 1 & 0 & 0 \\ 0 & 0 & 0 &1 \\ 0& 0& 1 & 0 \end{bmatrix} = I \otimes X.$$
$$ C (X \otimes I) C^\dagger =  \begin{bmatrix} 1 & 0 & 0 & 0 \\ 0 & 1 & 0 & 0 \\ 0 & 0 & 0 &1 \\ 0& 0& 1 & 0 \end{bmatrix} \begin{bmatrix}  0 & 0 & 1 & 0 \\ 0 & 0 & 0 & 1 \\ 1 & 0 & 0 & 0 \\ 0 &1 & 0 &0\end{bmatrix} \begin{bmatrix} 1 & 0 & 0 & 0 \\ 0 & 1 & 0 & 0 \\ 0 & 0 & 0 &1 \\ 0& 0& 1 & 0 \end{bmatrix}  =  \begin{bmatrix} 0 & 0 &1 & 0 \\ 0 & 0 & 0 & 1 \\ 0 & 1 & 0 & 0 \\ 1 & 0 & 0 &0\end{bmatrix} \begin{bmatrix} 1 & 0 & 0 & 0 \\ 0 & 1 & 0 & 0 \\ 0 & 0 & 0 &1 \\ 0& 0& 1 & 0 \end{bmatrix} = X \otimes X$$
\begin{align*}C (I \otimes Z) C^\dagger &=  \begin{bmatrix} 1 & 0 & 0 & 0 \\ 0 & 1 & 0 & 0 \\ 0 & 0 & 0 &1 \\ 0& 0& 1 & 0 \end{bmatrix}\begin{bmatrix}1 &0 &0 &0 \\ 0 & -1 & 0 &0 \\ 0 & 0 &1 & 0\\ 0& 0& 0 & -1 \end{bmatrix} \begin{bmatrix} 1 & 0 & 0 & 0 \\ 0 & 1 & 0 & 0 \\ 0 & 0 & 0 &1 \\ 0& 0& 1 & 0 \end{bmatrix}\\
 &=  \begin{bmatrix} 1 & 0 & 0 &0 \\ 0 & -1 & 0 &0 \\ 0 & 0 & 0 & -1 \\ 0 &0 &1 & 0\end{bmatrix}\begin{bmatrix} 1 & 0 & 0 & 0 \\ 0 & 1 & 0 & 0 \\ 0 & 0 & 0 &1 \\ 0& 0& 1 & 0 \end{bmatrix}\\ &= \begin{bmatrix} 1 & 0 &0 &0 \\ 0 & -1 &0 &0 \\ 0 &0 &-1 & 0\\ 0 &0 &0 &1\end{bmatrix} = Z \otimes Z.\end{align*}
 \begin{align*}
 C(Z\otimes I)C^\dagger &=  \begin{bmatrix} 1 & 0 & 0 & 0 \\ 0 & 1 & 0 & 0 \\ 0 & 0 & 0 &1 \\ 0& 0& 1 & 0 \end{bmatrix} \begin{bmatrix}1 &0 &0 &0 \\0 & 1 & 0 &0 \\ 0 &0 &-1 & 0\\ 0 &0 &0 & -1 \end{bmatrix}  \begin{bmatrix} 1 & 0 & 0 & 0 \\ 0 & 1 & 0 & 0 \\ 0 & 0 & 0 &1 \\ 0& 0& 1 & 0 \end{bmatrix}\\
 &= \begin{bmatrix} 1& 0 & 0 &0 \\0 &1 & 0 &0 \\ 0 & 0 &0 &-1 \\ 0 & 0 & -1 &0\end{bmatrix} \begin{bmatrix} 1 & 0 & 0 & 0 \\ 0 & 1 & 0 & 0 \\ 0 & 0 & 0 &1 \\ 0& 0& 1 & 0 \end{bmatrix}\\
 &= \begin{bmatrix} 1 & 0 &0 &0 \\ 0 &1 & 0 &0\\ 0 & 0 &-1 & 0 \\ 0 & 0 & 0 & -1 \end{bmatrix} = Z\otimes I.
 \end{align*}
 Thus $C$ is in the $2$-qubit Clifford group.
 \subsection{e}
 Let $A$ be the Toffoli gate (CCNOT). For the sake of space we'll abbreviate the calculations. We have that
 $$A(I\otimes X\otimes I) A^\dagger = \begin{bmatrix} X\otimes I & 0 \\ 0 & Z\otimes Z\end{bmatrix}.$$
 This can't be written as a tensor product $\alpha V_1\otimes V_2 \otimes V_3$ with $\alpha$ scalar and $V_1,V_2, V_3 \in \{I,X,Y,Z\}$. To see this observe that if we take $V_1 \in  \{X, Y\}$ then the first $4x4$ diagonal block of $V_1\otimes V_2 \otimes V_3$ will be $0$, contradicting that it should be $X \otimes I$, and if we take $V_1 \in \{I,Z\}$ then the first $4x4$ diagonal block will be $\pm1$ times the second, but $X\otimes I \neq \pm Z\otimes Z$. Therefore no such tensor product expression for $A(I\otimes X \otimes I)A^\dagger$ is possible, and hence the Toffoli gate is not in the $3$-qubit Clifford group. $\blacksquare$
 \section{}
 \subsection{a}
 \paragraph{}
We use the same notation for gates as in question $2$. $X_4$ is given by the circuit
 \[ \Qcircuit @C=1em @R=1em {
 & \targ &   \qw & \qw\\
 & \ctrl{-1} & \gate{X} & \qw
}\]
$Z_4$ is given by the circuit
\[ \Qcircuit @C=1em @R=1em {
 & \gate{S} &  \gate{S} & \qw\\
 & \gate{S} & \qw & \qw
}\].
\subsection{b}
\paragraph{}
$X_8$ is given by the circuit (using one Toffoli gate)
\[ \Qcircuit @C=1em @R=1em {
 & \targ &  \qw & \qw & \qw & \qw\\
 & \ctrl{-1} & \targ& \qw  & \qw &\qw\\
 & \ctrl{-1} & \ctrl{-1} & \gate{X} &\qw & \qw
}\]
Bonus: Figure $4.9$ of Nielsen and Chuang gives a controlled single-qubit implementation of the Toffoli gate. Since my use of the Toffoli gate has different controls I reflect the picture, giving the following implementation of the above Toffoli gate where the first qubit is controlled by the second and third:

\[ \Qcircuit @C=1em @R=1em {
 & \gate{H} & \targ & \gate{T^\dagger} & \targ & \gate{T} & \targ & \gate{T^\dagger}& \targ & \gate{T} & \gate{H} & \qw & \qw & \qw \\
 & \qw & \ctrl{-1}& \qw  & \qw & \qw & \ctrl{-1} & \qw & \qw & \gate{T^\dagger} & \targ & \gate{T^\dagger} & \targ & \gate{S} &\qw\\
 & \qw & \qw & \qw & \ctrl{-2} & \qw & \qw & \qw & \ctrl{-2} & \qw & \ctrl{-1} & \qw & \ctrl{-1} & \gate{T} & \qw
}\]
$Z_8$ is given by the circuit
\[ \Qcircuit @C=1em @R=1em {
 & \gate{T} &  \gate{T} & \gate{T} & \gate{T} & \qw\\
 & \gate{T} & \gate{T}& \qw  & \qw &\qw\\
 & \gate{T} & \qw & \qw &\qw & \qw
}\]
\end{document}
