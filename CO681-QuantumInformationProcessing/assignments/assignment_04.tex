\documentclass[letterpaper,12pt,oneside,onecolumn]{article}
\usepackage[margin=1in, bottom=1in, top=1in]{geometry} %1 inch margins
\usepackage{amsmath, amssymb, amstext}
\usepackage{fancyhdr}
\usepackage{mathtools}
\usepackage{algorithm}
\usepackage{algpseudocode}
\usepackage{theorem}
\usepackage{tikz}
\usepackage{tkz-berge}
\usepackage[braket, qm]{qcircuit}

%Macros
\newcommand{\A}{\mathbb{A}} \newcommand{\C}{\mathbb{C}}
\newcommand{\D}{\mathbb{D}} \newcommand{\F}{\mathbb{F}}
\newcommand{\N}{\mathbb{N}} \newcommand{\R}{\mathbb{R}}
\newcommand{\T}{\mathbb{T}} \newcommand{\Z}{\mathbb{Z}}
\newcommand{\Q}{\mathbb{Q}}
 
 
\newcommand{\cA}{\mathcal{A}} \newcommand{\cB}{\mathcal{B}}
\newcommand{\cC}{\mathcal{C}} \newcommand{\cD}{\mathcal{D}}
\newcommand{\cE}{\mathcal{E}} \newcommand{\cF}{\mathcal{F}}
\newcommand{\cG}{\mathcal{G}} \newcommand{\cH}{\mathcal{H}}
\newcommand{\cI}{\mathcal{I}} \newcommand{\cJ}{\mathcal{J}}
\newcommand{\cK}{\mathcal{K}} \newcommand{\cL}{\mathcal{L}}
\newcommand{\cM}{\mathcal{M}} \newcommand{\cN}{\mathcal{N}}
\newcommand{\cO}{\mathcal{O}} \newcommand{\cP}{\mathcal{P}}
\newcommand{\cQ}{\mathcal{Q}} \newcommand{\cR}{\mathcal{R}}
\newcommand{\cS}{\mathcal{S}} \newcommand{\cT}{\mathcal{T}}
\newcommand{\cU}{\mathcal{U}} \newcommand{\cV}{\mathcal{V}}
\newcommand{\cW}{\mathcal{W}} \newcommand{\cX}{\mathcal{X}}
\newcommand{\cY}{\mathcal{Y}} \newcommand{\cZ}{\mathcal{Z}}

\newcommand\numberthis{\addtocounter{equation}{1}\tag{\theequation}}


\newenvironment{proof}{{\bf Proof:  }}{\hfill\rule{2mm}{2mm}}
\newenvironment{proofof}[1]{{\bf Proof of #1:  }}{\hfill\rule{2mm}{2mm}}
\newenvironment{proofofnobox}[1]{{\bf#1:  }}{}\newenvironment{example}{{\bf Example:  }}{\hfill\rule{2mm}{2mm}}

%\renewcommand{\thesection}{\lecnum.\arabic{section}}
%\renewcommand{\theequation}{\thesection.\arabic{equation}}
%\renewcommand{\thefigure}{\thesection.\arabic{figure}}

\newtheorem{fact}{Fact}[section]
\newtheorem{lemma}[fact]{Lemma}
\newtheorem{theorem}[fact]{Theorem}
\newtheorem{definition}[fact]{Definition}
\newtheorem{corollary}[fact]{Corollary}
\newtheorem{proposition}[fact]{Proposition}
\newtheorem{claim}[fact]{Claim}
\newtheorem{exercise}[fact]{Exercise}
\newtheorem{note}[fact]{Note}
\newtheorem{conjecture}[fact]{Conjecture}

\newcommand{\size}[1]{\ensuremath{\left|#1\right|}}
\newcommand{\ceil}[1]{\ensuremath{\left\lceil#1\right\rceil}}
\newcommand{\floor}[1]{\ensuremath{\left\lfloor#1\right\rfloor}}

\DeclarePairedDelimiter\abs{\lvert}{\rvert}%
\DeclarePairedDelimiter\norm{\lVert}{\rVert}%
%END MACROS
%Page style
\pagestyle{fancy}

\listfiles

\raggedbottom

\lhead{\today}
\rhead{W. Justin Toth CO681-Quantum Information Processing A4} %CHANGE n to ASSIGNMENT NUMBER ijk TO COURSE CODE
\renewcommand{\headrulewidth}{1pt} %heading underlined
%\renewcommand{\baselinestretch}{1.2} % 1.2 line spacing for legibility (optional)

\begin{document}
\section{}
\subsection{a}
Observe that 
$$A:= r_0I + r_xX + r_yY + r_zZ = \begin{bmatrix}
r_0 + r_z & r_x - ir_y \\ r_x + ir_y & r_o -r_z
\end{bmatrix}.$$
We consider the characteristic polynomial
$$\det(\lambda I - A) = (\lambda - (r_o+r_z))(\lambda - (r_0 -r_z)) -(r_x+ir_y)(r_x-ir_y) = \lambda^2 - 2r_0\lambda +r_0^2 - ||r||_2^2.$$
Substituting $\lambda = r_0 + ||r||_2$ we obtain
$$(r_0 + ||r||_2)^2 - 2r_0(r_0 + ||r||_2) + r_0^2 - ||r||_2^2 = r_0^2 + 2r_0||r||_2 + ||r||_2^2 - 2r_o^2 -2r_0||r||_2 +r_0^2 - ||r||_2^2 = 0,$$
and similarly if we substitute $\lambda=r_0 - ||r||_2$ we obtain
$$(r_0 - ||r||_2)^2 - 2r_0(r_0 -||r||_2) + r_0^2 - ||r||_2^2 = r_0^2 - 2r_0||r||_2 + ||r||_2^2 - 2r_o^2 +2r_0||r||_2 +r_0^2 - ||r||_2^2 = 0.$$
Hence the roots of the characteristic polynomial are $\lambda = r_0 \pm ||r||_2$ and hence the eigenvalues of $A$ are  $r_0 \pm ||r||_2$.
\subsection{b}
Let $\rho = \frac{1}{2}(I + r_xX + r_yY + r_zZ)$ and $\sigma =  \frac{1}{2}(I + r_x'X + r_y'Y + r_z'Z)$. Then
$$\rho-\sigma =\frac{1}{2}( (r_x-r'_x)X + (r_y-r_y')Y + (r_z-r_z')Z).$$
So from part $1(a)$ we know the eigenvalues of $\rho-\sigma$ are $\pm \frac{1}{2}||r - r'||_2.$
Using that the trace norm is the sum of the eigenvalues in absolute value we see that
$$||\rho-\sigma||_1 = \frac{1}{2}||r-r'||_2 + \frac{1}{2}||r-r'||_2 = ||r-r'||_2$$
as desired.
\section{}
\subsection{a}
\paragraph{}
Via the Gram-Schmidt there exists a pair of orthonormal vectors $\ket{a}$ and $\ket{b}$ so that $\ket{\psi}=\ket{a}$ and $\ket{b}$ is the normalization of $\ket{\phi} - \ip{\psi}{\phi}\ket{\psi}$. Since $\ket{\phi}$ lies in the hyperplane spanned by $\ket{a} and \ket{b}$, there exists an angle $\theta$ such that 
$$\ket{\phi} = \cos \theta \ket{a} + \sin \theta\ket{b}.$$
Now we have that
$$(\ket{\psi}\bra{\psi} - \bra{\phi}\ket{\phi})^2 = (1-\cos^2\theta)\ket{a}\bra{a} - \cos\theta\sin\theta(\ket{a}\bra{b} + \ket{b}\bra{a}) -\sin^2\theta\ket{b}\bra{b}$$
So then
$$||\ket{\psi}\bra{\psi} - \bra{\phi}\ket{\phi}||_1 = Tr\abs{\begin{bmatrix} 1 - \cos^2\theta & -\cos\theta\sin\theta \\ -\cos\theta\sin\theta & -\sin^2\theta\end{bmatrix}} = 2|\sin\theta| = 2\sqrt{1-\cos^2\theta}.$$
Since $\ip{\psi}{\phi} = \cos\theta$ this show the desired equality.
\subsection{b}
\paragraph{}
The eigenvalues of $\ket{\psi}\bra{\psi} - \frac{1}{d}I$ are $1-\frac{1}{d}$ with multiplicity $1$ and $-\frac{1}{d}$ with multiplicity $d-1$.
So we have $$||\ket{\psi}\bra{\psi} - \frac{1}{d}I ||_1 = |1-\frac{1}{d}| + \sum_{i=1}^{d-1} |-\frac{1}{d}| =1 - \frac{d-2}{d}.$$
For fidelity we have
$$F(\ket{\psi}\bra{\psi}, \frac{1}{d}I) = Tr\sqrt{(\ket{\psi}\bra{\psi})^\frac{1}{2}\frac{1}{d}I (\ket{\psi}\bra{\psi})^\frac{1}{2} } = \sqrt{\frac{1}{d}}Tr\sqrt{(\ket{\psi}\bra{\psi})^\frac{1}{2}(\ket{\psi}\bra{\psi})^\frac{1}{2}} = \sqrt{\frac{1}{d}\ip{\psi}{\psi}} = \sqrt{\frac{1}{d}}.$$
By the Holevo-Helstrom Theorem the optimal distinguishing probability between $\ket{\psi}\bra{\psi}$ and $\frac{1}{d}I$ given each with equal probability is
$$\frac{1}{2} + \frac{1}{4}||\ket{\psi}\bra{\psi} - \frac{1}{d}I ||_1 = \frac{1}{2} + \frac{1}{4}(1 - \frac{d-2}{d}).$$
\section{}

\end{document}
