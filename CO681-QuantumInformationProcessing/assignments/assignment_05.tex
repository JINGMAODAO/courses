\documentclass[letterpaper,12pt,oneside,onecolumn]{article}
\usepackage[margin=1in, bottom=1in, top=1in]{geometry} %1 inch margins
\usepackage{amsmath, amssymb, amstext}
\usepackage{fancyhdr}
\usepackage{mathtools}
\usepackage{algorithm}
\usepackage{algpseudocode}
\usepackage{theorem}
\usepackage{tikz}
\usepackage{tkz-berge}
\usepackage[braket, qm]{qcircuit}

%Macros
\newcommand{\A}{\mathbb{A}} \newcommand{\C}{\mathbb{C}}
\newcommand{\D}{\mathbb{D}} \newcommand{\F}{\mathbb{F}}
\newcommand{\N}{\mathbb{N}} \newcommand{\R}{\mathbb{R}}
\newcommand{\T}{\mathbb{T}} \newcommand{\Z}{\mathbb{Z}}
\newcommand{\Q}{\mathbb{Q}}
 
 
\newcommand{\cA}{\mathcal{A}} \newcommand{\cB}{\mathcal{B}}
\newcommand{\cC}{\mathcal{C}} \newcommand{\cD}{\mathcal{D}}
\newcommand{\cE}{\mathcal{E}} \newcommand{\cF}{\mathcal{F}}
\newcommand{\cG}{\mathcal{G}} \newcommand{\cH}{\mathcal{H}}
\newcommand{\cI}{\mathcal{I}} \newcommand{\cJ}{\mathcal{J}}
\newcommand{\cK}{\mathcal{K}} \newcommand{\cL}{\mathcal{L}}
\newcommand{\cM}{\mathcal{M}} \newcommand{\cN}{\mathcal{N}}
\newcommand{\cO}{\mathcal{O}} \newcommand{\cP}{\mathcal{P}}
\newcommand{\cQ}{\mathcal{Q}} \newcommand{\cR}{\mathcal{R}}
\newcommand{\cS}{\mathcal{S}} \newcommand{\cT}{\mathcal{T}}
\newcommand{\cU}{\mathcal{U}} \newcommand{\cV}{\mathcal{V}}
\newcommand{\cW}{\mathcal{W}} \newcommand{\cX}{\mathcal{X}}
\newcommand{\cY}{\mathcal{Y}} \newcommand{\cZ}{\mathcal{Z}}

\newcommand\numberthis{\addtocounter{equation}{1}\tag{\theequation}}


\newenvironment{proof}{{\bf Proof:  }}{\hfill\rule{2mm}{2mm}}
\newenvironment{proofof}[1]{{\bf Proof of #1:  }}{\hfill\rule{2mm}{2mm}}
\newenvironment{proofofnobox}[1]{{\bf#1:  }}{}\newenvironment{example}{{\bf Example:  }}{\hfill\rule{2mm}{2mm}}

%\renewcommand{\thesection}{\lecnum.\arabic{section}}
%\renewcommand{\theequation}{\thesection.\arabic{equation}}
%\renewcommand{\thefigure}{\thesection.\arabic{figure}}

\newtheorem{fact}{Fact}[section]
\newtheorem{lemma}[fact]{Lemma}
\newtheorem{theorem}[fact]{Theorem}
\newtheorem{definition}[fact]{Definition}
\newtheorem{corollary}[fact]{Corollary}
\newtheorem{proposition}[fact]{Proposition}
\newtheorem{claim}[fact]{Claim}
\newtheorem{exercise}[fact]{Exercise}
\newtheorem{note}[fact]{Note}
\newtheorem{conjecture}[fact]{Conjecture}

\newcommand{\size}[1]{\ensuremath{\left|#1\right|}}
\newcommand{\ceil}[1]{\ensuremath{\left\lceil#1\right\rceil}}
\newcommand{\floor}[1]{\ensuremath{\left\lfloor#1\right\rfloor}}

\DeclarePairedDelimiter\abs{\lvert}{\rvert}%
\DeclarePairedDelimiter\norm{\lVert}{\rVert}%
%END MACROS
%Page style
\pagestyle{fancy}

\listfiles

\raggedbottom

\lhead{\today}
\rhead{W. Justin Toth CO681-Quantum Information Processing A5} %CHANGE n to ASSIGNMENT NUMBER ijk TO COURSE CODE
\renewcommand{\headrulewidth}{1pt} %heading underlined
%\renewcommand{\baselinestretch}{1.2} % 1.2 line spacing for legibility (optional)

\begin{document}
\section{}
\subsection{a}
\paragraph{}
We have (leaving off normalization)
\begin{align*}
\ket{\phi^+} &= \ket{00} + \ket{11}\\
\ket{\phi^-} &= \ket{00} - \ket{11} \\
\ket{\psi^+} &= \ket{10} + \ket{01} \\
\ket{\psi^-} &= \ket{10} - \ket{01}
\end{align*}
So then
$$\ket{\phi^+}\bra{\phi^+} + \ket{\phi^-}\bra{\phi^-} + \ket{\psi^+}\bra{\psi^-} = \frac{1}{2}\begin{bmatrix}
2 & 0 & 0 &0 \\
0 & 1 & 1 & 0 \\
0 & 1 & 1 & 0 \\
0 &0 &0 &2
\end{bmatrix}$$
and 
$$\ket{\psi^-}\bra{\psi^-} =\frac{1}{2} \begin{bmatrix}
0 & 0 &0 & 0\\
0 & 1 & -1 & 0 \\
0 & -1 & 1 & 0 \\
0 &0 &0 &0
\end{bmatrix}$$
Hence our Werner state $\rho$ as a $4\times 4$ matrix is
$$\rho = \frac{1}{6}\begin{bmatrix}
2(1-p) & 0 & 0 &0 \\
0 & 1+2p & 1-4p & 0 \\
0 & 1-4p & 1+2p & 0 \\
0 & 0 & 0 & 2(1-p)
\end{bmatrix}$$
The partial transpose of $\rho$ is 
$$(I \otimes T)\rho = \frac{1}{6}\begin{bmatrix}
2(1-p) & 0 & 0 &1-4p \\
0 & 1+2p3 & 0 & 0 \\
0 & 0 & 1+2p & 0 \\
1-4p & 0 & 0 & 1-3p
\end{bmatrix}$$
\subsection{b}
The eigenvalues $(I \otimes T)\rho$ are $\frac{1+2p}{6}$, $0$, $0$, and $\frac{3-6p}{6} = \frac{1-2p}{2}$. The least of which is $\frac{1-2p}{2}$. From our slides we know that $\rho$ is separable if and only if $(I \otimes T)\rho$ is positive semidefinite. This happens if and only if $\frac{1-2p}{2} \geq 0$, that is when $p \leq 1/2$.
\end{document}
