\documentclass[letterpaper,12pt,oneside,onecolumn]{article}
\usepackage[margin=1in, bottom=1in, top=1in]{geometry} %1 inch margins
\usepackage{amsmath, amssymb, amstext}
\usepackage{fancyhdr}
\usepackage{mathtools}
\usepackage{algorithm}
\usepackage{algpseudocode}
\usepackage{theorem}
\usepackage{tikz}
\usepackage{tkz-berge}
\usepackage[braket, qm]{qcircuit}

%Macros
\newcommand{\A}{\mathbb{A}} \newcommand{\C}{\mathbb{C}}
\newcommand{\D}{\mathbb{D}} \newcommand{\F}{\mathbb{F}}
\newcommand{\N}{\mathbb{N}} \newcommand{\R}{\mathbb{R}}
\newcommand{\T}{\mathbb{T}} \newcommand{\Z}{\mathbb{Z}}
\newcommand{\Q}{\mathbb{Q}}
 
 
\newcommand{\cA}{\mathcal{A}} \newcommand{\cB}{\mathcal{B}}
\newcommand{\cC}{\mathcal{C}} \newcommand{\cD}{\mathcal{D}}
\newcommand{\cE}{\mathcal{E}} \newcommand{\cF}{\mathcal{F}}
\newcommand{\cG}{\mathcal{G}} \newcommand{\cH}{\mathcal{H}}
\newcommand{\cI}{\mathcal{I}} \newcommand{\cJ}{\mathcal{J}}
\newcommand{\cK}{\mathcal{K}} \newcommand{\cL}{\mathcal{L}}
\newcommand{\cM}{\mathcal{M}} \newcommand{\cN}{\mathcal{N}}
\newcommand{\cO}{\mathcal{O}} \newcommand{\cP}{\mathcal{P}}
\newcommand{\cQ}{\mathcal{Q}} \newcommand{\cR}{\mathcal{R}}
\newcommand{\cS}{\mathcal{S}} \newcommand{\cT}{\mathcal{T}}
\newcommand{\cU}{\mathcal{U}} \newcommand{\cV}{\mathcal{V}}
\newcommand{\cW}{\mathcal{W}} \newcommand{\cX}{\mathcal{X}}
\newcommand{\cY}{\mathcal{Y}} \newcommand{\cZ}{\mathcal{Z}}

\newcommand\numberthis{\addtocounter{equation}{1}\tag{\theequation}}


\newenvironment{proof}{{\bf Proof:  }}{\hfill\rule{2mm}{2mm}}
\newenvironment{proofof}[1]{{\bf Proof of #1:  }}{\hfill\rule{2mm}{2mm}}
\newenvironment{proofofnobox}[1]{{\bf#1:  }}{}\newenvironment{example}{{\bf Example:  }}{\hfill\rule{2mm}{2mm}}

%\renewcommand{\thesection}{\lecnum.\arabic{section}}
%\renewcommand{\theequation}{\thesection.\arabic{equation}}
%\renewcommand{\thefigure}{\thesection.\arabic{figure}}

\newtheorem{fact}{Fact}[section]
\newtheorem{lemma}[fact]{Lemma}
\newtheorem{theorem}[fact]{Theorem}
\newtheorem{definition}[fact]{Definition}
\newtheorem{corollary}[fact]{Corollary}
\newtheorem{proposition}[fact]{Proposition}
\newtheorem{claim}[fact]{Claim}
\newtheorem{exercise}[fact]{Exercise}
\newtheorem{note}[fact]{Note}
\newtheorem{conjecture}[fact]{Conjecture}

\newcommand{\size}[1]{\ensuremath{\left|#1\right|}}
\newcommand{\ceil}[1]{\ensuremath{\left\lceil#1\right\rceil}}
\newcommand{\floor}[1]{\ensuremath{\left\lfloor#1\right\rfloor}}

\DeclarePairedDelimiter\abs{\lvert}{\rvert}%
\DeclarePairedDelimiter\norm{\lVert}{\rVert}%
%END MACROS
%Page style
\pagestyle{fancy}

\listfiles

\raggedbottom

\lhead{\today}
\rhead{W. Justin Toth CO681-Quantum Information Processing A3} %CHANGE n to ASSIGNMENT NUMBER ijk TO COURSE CODE
\renewcommand{\headrulewidth}{1pt} %heading underlined
%\renewcommand{\baselinestretch}{1.2} % 1.2 line spacing for legibility (optional)

\begin{document}
\section{}
\subsection{a}
\paragraph{}
The phase estimation algorithm can be thought of as running a unitary matrix $W$ which maps $\ket{0}\ket{\phi}$ to $\ket{a}\ket{\phi}$ for any eigenvector $\ket{\phi}$ of $U$ with eigenvalue $\omega^a$, and then measuring the first register. So if we consider $\alpha\ket{\psi} + \beta\ket{\phi}$ then running phase estimation yields
$$W(\ket{0}(\alpha\ket{\psi} + \beta\ket{\phi})) = \alpha W(\ket{0}\ket{\psi}) + \beta W(\ket{0}\ket{\phi}) = \alpha \ket{a}\ket{\psi} + \beta\ket{b}\ket{\phi}$$
before measuring. So measuring the first register yields $\ket{a}$ with probability $|\alpha|^2$ and $b$ with probability $|\beta|^2$.
\subsection{b}
\paragraph{}

\section{}
\subsection{a}
\paragraph{}
The probability of getting outcome $j$ is
$$Tr(A_j\ket{\psi_j}\bra{\psi_j}A_j^\dagger) = \frac{1}{2} Tr(\ket{\psi_j}\bra{\psi_j}\ket{\psi_j}\bra{\psi_j}\ket{\psi_j}\bra{\psi_j}) = \frac{1}{2}Tr(\ip{\psi}{\psi}^3) = \frac{1}{2}.$$
\subsection{b}
\paragraph{}
Let $j \in \{0,1,2,3\}$. The density matrix resulting from measuring $\ket{\psi_j}$ is
$$\frac{1}{2}\ket{\psi_j}\bra{\psi_j} + \frac{1}{2}\ip{\psi_j}{\psi_{j+1}}^2\ket{\psi_{j+1}}\bra{\psi_{j+1}} + \frac{1}{2}\ip{\psi_j}{\psi_{j+3}}^2\ket{\psi_{j+3}}\bra{\psi_{j+3}}$$
where addition in the subscripts is taken modulo $4$. This simplifies as
$$\frac{1}{2}\ket{\psi_j}\bra{\psi_j} + \frac{1}{4}\ket{\psi_{j+1}}\bra{\psi_{j+1}} + \frac{1}{4}\ket{\psi_{j+3}}\bra{\psi_{j+3}}.$$
\end{document}
