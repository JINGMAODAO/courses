\section{Network Flows}\label{sec:nf}

\subsection{Directed Graph Layering}
\paragraph{}
An important subproblem we will use in solving Network Flow problems, and again later in solving Matching problems is the computing of a {\it layering}. Given a connected directed graph $G = (V,E)$ and a source vertex $r \in V$ we are to find a labelling, which we call a layering, on the vertices, $\ell : V \rightarrow \N$ such that $\ell(r) = 0$ and for all $v\neq r$
$$ \ell(v) = 1+ \min_{u:(u,v) \in E} \ell(u).$$
\paragraph{Algorithm}
We can compute a layering as follows:
\begin{enumerate}
\item Initialize $\ell(r) = 0$ and $\ell(v) = \infty$ for all $v \neq r$. Further initialize a queue $W = \{r\}$.
\item While $W\neq \emptyset$
	\begin{enumerate}
	\item Pop $v$ off the head of the queue $W$
	\item Find by Grover's Search all neighbours $w$ with $\ell(w) = \infty$.
	\item Set $\ell(w) = \ell(v) + 1$ and append $w$ to $W$.
	\end{enumerate} 
\end{enumerate}
\begin{theorem}\label{th:layering}
The algorithm above computes a layering for $G=(V,E)$  in time $O(n^{3/2}\log n)$ in the adjacency model and in time $O(\sqrt{nm}\log n)$ in the list model, where $n = |V|$ and $m = |E|$.
\end{theorem}
\begin{proof}
Clearly initialization takes time $O(n)$, and every vertex is processed at most once. In the adjacency model the processing of every vertex takes $O(\sqrt{n})$ by Theorem \ref{th:grovers}. The log-factor appears in accordance with Note \ref{note:log}.
\paragraph{}
In the list model, each vertex is processed in time $$O\big(\sqrt{\frac{n_v}{d(v)}}d_v + \sqrt{d(v) + 1}\big) = O(\sqrt{n_vd(v)} + \sqrt{d_v + 1})$$ where $d(v)$ is the out-degree of $v$, and $n_v$ is the number of neighbours of $v$ with $\ell(v)=\infty$ when $v$ is searched from. Using the Cauchy-Schwarz inequality we can bound the expected running time for all vertices by
$$\sum_{v\in V} \sqrt{n_vd(v)} \leq \sqrt{\sum_{v\in V} n_v}\sqrt{\sum_{v \in V} d(v)} = O(\sqrt{nm}),$$
and similarly for $\sum_{v \in V} \sqrt{d(v) + 1}$. Hence we have the claimed running time.
\end{proof}
\subsection{Integer Network Flow Algorithm}
\paragraph{}
In the {\it Integer Network Flow} problem we are given a directed network $G=(V,E)$ with special vertices source $r$ and sink $s$, and arc capacities $\mu : E \rightarrow \Z_{\geq 0}$ bounded by $U$, i.e. for each $e \in E$, $\mu(e) \leq U$. Our goal is to find the a {\it flow} $x: E \rightarrow \Z_{\geq 0}$ from $r$ to $s$ satisfying capacities $\mu$ of largest value, i.e. we want to solve:
\begin{align*}
\max \sum_{(u,s): E} x(u,s) \\
\text{s.t.} \sum_{(u,v) \in E} x(u,v) - \sum_{(v,u) \in E} x(v,u) &= 0 &\forall v \in V\backslash\{r,s\} \\
0 \leq &x\leq \mu.
\end{align*}
The current best known classical algorithm, given in \cite{goldberg1998beyond}, runs in time $O(\min(n^{2/3}m,m^{3/2})\log U)$.
\paragraph{}
Given a network $G$ and a flow $x$ on $G$, we define the {\it residual} network $H$ as follows. The vertex set of $H$ is the same as that of $G$. Add ``forward" arc $(u,v)$ to $H$ if $ x(u,v) < \mu(u,v)$ and set the capacity of $(u,v)$ in $H$ to be $\mu_H(u,v) = \mu(u,v) - x(u,v)$. Also add (possibly parallel) ``backward" arc $(u,v)$ to $H$ if $x(v,u) > 0$, and set $\mu_H(u,v) = x(v,u)$. Omit any edges with $0$ capacity.
\paragraph{}
A {\it flow-augmenting path} $P$ on $G$ with respect to $x$ is an $r$-$s$ path in residual network $H$. It is called such because $x$ can increased by the minimum capacity on $P$ on each forward arc in $P$ and decreased by the minimum capacity on $P$ on each backward arc to obtain a flow of greater value on $G$. A {\it blocking flow} is a set of flow-augmenting paths on $H$ that hits (has an arc intersecting) every $r$-$s$ path in $H$.
\paragraph{}
In \cite{even1975network} Even and Tarjan present a network flow algorithm when $U=1$ wherein the idea of augmenting the current flow by blocking flows in layered residual networks until the layer depth reaches a certain bound, and then switching to augmenting by single augmenting paths is employed. We will also use this technique in our quantum algorithm. Even and Tarjan give the following lemma which we will employ in our analysis.
\begin{lemma}\label{lemma:5}
\cite{even1975network}\cite{ambainis2006quantum} Let $H$ be the residual network of $G$ with respect to flow $x$. Let $k$ be the depth of an optimal layering on $H$. Then the value of the residual flow is at most $\min((2n/k)^2,m,k)\cdot U$.
\end{lemma}
\begin{proof}
See Lemma $5$ of \cite{ambainis2006quantum}.
\end{proof}
\paragraph{Algorithm}
Let $k = \min(n^{2/3}U^{1/3}\sqrt{mU})$. Our quantum algorithm starting with flow $x=0$:
\begin{enumerate}
\item Compute a layering $\ell$ of the residual graph $H$.
\item If depth of $\ell$ is at most $k$ compute a blocking flow as follows:
	\begin{enumerate}
	\item Mark all vertices as ``enabled".
	\item Compute by depth-first Grover's search an $r$-$s$ path $p$ in $H$ using layering $\ell$ that only uses enabled vertices (quit if no such path). During the back-tracking phase of the search, mark all vertices with no path to $s$ as ``disabled".
	\item Let $u_H$ be the minimum of $\mu_H$ along $p$. Augment $x$ on $G$ by $u_H$ along $p$.
	\item Return to Step $2a$.
	\end{enumerate}
\item Otherwise compute by depth-first Grover's search an $r$-$s$ path $p$ with min capacity $u_H$ (quit if no such path). Augment $x$ on $G$ by $u_H$ along $p$.
\item Return to Step $1$.
\end{enumerate}
\paragraph{}
We present our analysis of the algorithm in the case $U\leq n^{1/4}$ and will show after how to give upper bounds when $U > n^{1/4}$.
\begin{theorem}
Suppose $U \leq n^{1/4}$. Our algorithm solves Integer Network Flow in expected time $$O(n^{13/6}\cdot U^{1/3}\log n)$$ in the adjacency model and expected time $$O(\min(n^{7/6}\sqrt{m}\cdot U^{1/3}, \sqrt{nU}m)\log n)$$ in the list model.
\end{theorem}
\begin{proof}
We can compute $\ell$ in $O(\sqrt{nm}\log n)$ time using Theorem \ref{th:layering}. We compute the capacities of $H$ online in constant time as needed. Since the edges of $H$ are searched online using the layering $\ell$ our algorithm automatically ``deletes" saturated edges when their capacity is queried by Grover's search.
\paragraph{Processing a vertex of the depth-first search}
Let $v \in V$.  Let $a_v$ be the number of augmenting paths using $v$ and let $e_{v,i}$ be the number of outgoing edges $(v,\cdot)$ when $i$ augmenting paths remain with respect to $H$. Since every arc has capacity at most $U$, we have $e_{v,i} \geq \ceil{i/U}$. Thus the total time spent in expectation on Grover's searches at node $v$ of the depth-first search is
$$\sum_{i=1}^{a_v} \sqrt{\frac{d(v)}{e_{v,i}}} \leq \sqrt{U}\cdot\sum_{i=1}^{a_v}\sqrt{d(v)/i} = O(\sqrt{Ua_vd(v)}).$$
Let $c_v$ be the number of enabled vertices found from node $v$ of the depth-first search which do not lie on an augmenting path in $H$, and are hence disabled. In expectation $O(\sqrt{c_vd(v)})$ time is spent in Grover's searches for these vertices. Finally, it takes $O(\sqrt{d(v) + 1})$ time to discover there is no $r$-$s$ path in $H$ using $v$, at which point $v$ is marked ``disabled" and never returned to.
\paragraph{Time to find a blocking flow}
Let $j$ be the depth of layering $\ell$ on $H$ and let $A_j$ be the value of a blocking flow on $H$. Since the total number of augmenting paths through any vertex in a given layer is at most $A_j$, we have $\sum_{v \in V} a_v \leq jA_j$. Further we can observe that $\sum_{v \in V} c_v \leq n$ and $\sum_{v \in V} d(v) \leq m$.  Thus, using Cauchy-Schwarz, the total time in expectation for computing a blocking flow from $H$ is
$$\sum_{v \in V}(\sqrt{Ua_v(d)} + \sqrt{c_vd(v)} + \sqrt{d(v)+1}) \leq \sqrt{U}\sqrt{\sum_{v\in V}a_v}\sqrt{\sum_{v\in V} d(v)} + 2\sqrt{nm}$$
which is on the order of $O(\sqrt{jmA_jU} + \sqrt{nm})$.
\paragraph{}
So our algorithm performs at most $k$ iterations finding a blocking flow. Thus the total expected time before switching to finding single flow-augmenting paths is at most
$$\sqrt{mU}\cdot\sum_{j=1}^k\sqrt{jA_j} + k\sqrt{nm}.$$
\paragraph{Counting remaining flow-augmenting paths}
If the algorithm has not yet terminated after $\ell$ reaches depth greater than $k$ then by Lemma \ref{lemma:5}, $H$ has a residual flow of size $O(\min((2n/k)^2,m,k)\cdot U)=O(k)$. Therefore the algorithm needs at most $O(k)$ more iterations to terminate. Since we now only search for one augmenting path, each iteration has complexity $O(\sqrt{jm}) = O(\sqrt{nm})$. Hence the total running time of our algorithm in expectation is
$$O(\sqrt{mU}\cdot \sum_{j=1}^k\sqrt{jA_j} + k\sqrt{nm}) + O(k\sqrt{nm}).$$
\paragraph{Final Bounding}
We want to show $\sum_{j=1}^k\sqrt{jA_j} = O(k^{3/2})$. We will divide the index sequence $1, \dots, k$ of summands into $\log k$ intervals of the form $S_i := \{2^i, 2^i + 1, \dots, 2^{i+1}-1\}$ of length $2^i$. After $k_i=\frac{k}{2^i}$ iterations the value of the residual flow on $H$ is at most 
$$O(\min((n/k)^2\cdot 2^{2i}, m/k\cdot 2^i)\cdot U) \leq O(2^{2i}k) = O(k^2/k_i)$$
by Lemma \ref{lemma:5}. Using Cauchy-Schwartz on each interval $S_i$ we observe that
\begin{align*}
\sum_{j=1}^k \sqrt{jA_j} &= \sum_{i=0}^{\log k} \sum_{j \in S_i} \sqrt{jA_j}\\
 &\leq \sum_{i=0}^{\log k} \sqrt{2^i\cdot 2^{i+1}}\sqrt{\sum_{j\in S_i} A_j} \\
 &\leq \sqrt{2}\sum_{i=0}^{\log k} 2^i\sqrt{kk_i} \\
 &= O(k^{3/2}).
 \end{align*}
 Hence the total expected running time of our algorithm is $O(k\sqrt{m}(\sqrt{kU} + \sqrt{n}))$. Since $U\leq n^{1/4}$, we have $kU \leq n$. Therefore the running time in the list model is
 $$O(\min(n^{7/6}\sqrt{m}\cdot U^{1/3}, \sqrt{nU})\log n)$$
 with the log factor appearing by Note \ref{note:log}. In the adjacency model the desired runtime appears by setting $m=n^2$ in the above.
\end{proof}
\paragraph{}
When $U > n^{1/4}$ we can obtain a runtime bound of $O(\min(n^{7/6}\sqrt{m},\sqrt{n}m)\cdot U \log n)$ using the same analysis as above, but with a slightly modified algorithm: that is, needing to change $k$ to $\min(n^{2/3}, \sqrt{m})$. The speedup of our quantum network flow algorithm is best felt for dense graphs, i.e. those with many edges. When $m = \Omega(n^{1+\epsilon})$ for some $\epsilon >0$ and $U$ is small our algorithm gives a polynomial speedup over the best known classical algorithm.
\paragraph{}
The authors also present a lower bound via a reduction from deciding the majority on a bit sequence and using the max-flow min-cut theorem. In some sense the gap between the upper and lower bounds remains large. We state without proof their result here.
\begin{theorem}
\cite{ambainis2006quantum} A quantum algorithm for Integer Network Flow with capacities bounded by $U=n$ has quantum query complexity $\Omega(n^2)$.
\end{theorem}