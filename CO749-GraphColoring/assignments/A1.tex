\documentclass[letterpaper,12pt,oneside,onecolumn]{article}
\usepackage[margin=1in, bottom=1in, top=1in]{geometry} %1 inch margins
\usepackage{amsmath, amssymb, amstext}
\usepackage{fancyhdr}
\usepackage{mathtools}
\usepackage{algorithm}
\usepackage{algpseudocode}
\usepackage{theorem}
\usepackage{tikz}
\usepackage{tkz-berge}
\usepackage[braket, qm]{qcircuit}
\usepackage{hyperref}

%Macros
\newcommand{\A}{\mathbb{A}} \newcommand{\C}{\mathbb{C}}
\newcommand{\D}{\mathbb{D}} \newcommand{\F}{\mathbb{F}}
\newcommand{\N}{\mathbb{N}} \newcommand{\R}{\mathbb{R}}
\newcommand{\T}{\mathbb{T}} \newcommand{\Z}{\mathbb{Z}}
\newcommand{\Q}{\mathbb{Q}}
 
 
\newcommand{\cA}{\mathcal{A}} \newcommand{\cB}{\mathcal{B}}
\newcommand{\cC}{\mathcal{C}} \newcommand{\cD}{\mathcal{D}}
\newcommand{\cE}{\mathcal{E}} \newcommand{\cF}{\mathcal{F}}
\newcommand{\cG}{\mathcal{G}} \newcommand{\cH}{\mathcal{H}}
\newcommand{\cI}{\mathcal{I}} \newcommand{\cJ}{\mathcal{J}}
\newcommand{\cK}{\mathcal{K}} \newcommand{\cL}{\mathcal{L}}
\newcommand{\cM}{\mathcal{M}} \newcommand{\cN}{\mathcal{N}}
\newcommand{\cO}{\mathcal{O}} \newcommand{\cP}{\mathcal{P}}
\newcommand{\cQ}{\mathcal{Q}} \newcommand{\cR}{\mathcal{R}}
\newcommand{\cS}{\mathcal{S}} \newcommand{\cT}{\mathcal{T}}
\newcommand{\cU}{\mathcal{U}} \newcommand{\cV}{\mathcal{V}}
\newcommand{\cW}{\mathcal{W}} \newcommand{\cX}{\mathcal{X}}
\newcommand{\cY}{\mathcal{Y}} \newcommand{\cZ}{\mathcal{Z}}

\newcommand\numberthis{\addtocounter{equation}{1}\tag{\theequation}}


\newenvironment{proof}{{\bf Proof:  }}{\hfill\rule{2mm}{2mm}}
\newenvironment{proofof}[1]{{\bf Proof of #1:  }}{\hfill\rule{2mm}{2mm}}
\newenvironment{proofofnobox}[1]{{\bf#1:  }}{}\newenvironment{example}{{\bf Example:  }}{\hfill\rule{2mm}{2mm}}

%\renewcommand{\thesection}{\lecnum.\arabic{section}}
%\renewcommand{\theequation}{\thesection.\arabic{equation}}
%\renewcommand{\thefigure}{\thesection.\arabic{figure}}

\newtheorem{fact}{Fact}[section]
\newtheorem{lemma}[fact]{Lemma}
\newtheorem{theorem}[fact]{Theorem}
\newtheorem{definition}[fact]{Definition}
\newtheorem{corollary}[fact]{Corollary}
\newtheorem{proposition}[fact]{Proposition}
\newtheorem{claim}[fact]{Claim}
\newtheorem{exercise}[fact]{Exercise}
\newtheorem{note}[fact]{Note}
\newtheorem{conjecture}[fact]{Conjecture}

\newcommand{\size}[1]{\ensuremath{\left|#1\right|}}
\newcommand{\ceil}[1]{\ensuremath{\left\lceil#1\right\rceil}}
\newcommand{\floor}[1]{\ensuremath{\left\lfloor#1\right\rfloor}}

\DeclarePairedDelimiter\abs{\lvert}{\rvert}%
\DeclarePairedDelimiter\norm{\lVert}{\rVert}%
%END MACROS
%Page style
\pagestyle{fancy}

\listfiles

\raggedbottom

\lhead{CO 749 - \today}
\rhead{W. Justin Toth - A1} %CHANGE n to ASSIGNMENT NUMBER ijk TO COURSE CODE
\renewcommand{\headrulewidth}{1pt} %heading underlined
%\renewcommand{\baselinestretch}{1.2} % 1.2 line spacing for legibility (optional)

\begin{document}
\section{}
\paragraph{}
Let $G$ be a graph of minimum degree at least $3$ satisfying all of the following:
\begin{itemize}
	\item Every triangle contains at most one vertex of degree three.
	\item If two vertices of degree three are adjacent then they are in triangles.
	\item No two distinct triangles share a common edge.
\end{itemize}
Suppose for a contradiction that $\abs{E(G)} < 5\abs{V(G)}/3$.
\begin{claim}\label{1.1}
	For every $v \in V(G)$, if $d(v) = 3$ then $v$ has at most one neighbour of degree $3$.
\end{claim}
\begin{proof}
	Assume for a contradiction that there exist $u,v, w \in V(G)$ satisfying
	$$d(u) = d(v) = d(w) = 3$$
	and $uv, vw \in E(G)$, i.e. $u$ and $w$ are adjacent to $v$. But then, using the second property of $G$, $v$ is in a triangle. Denote this triangle by $C$. Since every triangle contains at most one vertex of degree three, $u$ and $w$ are not contained in $C$. But then since $d(v) = 3$, $v$ has at most one neighbour in $C$, a contradiction to $C$ being a $K_3$.
\end{proof}
\paragraph{}
We proceed by a discharging argument. For each vertex $v \in V(G)$ set the initial charge as
$$ch_0(v) = d(v) - 4$$
and for each edge $e \in E(G)$ set the initial charge
$$ch_0(e) = 1/3.$$
Then our total charge is, using handshaking,
\begin{align*}
\sum_{v\in V(G)} ch_0(v) + \sum_{e \in E(G)} ch_0(e) &= \sum_{v \in V(G)} d(v) - 4|V(G)| + \frac{1}{3} |E(G)|\\ &=  \frac{7}{3}|E(G)| - 4|V(G)\\ &< \frac{35}{9}|V(G)| - \frac{36}{9}|V(G)|\\ &< 0
\end{align*}
with the first inequality following from our assumption that $|E(G)| < 5|V(G)|/3$.
\paragraph{}
Redistribute the charge, to obtain $ch_1$, according to the following Rules:
\begin{enumerate}
	\item If edge $uv$ contains precisely one vertex $u$ such that $d(u) = 3$ then $uv$ sends $1/3$ charge to $u$.
	\item If edge $uv$ is such that $d(u) \geq 4$ and $d(v) \geq 4$ and $u,v$ are in a triangle with vertex $w$ then $uv$ sends $1/3$ charge to $w$.
\end{enumerate}
\paragraph{}
Since vertices sends no charge, clearly every vertex $v \in V(G)$ with $d(v) \geq 4$ has $$ch_1(v) = ch_0(v) \geq 0.$$ Now we will show that if $v$ is a vertex with $d(v) = 3$ then $ch_1(v) \geq 0$. From Claim \ref{1.1} $v$ is adjacent with at most one vertex of degree $3$. If $v$ is adjacent with no degree three vertices then each edge incident with $v$ sends $\frac{1}{3}$ charge to $v$ by Rule $1$ and hence
$$ch_1(v) = ch_0(v) + 3\cdot\frac{1}{3} = 0.$$
On the other hand, if $v$ is adjacent with one degree three vertex then $v$ receives $\frac{2}{3}$ charge from incident edges by Rule $1$. The remaining needed $\frac{1}{3}$ is obtained from Rule $2$ as follows. Since $v$ is adjacent with a degree three vertices, $v$ lies in a triangle, and since every triangle contains at most one degree three vertex the edge containing the other two vertices of this triangle sends $\frac{1}{3}$ charge to $v$. Therefore $ch_1(v) \geq 0$.
\paragraph{}
It remains to verify that each edge $uv$ satisfies $ch_1(uv) \geq 0$. Edge $uv$ can satisfy at most one of Rule $1$ and Rule $2$. If $uv$ satisfies Rule $1$ it sends $\frac{1}{3}$ charge and so $ch_1(uv) = 0$. Instead if $uv$ satisfies Rule $2$ it can only do so for one triangle, since no two distinct triangles share a common edge. So $uv$ only sends $\frac{1}{3}$ charge by Rule $2$. Therefore $ch_1(uv) \geq 0$.
\paragraph{}
Therefore we have a contradiction as our total charge is negative but $ch_1$ of each object is non-negative.$\blacksquare$

\newpage
\section{}
\paragraph{}
Assume Lemma $2$ is false. Let $G$ be a minimum counterexample. That is $G$ is a smallest planar graph satisfying:
\begin{itemize}
	\item The girth of $G$ is at least seven,
	\item The minimum degree of $G$ is at least two, and
	\item If $v$ is a degree $2$ vertex of $G$ with neighbours $u$ and $w$ then the degrees of $u$ and $w$ are each at least $4$.
\end{itemize}  
\paragraph{}
Since the minimum degree is at least $2$, $G$ is not acyclic. So $G$ contains cycles, and each cycle is of length at least $7$. Hence every bounded face of $G$ is bounded by a cycle of length at least $7$ and the unbounded face is incident with at least $7$ edges.
\paragraph{}
We proceed by a discharging argument. For each vertex $v \in V(G)$ assign initial charge to $v$:
$$ch_0(v) = 2d(v) - 6.$$
For each face $f$ assign initial charge to $f$:
$$ch_0(f) = |f| - 6.$$
Then by our Lemma from class the sum of the initial charges is $-12$. By the girth and minimum degree of $G$, the charge of every face is at least $1$ and the charge of every vertex is non-negative except for degree $2$ vertices.
\paragraph{}
Discharge, establishing final charges $ch_1$ according to the following Rules:
\begin{enumerate}
	\item Each face $f$ sends $1$ charge to each incident vertex of degree $2$.
	\item Each vertex of degree at least $4$ sends $\frac{1}{2}$ charge to each incident face.
\end{enumerate}
\paragraph{}
If a vertex has degree at least $4$ then their initial charge is at least half their degree, so their final charge is non-negative after sending $\frac{1}{2}$ to each incident face. Degree $3$ vertices neither send nor receive charge and so their final charge is $0$.
\paragraph{}
If $v$ is vertex of degree $2$ then $v$ is incident with two faces, and so receives $2$ charge. Thus the final charge of $v$ is
$$ch_1(v) = ch_0(v) + 2 = -2 + 2 = 0.$$
Hence we have shown that the final charge of each vertex is non-negative. It remains to show the final charge of each face is non-negative to achieve the desired contradiction.
\paragraph{}
Let $f$ be a face of $G$. Let $q$ be the number of degree two vertices incident with $f$. Since the neighbours of any degree two vertex cannot be of degree two we observe that
$$q \leq \floor{\frac{|f|}{2}}.$$
Let $p$ be the number of vertices incident with $f$ of degree at least $4$. Since each degree two vertex has both neighbours of degree four, we have
$$p \geq q.$$
We have that $f$ loses $q$ charge by Rule $1$ and gains $\frac{p}{2}$ charge by Rule $2.$ Thus the final charge of $f$ is
$$ch_1(f) = ch_0(f) -q + p \geq |f| -6 -\frac{q}{2} \geq \frac{3}{4}|f| - 6.$$
So the final charge of $f$ is non-negative when $|f| \geq 8$.
\paragraph{}
Therefore we may assume that $|f| = 7$. In this case we may strengthen our bound on $p$ by observing that
$p \geq q+1$
by a parity argument. Indeed if $p = q$ then $f$ is a cycle of vertices alternating degree $2$ and degree $4$, but the number of vertices on $f$ is odd, so this cannot happen. Now we tighten our estimate of the final charge:
$$ch_1(f) \geq |f| - 6 -\frac{q}{2} + 1 \geq \frac{3}{4}\cdot 7 - 6 + 1 \geq 0.$$
Thus we have shown every face has non-negative final charge, a contradiction. $\blacksquare$

\newpage
\section{}
\paragraph{Stronger Lemma}
We will prove the following stronger result. If $G$ is a $2$-connected plane graph with outer cycle $C$ such that $|C| \leq 6$, and at most one degree $2$ vertices on $C$, and each inner vertex has degree at least $3$ then $G$ has either:
\begin{itemize}
	\item At inner face of length at most $4$, or
	\item At inner face $f$ of length $5$ incident with $4$ inner vertices, each of which has an inner neighbour not incident with $f$.
\end{itemize}
\paragraph{Proof}
Suppose not. Let $G$ be a minimum counterexample. Then each inner face of $G$ is of length at least $5$. Furthermore if $f$ is an inner face of $G$ of length $5$ with four degree $3$ inner vertices $v_1,v_2, v_3,v_4$ then there exists some $v_i$ such that $v_i$ has either:
\begin{enumerate}
	\item all of $v_i$'s neighbours are on $f$, or
	\item $v_i$ has a neighbour on $C$.
\end{enumerate}
We claim that the first case cannont happen in a minimum counterexample. So suppose that $f$ is a length $5$ inner face with vertices $v_1,v_2, v_3, v_4, v_5$ appearing conesecutively in that order about the cycle bounding $f$, and $v_1, \dots, v_4$ have degree $3$, and without loss of generality $v_1$ is adjacent to $v_3$ or $v_4$. 
\paragraph{}
If $v_1$ is adjacent to $v_3$ then since $v_1, v_2$, and $v_3$ are of degree $3$, then $v_2$ is a cut vertex. Since $G$ is plane $v_2$'s third neighbour $v$ is not on $f$ but drawn in a disc bounded by $v_1,v_2,v_3$. But then the only paths to $v_1$ from $v$ use $v_2$, hence $v_2$ is a cut vertex, contradiction $2$-connectedness.
\paragraph{}
If $v_1$ is adjacent to $v_4$ then the subgraph $H$ consisting of the cycle bounding $f$ and the edge $v_1v_4$ is a smaller counterexample than $G$, contradicting minimality. Indeed the outer cycle if $H$ is $v_1v_4v_5$ of length $3 \leq 6$ with one degree $2$ vertex: $v_5$. The subgraph $H$ is $2$-connected, and the minimum degree among inner vertices is $3$.
\paragraph{}
Thus we have shown if $f$ is an inner face of $G$ of length $5$ with four degree $3$ inner vertices $v_1,v_2, v_3,v_4$ then there exists some $v_i$ such that $v_i$ has a neighbour on $C$.
\paragraph{}
We proceed by a discharging argument. For each vertex $v \in V(G)$ set the initial charge $ch_0$ to be
$$ch_0(v) = 2d(v) -6$$
and for each face $f$ set the initial charge to be
$$ch_0(f) = |f| - 6.$$
Then by our Lemma from class the sum of the initial charges is $-12$. We reassign charge, obtaining final charges $ch_1$ according to the following Rules:
\begin{enumerate}
	\item Each inner vertex of degree at least $4$ sends $\frac{1}{2}$ charge to each incident (inner) face.
	\item Each vertex on $C$ of degree at least $3$ sends $\frac{1}{2}$ charge to each incident inner face of length $5$.
	\item If $f$ is an inner face of length $5$ with an incident inner vertex $v$ of degree $3$ such that $v$ is adjacent to vertex $u$ on $C$ then $u$ sends $\frac{1}{2}$ charge to $f$.
\end{enumerate}
\paragraph{}
Each inner vertex of degree $3$ sends no charge and so its final charge is non-negative. If $v$ is an inner vertex of degree at least $4$ then the final charge of $v$ satisfies
$$ch_1(v) \geq ch_0(v) - \frac{d(v))}{2} = \frac{3}{2}d(v) - 6 \geq 0.$$
Therefore each inner vertex has non-negative final charge.
\paragraph{}
Each face sends no charge, so each inner face of length at least $6$ immediately has non-negative final charge. Consider an inner face $f$ of length $5$ with bounding cycle consisting of vertices $v_1,v_2,v_3,v_4,v_5$. Since $ch_0(f) = -1$ we want to show $f$ receives at least $1$ charge. If $f$ is incident with at least $2$ inner vertices of degree at least $4$ then $f$ receives at least $1$ charge by Rule $1$ and so $ch_1(f) \geq 0$.
\paragraph{}
So we may assume $f$ is incident with at most $1$ inner vertex at of degree at least $4$. First suppose, without loss of generality, $v_5$ is the only inner vertex incident with $f$ of degree at least $4$. So $f$ receives $\frac{1}{2}$ charge from $v_5$. Observe that if $f$ shares a vertex with $C$, it necessarily shares a degree $3$ vertex. Then either 
\begin{enumerate}
	\item one of $v_1,\dots, v_4$ is on $C$ and of degree at least $3$, or
	\item all of $v_1, \dots, v_4$ are inner vertices of degree $3$.
\end{enumerate}
In the first case $f$ receives $\frac{1}{2}$ from the vertex on $C$ by Rule $2$ and hence $ch_1(f) \geq 0$. In the second case there exists $v_i$ with $i \in \{1,\dots, 4\}$, such that $v_i$ is adjacent to $u$ on $C$. By Rule $3$, $f$ receives $\frac{1}{2}$ charge from $u$. Therefore $ch_1(f) \geq 0$. Thus in either case the final charge of $f$ is non-negative.
\paragraph{}
Now we may assume that $f$ is incident with no inner vertices of degree at least $4$. Now among $v_1, \dots, v_4$ there exists $i \in \{1,\dots, 4\}$ such that either:
\begin{itemize}
	\item $v_i$ is on $C$ of degree at least $3$, or
	\item $v_i$ is adjacent with a vertex $u$ on $C$
\end{itemize}
In the former case $f$ receives $\frac{1}{2}$ from $v_i$ by Rule $2$and in the latter $f$ receives $\frac{1}{2}$ charge from $u$. Then among $\{v_1, \dots, v_5\} \backslash \{v_i\}$ there exists $j \in \{1,\dots, 5\}\backslash\{i\}$ such that either:
\begin{itemize}
	\item $v_j$ is on $C$ of degree at least $3$, or
	\item $v_j$ is adjacent with a vertex $v$ on $C$
\end{itemize}
In the former case $f$ receives $\frac{1}{2}$ from $v_j$ by Rule $2$and in the latter $f$ receives $\frac{1}{2}$ charge from $v$. Finally, $f$ receives at least $1$ charge total, and hence $ch_1(f) \geq 0$.
\paragraph{}
Thus each inner vertex and each inner face has non-negative final charge. The vertex on $C$ of degree $2$ has final charge $-2$. We lower bound the charge of a vertex $v$ on $C$ of degree at least $3$. The vertex $v$ gives away at most $\frac{1}{2}(d(v) - 1)$ charge by Rule $2$ and at most $\frac{1}{2}(d(v) - 2)$ charge by Rule $3$. So the final charge of $v$ is
$$ch_1(v) \geq 2d(v) - 6 - d(v) + \frac{3}{2} = d(v) - \frac{9}{2} \geq \frac{6}{2} - \frac{9}{2} = -\frac{3}{2}.$$
Thus, using $|C| \leq 6$, our total final charge is at least
$$|C| - 6 -\frac{3}{2}(|C| -1) - 2 = -\frac{1}{2}|C| - 6 -\frac{1}{2} \geq -3 -6 -\frac{1}{2} > -12$$
But since no charge is added or lost during this process, this contradicts that the sum of charges is $-12$. $\blacksquare$

\newpage
\section{}
\paragraph{Stronger Result}
We prove the same Lemma with a better lower bound: $$|V(G) \backslash V(C)| \geq 8|V(C)| - 48.$$ 
\paragraph{Proof}
Suppose not. Let $G$ be a minimum counterexample. Then $G$ is a $2$-connected plane graph with outer cycle $C$ such that $|V(G)\backslash V(C)| < 8|V(C)| - 48$ and $|V(G) \backslash V(C)| \geq 1$ satisfying
\begin{itemize}
	\item Every inner vertex has degree at least $6$, and
	\item Every inner vertex of degree $6$ has one of the following:
	\begin{itemize}
		\item An incident inner face of length at least $4$,
		\item An inner neighbour of degree at least $7$, or
		\item A neighbour on $C$.
	\end{itemize}
\end{itemize}
\paragraph{}
We begin by observing that each $v \in V(C)$ has $d(v) \geq 3$. Indeed suppose there exists some $v \in V(C)$ such that $d(v) = 2$. Suppressing $v$ leads to a smaller counterexample than $G$, a contradiction. This follows since suppressing $v$ does not change the degree of inner vertices or the length of faces incident only with inner vertices.
\paragraph{} 
We proceed by a discharging argument. For each $v \in V(G)$ set the initial charge of $v$ to be
$$ch_0(v) = d(v) - 6$$
and for each face $f$ of $G$ set the initial charge of $f$ to be
$$ch_0(f) = 2|f| - 6.$$
Then by our Lemma from class the total sum of the charges is $-12$. We discharge according to the following Rules:
\begin{enumerate}
	\item Each inner face of length at least $4$ sends $\frac{1}{8}$ charge to each incident vertex.
	\item Each inner vertex of degree at least $7$ sends $\frac{1}{8}$ charge to each adjacent vertex.
	\item Each vertex on $C$ of degree at least $4$ sends $\frac{1}{8}$ charge to each adjacent vertex.
\end{enumerate}
\paragraph{}
Each length $3$ face sends no charge and so has non-negative final charge. Each face $f$ of length at least $4$ sends at most $\frac{1}{8}|f|$ charge so its final charge is
$$ch_1(f) \geq ch_0(f) -\frac{1}{8}|f| \geq \frac{15}{8}|f| - 6 \geq\frac{15}{2} - 6 \geq 0.$$
Hence every face has non-negative charge.
\paragraph{}
We argue that every inner vertex has at least $\frac{1}{8}$ charge. Let $v$ be an inner vertex. If $d(v) \geq 7$ then $v$ sends $\frac{1}{8}d(v)$ charge. So the final charge of $v$ is 
$$ch_1(v) = \frac{7}{8}d(v) - 6 \geq \frac{49}{8} - \frac{48}{8}  = \frac{1}{8}.$$
Now suppose $d(v) = 6.$ Since $ch_0(v) = 0$ we need $v$, which sends no charge, to receive $\frac{1}{8}$ charge. We proceed by case analysis. If $v$ has an incident inner face of length at least $4$ then $v$ receives $\frac{1}{8}$ charge by Rule $1.$ Similarly if $v$ has an incident inner neighbour of degree at least $7$ then $v$ receives $\frac{1}{8}$ charge by Rule $2.$
\paragraph{}
So we may assume that every incident face to $v$ is a triangle, and every inner neighbour to $v$ is of degree $6$. Then we know $v$ is adjacent to a vertex $v_1$ on $C$. Label the neighbours of $v$ as $v_1, \dots, v_6$ in clockwise order from $v_1$ with respect to their drawing in $G$. If $d(v_1) \geq 4$ then $v$ receives $\frac{1}{8}$ charge from $v_1$. Otherwise $d(v_1) = 3$. Since every incident face to $v$ is a triangle this implies $v_2$ is on $C$.  Again, if $d(v_2) \geq 4$ then $v$ receives $\frac{1}{8}$ charge from $v_2$ and otherwise $d(v_2) = 3$. Repeating this argument for each $i \in \{3, \dots, 6\}$ we see that either $v_i$ sends $v$ $\frac{1}{8}$ charge or $d(v_i) = 3$ and $v_{i+1}$ (when $v_{i+1}$ exists) is on $C$. If we complete this process with $v$ receiving no charge then each neighbour of $v$ is a degree $3$ vertex on $C$. That is $G = G[\{v,v_1, \dots, v_6\}]$. But that is a contradiction since it would imply
$$1 = |V(G) \backslash V(C)|  < 8|V(C)| - 48 = 8\cdot 6 - 48 = 0$$
i.e. $1 < 0$.
\paragraph{}
Hence we have shown each inner vertex has final charge at least $\frac{1}{8}$. Now we claim each vertex $v$ on $C$ has final charge at least $-3$. Indeed if $d(v) = 3$ then $v$ sends no charge, and $ch_1(v) = ch_0(v) = -3$ as desired. Otherwise $d(v) \geq 4$ and $v$ sends $\frac{1}{8}d(v)$ charge to each adjacent vertex. Thus the final charge is
$$ch_1(v) \geq \frac{7}{8}d(v) - 6 \geq \frac{7}{2} - 6 > -3.$$
So our total charge is at least the charge of $C$ plus $-3$ times the number of vertices on $C$ plus $\frac{1}{8}$ the number of inner vertices. That is
$$-12 \geq \frac{1}{8}|V(G)\backslash V(C)| + 2|V(C)| - 6 -3|V(C)|.$$
Rearranging we see that
$$8|V(C)| - 48 \geq |V(G) \backslash V(C)|$$
contradicting that $|V(G)\backslash V(C)| < 8|V(C)| - 48.$ $\blacksquare$
\end{document}
