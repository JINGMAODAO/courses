\documentclass[letterpaper,12pt,oneside,onecolumn]{article}
\usepackage[margin=1in, bottom=1in, top=1in]{geometry} %1 inch margins
\usepackage{amsmath, amssymb, amstext}
\usepackage{fancyhdr}
\usepackage{mathtools}
\usepackage{algorithm}
\usepackage{algpseudocode}
\usepackage{theorem}
\usepackage{tikz}
\usepackage{tkz-berge}
\usepackage[braket, qm]{qcircuit}
\usepackage{hyperref}

%Macros
\newcommand{\A}{\mathbb{A}} \newcommand{\C}{\mathbb{C}}
\newcommand{\D}{\mathbb{D}} \newcommand{\F}{\mathbb{F}}
\newcommand{\N}{\mathbb{N}} \newcommand{\R}{\mathbb{R}}
\newcommand{\T}{\mathbb{T}} \newcommand{\Z}{\mathbb{Z}}
\newcommand{\Q}{\mathbb{Q}}
 
 
\newcommand{\cA}{\mathcal{A}} \newcommand{\cB}{\mathcal{B}}
\newcommand{\cC}{\mathcal{C}} \newcommand{\cD}{\mathcal{D}}
\newcommand{\cE}{\mathcal{E}} \newcommand{\cF}{\mathcal{F}}
\newcommand{\cG}{\mathcal{G}} \newcommand{\cH}{\mathcal{H}}
\newcommand{\cI}{\mathcal{I}} \newcommand{\cJ}{\mathcal{J}}
\newcommand{\cK}{\mathcal{K}} \newcommand{\cL}{\mathcal{L}}
\newcommand{\cM}{\mathcal{M}} \newcommand{\cN}{\mathcal{N}}
\newcommand{\cO}{\mathcal{O}} \newcommand{\cP}{\mathcal{P}}
\newcommand{\cQ}{\mathcal{Q}} \newcommand{\cR}{\mathcal{R}}
\newcommand{\cS}{\mathcal{S}} \newcommand{\cT}{\mathcal{T}}
\newcommand{\cU}{\mathcal{U}} \newcommand{\cV}{\mathcal{V}}
\newcommand{\cW}{\mathcal{W}} \newcommand{\cX}{\mathcal{X}}
\newcommand{\cY}{\mathcal{Y}} \newcommand{\cZ}{\mathcal{Z}}

\newcommand\numberthis{\addtocounter{equation}{1}\tag{\theequation}}


\newenvironment{proof}{{\bf Proof:  }}{\hfill\rule{2mm}{2mm}}
\newenvironment{proofof}[1]{{\bf Proof of #1:  }}{\hfill\rule{2mm}{2mm}}
\newenvironment{proofofnobox}[1]{{\bf#1:  }}{}\newenvironment{example}{{\bf Example:  }}{\hfill\rule{2mm}{2mm}}

%\renewcommand{\thesection}{\lecnum.\arabic{section}}
%\renewcommand{\theequation}{\thesection.\arabic{equation}}
%\renewcommand{\thefigure}{\thesection.\arabic{figure}}

\newtheorem{fact}{Fact}[section]
\newtheorem{lemma}[fact]{Lemma}
\newtheorem{theorem}[fact]{Theorem}
\newtheorem{definition}[fact]{Definition}
\newtheorem{corollary}[fact]{Corollary}
\newtheorem{proposition}[fact]{Proposition}
\newtheorem{claim}[fact]{Claim}
\newtheorem{exercise}[fact]{Exercise}
\newtheorem{note}[fact]{Note}
\newtheorem{conjecture}[fact]{Conjecture}

\newcommand{\size}[1]{\ensuremath{\left|#1\right|}}
\newcommand{\ceil}[1]{\ensuremath{\left\lceil#1\right\rceil}}
\newcommand{\floor}[1]{\ensuremath{\left\lfloor#1\right\rfloor}}

\DeclarePairedDelimiter\abs{\lvert}{\rvert}%
\DeclarePairedDelimiter\norm{\lVert}{\rVert}%
%END MACROS
%Page style
\pagestyle{fancy}

\listfiles

\raggedbottom

\lhead{\today}
\rhead{W. Justin Toth - A4} %CHANGE n to ASSIGNMENT NUMBER ijk TO COURSE CODE
\renewcommand{\headrulewidth}{1pt} %heading underlined
%\renewcommand{\baselinestretch}{1.2} % 1.2 line spacing for legibility (optional)

\begin{document}
\section{}
\paragraph{}
Let $v(G)$ and $e(G)$ denote the number of vertices and edges respectively of a graph $G$. Also let $c(G)$ denote the number of connected components of $G$. Let $\phi$ be a $k$-coloring of $G$. Define $\cB(G, \phi)$ so that
$$\cB(G,\phi) = \{G[\phi^{-1}(c) \cup \phi^{-1}(c')] : c \in \{1,\dots, k\}, c'\in \{1,\dots, k\}\backslash\{c\}\}.$$
I.e. $\cB(G,\phi)$ is the set of bipartite subgraphs of $G$ induced by two distinct color classes under $\phi$.
\begin{lemma}\label{lemma:bipartite-cc}
For all $k \geq 3$, for all $k$-colorable graphs $G$, if $\phi$ is a $k$-coloring of $G$ then
$$\sum_{H \in \cB(G,\phi)} c(H) \geq (k-1)v(G) - e(G).$$
\end{lemma}
\begin{proof}
Fix $k \geq 3$ and fix $n \geq k$. Let $G$ be a graph with $v(G) = n$. We proceed by induction on $e(G)$. First suppose $e(G) = 0$. Let $\phi$ be a $k$-coloring of $G$. If $v \in V(G)$ then $v$ is assigned a color class $c$ and hence is a vertex of each graph $G[\phi^{-1}(c) \cup \phi^{-1}(c')]$ for $c' \neq c$. There are $k-1$ such graphs, and since $e(G) = 0$, $v$ itself is a unique connected component of each such graph. Therefore
$$\sum_{H \in \cB(G,\phi)} c(H)= \sum_{v \in V(G)} (k-1) = (k-1)v(G) - e(G).$$
Now suppose $e(G) >0$ and the lemma holds for $k$-colorable, $n$ vertex graphs with fewer than $e(G)$ edges. Again let $\phi$ be a $k$-coloring of $G$. Let $uv \in E(G)$. Let $c = \phi(u)$ and $c' = \phi(v)$. Let $H' := G[\phi^{-1}(c) \cup \phi^{-1}(c')]$. Then $H'$ is the unique graph in $\cB(G,\phi)$ containing edge $uv$. If $uv$ is a bridge of its connected component in $H'$ then $c(H'-uv) = c(H') + 1$, otherwise $c(H'-uv) = c(H')$. Hence $c(H'-uv) \leq c(H')+1$. So
\begin{align*}
\sum_{H \in \cB(G,\phi)} c(H) &= c(H') + \sum_{H \in \cB(G, \phi)\backslash\{H'\}} c(H)\\
&= c(H') +  \sum_{H \in \cB(G-uv, \phi)\backslash\{H'\}} c(H) &\text{(since $uv$ is only an edge of $H'$)}\\ 
&\geq c(H'-uv)-1 +  \sum_{H \in \cB(G-uv, \phi)\backslash\{H'\}} c(H)\\
&=  -1 + \sum_{H \in \cB(G-uv, \phi)} c(H)\\
&\geq - 1 + (k-1)v(G) - e(G) + 1 &\text{(by induction hypothesis on $G-uv$)}\\
&= (k-1)v(G) - e(G).
\end{align*}
Therefore by induction the lemma holds.
\end{proof}
\begin{lemma}
Let $k\geq 3$. If $G$ is a $k$-colorable graph then $G$ has at least $2^{\frac{(k-1)v(G) - e(G)}{{k \choose 2}}}$ distinct $k$-colorings.
\end{lemma}
\begin{proof}
Let $\phi$ be a $k$-coloring of $G$. Let $H = G[\phi^{-1}(c) \cup \phi^{-1}(c')] \in \cB(G,\phi)$. We can obtain a distinct $k$-coloring of $G$ by choosing any number of connected components $H_1, \dots, H_k$ of $H$ and switching colors $c$ and $c'$ on the vertices of $H_1, \dots, H_k$. Thus in this manner we can obtain $2^{c(H)}$ distinct $k$-colorings of $G$. It will then suffice to show that there exists $H \in \cB(G,\phi)$ such that
$$c(H) \geq \frac{(k-1)v(G) - e(G)}{{k \choose 2}}.$$
Since there are ${k\choose 2}$ graphs in $\cB(G,\phi)$, by the pigeonhole principle, there exists $H' \in \cB(G,\phi)$ such that
$$c(H') \geq \frac{\sum_{H \in \cB(G,\phi)} c(H)}{{k\choose 2}}$$
Therefore by Lemma \ref{lemma:bipartite-cc} 
$$c(H') \geq \frac{(k-1)v(G) - e(G)}{{k\choose 2}}$$
as desired.
\end{proof}
\end{document}
