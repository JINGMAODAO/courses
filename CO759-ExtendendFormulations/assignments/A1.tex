\documentclass[letterpaper,12pt,oneside,onecolumn]{article}
\usepackage[margin=1in, bottom=1in, top=1in]{geometry} %1 inch margins
\usepackage{amsmath, amssymb, amstext}
\usepackage{fancyhdr}
\usepackage{mathtools}
\usepackage{algorithm}
\usepackage{algpseudocode}
\usepackage{theorem}
\usepackage{tikz}
\usepackage{tkz-berge}

%Macros
\newcommand{\A}{\mathbb{A}} \newcommand{\C}{\mathbb{C}}
\newcommand{\D}{\mathbb{D}} \newcommand{\F}{\mathbb{F}}
\newcommand{\N}{\mathbb{N}} \newcommand{\R}{\mathbb{R}}
\newcommand{\T}{\mathbb{T}} \newcommand{\Z}{\mathbb{Z}}
\newcommand{\Q}{\mathbb{Q}}
 
 
\newcommand{\cA}{\mathcal{A}} \newcommand{\cB}{\mathcal{B}}
\newcommand{\cC}{\mathcal{C}} \newcommand{\cD}{\mathcal{D}}
\newcommand{\cE}{\mathcal{E}} \newcommand{\cF}{\mathcal{F}}
\newcommand{\cG}{\mathcal{G}} \newcommand{\cH}{\mathcal{H}}
\newcommand{\cI}{\mathcal{I}} \newcommand{\cJ}{\mathcal{J}}
\newcommand{\cK}{\mathcal{K}} \newcommand{\cL}{\mathcal{L}}
\newcommand{\cM}{\mathcal{M}} \newcommand{\cN}{\mathcal{N}}
\newcommand{\cO}{\mathcal{O}} \newcommand{\cP}{\mathcal{P}}
\newcommand{\cQ}{\mathcal{Q}} \newcommand{\cR}{\mathcal{R}}
\newcommand{\cS}{\mathcal{S}} \newcommand{\cT}{\mathcal{T}}
\newcommand{\cU}{\mathcal{U}} \newcommand{\cV}{\mathcal{V}}
\newcommand{\cW}{\mathcal{W}} \newcommand{\cX}{\mathcal{X}}
\newcommand{\cY}{\mathcal{Y}} \newcommand{\cZ}{\mathcal{Z}}

\newcommand\numberthis{\addtocounter{equation}{1}\tag{\theequation}}


\newenvironment{proof}{{\bf Proof:  }}{\hfill\rule{2mm}{2mm}}
\newenvironment{proofof}[1]{{\bf Proof of #1:  }}{\hfill\rule{2mm}{2mm}}
\newenvironment{proofofnobox}[1]{{\bf#1:  }}{}\newenvironment{example}{{\bf Example:  }}{\hfill\rule{2mm}{2mm}}

%\renewcommand{\thesection}{\lecnum.\arabic{section}}
%\renewcommand{\theequation}{\thesection.\arabic{equation}}
%\renewcommand{\thefigure}{\thesection.\arabic{figure}}

\newtheorem{fact}{Fact}[section]
\newtheorem{lemma}[fact]{Lemma}
\newtheorem{theorem}[fact]{Theorem}
\newtheorem{definition}[fact]{Definition}
\newtheorem{corollary}[fact]{Corollary}
\newtheorem{proposition}[fact]{Proposition}
\newtheorem{claim}[fact]{Claim}
\newtheorem{exercise}[fact]{Exercise}
\newtheorem{note}[fact]{Note}
\newtheorem{conjecture}[fact]{Conjecture}

\newcommand{\size}[1]{\ensuremath{\left|#1\right|}}
\newcommand{\ceil}[1]{\ensuremath{\left\lceil#1\right\rceil}}
\newcommand{\floor}[1]{\ensuremath{\left\lfloor#1\right\rfloor}}
\newcommand{\conv}[1]{\ensuremath{\text{conv}(#1) }}
\newcommand{\cone}[1]{\ensuremath{\text{cone}(#1) }}
\newcommand{\facets}[1]{\ensuremath{\text{facets}(#1) }}
%END MACROS
%Page style
\pagestyle{fancy}

\listfiles

\raggedbottom

\lhead{\today}
\rhead{W. Justin Toth. Extended Formulations - A1} %CHANGE n to ASSIGNMENT NUMBER ijk TO COURSE CODE
\renewcommand{\headrulewidth}{1pt} %heading underlined
%\renewcommand{\baselinestretch}{1.2} % 1.2 line spacing for legibility (optional)

\begin{document}
%Problem 1
\section{}
\paragraph{}
Let a polytope $Q\subseteq \R^n$ and a linear map $p:\R^n \rightarrow \R^d$ be an extension of a polytope $P\subseteq \R^d$. Since $p$ is a linear map, there exists a matrix $A \in \R^{d\times n}$ such that $p(x) = Ax$ for all $x \in Q$.
\paragraph{}
Let $k \in \Z$ such that $0\leq k \leq d$. We will show the existence of an injective map $f$ from the faces of $P$ of dimension $k$ to the faces of $Q$ of dimension at least $k$. This suffices to show the desired result.
\paragraph{}
Let $F$ be a face of $P$ of dimension $k$ such that $F = \{x \in P: c^Tx = \beta\}$. We define the map $f$ so that
$$f(F) = \{x \in Q: (c^TA)x = \beta\}.$$
We need to show that $f$ is well-defined, and that it is injective. To this end we begin by showing that $f(F)$ is a face of $Q$. Let $x \in Q$. Then
$$(c^TA)x = c^T(Ax) \leq \beta$$
with the last inequality following since $p(x) = Ax$ is in $P$. Note that equality holds for all $x$ such that $Ax$ is in $F$. Hence $f(F)$ is a face of $Q$. Since $(Q,p)$ is an extension of $P$, for all $y \in F$ there exists $x \in Q$ such that $Ax = y$. Thus $F \subseteq p(f(F))$, and using that $p$ is a linear map we see that, 
$$\text{dim}(f(F)) \geq \dim F = k.$$
\paragraph{}
Now suppose that $F$ and $F'$ are $k$-dimensional faces of $P$ such that $f(F) = f(F')$. Suppose $F = \{x \in P: c^Tx = \beta\}$ and $F' = \{x \in P: (c')^Tx = \beta\}$. Let $y \in F$. Then there exists $x \in f(F)$ such that $Ax = y$. Since $f(F) = f(F')$,
$$\beta = (c')^TAx = (c')^Ty $$
and hence $y \in F$. Thus $F \subseteq F'$ and by a symmetric argument we can see that $F' \subseteq F$. Thus $F = F'$ and hence $f$ is an injective map, as desired. $\blacksquare$
%Problem 2
\section{}
\paragraph{}
The claim is \emph{false}. Let $P = \conv{\{(0,0),(1,0),(1,1),(0,1)\}}$. Then $P$ is a square, so $|\text{facets}(P)| = 4$. Let $Q = \conv{\{(0,0,0),(1,0,0), (1,1,0), (0,1,1)\}}$, and let $p: \R^3\rightarrow \R^2$ be the projection map onto the first two coordinates (thus $p$ is linear). Then $Q$ is a triangular pyramid with the ``top" vertex skewed. so $|\text{facets}(Q)| = 4$. Clearly $p(Q) = P$ and thus $(Q,p)$ is an extension of $P$ with the same number of facets. But $p$ not a linear isomorphism. Observe that $\frac{1}{3}(1,0,0) + \frac{1}{3}(0,1,1) + \frac{1}{3}(0,0,0) = \frac{1}{3}(1,1,1) \in Q$ and $ \frac{1}{3}(1,1,0) + \frac{2}{3}(0,0,0) = \frac{1}{3}(1,1,0) \in Q$ have the same projection:
$$p(\frac{1}{3}(1,1,1)) = p(\frac{1}{3}(1,1,0)) = \frac{1}{3}(1,1) \in P$$
and so $p$ is not injective and thus not an isomorphism.$\blacksquare$
%Problem 3
\section{}
\paragraph{}
Let $Q \subseteq \R^n$ be a polyhedron and $p:\R^n \rightarrow \R^d$ a linear map, so that $(Q,p)$ is an extension of a polytope $P \subseteq \R^d$. If $Q$ is bounded, we are done, so we may assume that $Q$ is unbounded. Let $R$ be the recession cone of $Q$. We claim that $p(R) = 0$, for otherwise if $r \in R$ with $p(r) \neq 0$ then $p(r)$ is an unbounded direction in $P$ as $p(\lambda r) = \lambda p(r) \neq 0$ for all $\lambda  > 0$, contradicting that $P$ is a polytope. Let $L = R \cap (-R)$ be the lineality space of $Q$. By our previous claim $p(L) = 0$. We have that
$$Q = L + (Q\cap L^\perp)$$
where $L^\perp$ denotes the orthogonal complement of $L$. Thus
$$P = p(Q) = p(L) + p(Q\cap L^\perp) = p(Q\cap L^\perp).$$
Let $Q_1$ denote $Q\cap L^\perp$, so we have shown $p(Q_1) = P$, that is $(Q_1, p)$ is an extension of $p$. Since $Q_1$ was obtained from $Q$ by adding only equalities, $Q_1$ has no more facets than $Q$. Our choice of $Q_1$ ensures that $Q_1$ has a trivial lineality space, i.e. $Q_1$ is pointed. If $Q_1$ is bounded, we are done, so we may assume that $Q_1$ is unbounded. Let $R_1$ be the recession cone of $Q_1$ and $V_1$ be the vertices of $Q_1$. By our prior arguments neither $R_1$ nor $V_1$ are empty.
\paragraph{}
Let $c \in \R^n$ be a vector such that $c^Tr \geq 1$ for all $r \in R_1$. Such a $c$ exists since $R_1 \neq \emptyset$ and the lineality space of $Q_1$ is empty (clearly $c$ so that $c^Tr >0$ exists, and we simply scale $c$). Let $\beta \in \R$ be chosen so that $c^Tv < \beta$ for all $v \in V_1$. Since $V_1$ is finite, such $\beta$ exists. We choose $Q_2$ as
$$Q_2 = Q_1 \cap \{x \in \R^n : c^Tx = \beta\}.$$
We claim that $Q_2$ is bounded and that $p(Q_2) = P$. Let $\alpha = \min\{c^Tv : v \in V_1\}$. Let $x \in Q_2$. There exist $\lambda \geq 0$, $\sum_{v \in V_1} \lambda_v = 1$ and $\gamma \geq 0$ such that
$$x = \sum_{v \in V_1} \lambda_v v + \sum_{r \in R_1} \gamma_r r_1.$$
So
$$\beta = c^Tx =  \sum_{v \in V_1} \lambda_v c^Tv + \sum_{r \in R_1} \gamma_r c^Tr_1 \geq \alpha + \sum_{r \in R_1} \gamma_r c^Tr_1\geq \alpha + \sum_{r \in R_1} \gamma_r$$
and thus $$\sum_{r \in R_1} \gamma_r \leq \beta -\alpha.$$
Hence our choice of coefficients $\gamma$ on the vectors of $R_1$ is bounded, and so $Q_2$ is bounded.
\paragraph{}
It remains to verify that $p(Q_2) = P$. Similarly to the argument for $R$, we see that $p(R_1) = 0$. Hence $P = p(Q_1) = p(\conv{V_1}).$ So it will suffice to show that for every $x \in \conv{V_1}$ there exists $\lambda \geq 0$ and $r \in R_1$ such that $c^T(x + \lambda r) = \beta.$ Let $r$ be any vector in $R_1$ (since $R_1 \neq \emptyset$ this choice is well-defined). Choose $\lambda = (\beta - c^Tx)/c^Tr$. Since $\beta-c^Tx >0$ and $c^Tr >0$ by our choice of $\beta$ and $c$, we have $\lambda \geq 0$. Now we have
$$c^T(x + \lambda r) = c^Tx + \beta - c^Tx = \beta$$
as desired. $\blacksquare$
\section{}
\paragraph{}

%Problem 5
\section{}
\paragraph{}

%Problem 6
\section{}
\paragraph{}
\end{document}
